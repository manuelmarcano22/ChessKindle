\documentclass{book}
\usepackage[ps,mover]{skak}
\usepackage{xskak}
\begin{document}
\parindent=0pt 
\tableofcontents
\chapter{0}
\section{1.??.??}
\fenboard{r6r/1pp3k1/1b6/p2P1p2/P1N1pn2/2P2PP1/BP5P/4RR1K b - - 0 1}
\begin{center}
\showinverseboard 
\end{center}
\clearpage 
\newpage 
\mainline{1... Rxh2+ $1 2. Kxh2 Rh8# { mate * Black would have been lost without this resource. }}\fenboard{r6r/1pp3k1/1b6/p2P1p2/P1N1pn2/2P2PP1/BP5P/4RR1K b - - 0 1}
\begin{center}
\showinverseboard 
\end{center}
\fenboard{r6r/1pp3k1/1b6/p2P1p2/P1N1pn2/2P2PP1/BP5P/4RR1K b - - 0 1}
\mainline[level=1]{1... Rxh2+ $1 2. Kxh2 Rh8# }
 mate * Black would have been lost without this resource. 




\clearpage 
\newpage 

\section{2.??.??}
\fenboard{rnb3kr/ppp4p/3b3B/3Pp2n/2BP4/3K1Rp1/PPP3q1/RN1Q4 w - - 0 1}
\begin{center}
\showboard 
\end{center}
\clearpage 
\newpage 
\mainline{1. Rf8+ $1 Bxf8 2. d6+ Be6 3. Bxe6# { mate * }}\fenboard{rnb3kr/ppp4p/3b3B/3Pp2n/2BP4/3K1Rp1/PPP3q1/RN1Q4 w - - 0 1}
\begin{center}
\showboard 
\end{center}
\fenboard{rnb3kr/ppp4p/3b3B/3Pp2n/2BP4/3K1Rp1/PPP3q1/RN1Q4 w - - 0 1}
\mainline[level=1]{1. Rf8+ $1 Bxf8 2. d6+ Be6 3. Bxe6# }
 mate * 




\clearpage 
\newpage 

\section{3.??.??}
\fenboard{r2q1rk1/pppb1ppp/3b4/4p1P1/4Pn2/2N1B2P/PPPQBP2/2KR3R w - - 0 1}
\begin{center}
\showboard 
\end{center}
\clearpage 
\newpage 
\mainline{1. Bxf4 $1 exf4 2. e5 { * which won a piece. }}\fenboard{r2q1rk1/pppb1ppp/3b4/4p1P1/4Pn2/2N1B2P/PPPQBP2/2KR3R w - - 0 1}
\begin{center}
\showboard 
\end{center}
\fenboard{r2q1rk1/pppb1ppp/3b4/4p1P1/4Pn2/2N1B2P/PPPQBP2/2KR3R w - - 0 1}

 The bishop on d6 is pinned and Steinitz took advantage of that with 

\mainline[level=1]{ 1. Bxf4 $1 exf4 2. e5 }
 * which won a piece. 




\clearpage 
\newpage 

\section{4.??.??}
\fenboard{2kr4/1pp4p/1p1r4/5Pp1/1P2q3/2P1R2P/P3KP2/1Q1R4 b - - 0 1}
\begin{center}
\showinverseboard 
\end{center}
\clearpage 
\newpage 
\mainline{1... Rd2+ $1 $19 { * The queen loses its defender. }}\fenboard{2kr4/1pp4p/1p1r4/5Pp1/1P2q3/2P1R2P/P3KP2/1Q1R4 b - - 0 1}
\begin{center}
\showinverseboard 
\end{center}
\fenboard{2kr4/1pp4p/1p1r4/5Pp1/1P2q3/2P1R2P/P3KP2/1Q1R4 b - - 0 1}
\mainline[level=1]{1... Rd2+ $1 $19 }
 * The queen loses its defender. 




\clearpage 
\newpage 

\section{5.??.??}
\fenboard{rn1qk2r/ppp2ppp/5n2/2b1p3/2B1P1b1/3P1N2/PPP3PP/RNBQK2R w KQkq - 0 1}
\begin{center}
\showboard 
\end{center}
\clearpage 
\newpage 
\mainline{1. Bxf7+ Kxf7 2. Nxe5+ $18 { * White has won two pawns after he next plays Nxg4. }}\fenboard{rn1qk2r/ppp2ppp/5n2/2b1p3/2B1P1b1/3P1N2/PPP3PP/RNBQK2R w KQkq - 0 1}
\begin{center}
\showboard 
\end{center}
\fenboard{rn1qk2r/ppp2ppp/5n2/2b1p3/2B1P1b1/3P1N2/PPP3PP/RNBQK2R w KQkq - 0 1}
\mainline[level=1]{1. Bxf7+ Kxf7 2. Nxe5+ $18 }
 * White has won two pawns after he next plays Nxg4. 




\clearpage 
\newpage 

\section{6.??.??}
\fenboard{r2k3r/pp1b3p/1qn1p1p1/1B1pPn2/Q7/P4N2/1P1BNPPP/2R3K1 w - - 0 1}
\begin{center}
\showboard 
\end{center}
\clearpage 
\newpage 
\mainline{1. Rxc6 $1 bxc6 2. Ba5 $18 { * White emerges with a queen and a knight for a rook and bishop. }}\fenboard{r2k3r/pp1b3p/1qn1p1p1/1B1pPn2/Q7/P4N2/1P1BNPPP/2R3K1 w - - 0 1}
\begin{center}
\showboard 
\end{center}
\fenboard{r2k3r/pp1b3p/1qn1p1p1/1B1pPn2/Q7/P4N2/1P1BNPPP/2R3K1 w - - 0 1}
\mainline[level=1]{1. Rxc6 $1 bxc6 2. Ba5 $18 }
 * White emerges with a queen and a knight for a rook and bishop. 




\clearpage 
\newpage 

\section{7.??.??}
\fenboard{2r1k2r/1b1p2q1/p4p2/4p3/PpB1Pp1p/7P/1PPRQPP1/4R1K1 b k - 0 1}
\begin{center}
\showinverseboard 
\end{center}
\clearpage 
\newpage 
\mainline{1... f3 $1 { The queen is overloaded and White loses the bishop on c4: } 2. Qxf3 Rxc4 $19 { * }}\fenboard{2r1k2r/1b1p2q1/p4p2/4p3/PpB1Pp1p/7P/1PPRQPP1/4R1K1 b k - 0 1}
\begin{center}
\showinverseboard 
\end{center}
\fenboard{2r1k2r/1b1p2q1/p4p2/4p3/PpB1Pp1p/7P/1PPRQPP1/4R1K1 b k - 0 1}
\mainline[level=1]{1... f3 $1 }
 The queen is overloaded and White loses the bishop on c4: 

\variation[level=2]{ 1... Rxc4 $6 2. Qxc4 f3 3. Qf1 $14 \xskakcomment{ is of course not the way to go. }} 

\mainline[level=1]{ 2. Qxf3 Rxc4 $19 }
 * 




\clearpage 
\newpage 

\section{8.??.??}
\fenboard{r3k2r/p1ppbppp/1pn1q3/4P3/2BP2n1/5NB1/1PP1Q1PP/R4K1R b kq - 0 1}
\begin{center}
\showinverseboard 
\end{center}
\clearpage 
\newpage 
\mainline{1... Qxc4 $1 { There is a fork on e3 coming up: } 2. Qxc4 Ne3+ 3. Ke2 Nxc4 $19 { * }}\fenboard{r3k2r/p1ppbppp/1pn1q3/4P3/2BP2n1/5NB1/1PP1Q1PP/R4K1R b kq - 0 1}
\begin{center}
\showinverseboard 
\end{center}
\fenboard{r3k2r/p1ppbppp/1pn1q3/4P3/2BP2n1/5NB1/1PP1Q1PP/R4K1R b kq - 0 1}
\mainline[level=1]{1... Qxc4 $1 }
 There is a fork on e3 coming up: 

\mainline[level=1]{ 2. Qxc4 Ne3+ 3. Ke2 Nxc4 $19 }
 * 




\clearpage 
\newpage 

\section{9.??.??}
\fenboard{1b6/3n1p2/r1k1p1pp/Pr2P3/1PK2P2/3R4/3B2PP/R7 w - - 0 1}
\begin{center}
\showboard 
\end{center}
\clearpage 
\newpage 
\mainline{1. Rxd7 $1 { White wins back the piece with a winning advantage. } 1... Kxd7 2. Kxb5 $18 { * }}\fenboard{1b6/3n1p2/r1k1p1pp/Pr2P3/1PK2P2/3R4/3B2PP/R7 w - - 0 1}
\begin{center}
\showboard 
\end{center}
\fenboard{1b6/3n1p2/r1k1p1pp/Pr2P3/1PK2P2/3R4/3B2PP/R7 w - - 0 1}
\mainline[level=1]{1. Rxd7 $1 }
 White wins back the piece with a winning advantage. 

\mainline[level=1]{ 1... Kxd7 }

\variation[level=2]{ 1... Rxb4+ \xskakcomment{ captures a pawn, but the a-pawn still decides after: }} \variation[level=2]{ 2. Bxb4 Kxd7 3. Kb5 $18 \xskakcomment{ * }} 

\mainline[level=1]{ 2. Kxb5 $18 }
 * 




\clearpage 
\newpage 

\section{10.??.??}
\fenboard{2b5/4Q1pp/pp3n1k/3p3q/P2P1P2/BP1B2P1/7P/6K1 w - - 0 1}
\begin{center}
\showboard 
\end{center}
\clearpage 
\newpage 
\mainline{1. Qxf6+ $1 { Black resigned due to: } 1... gxf6 2. Bf8# { mate * }}\fenboard{2b5/4Q1pp/pp3n1k/3p3q/P2P1P2/BP1B2P1/7P/6K1 w - - 0 1}
\begin{center}
\showboard 
\end{center}
\fenboard{2b5/4Q1pp/pp3n1k/3p3q/P2P1P2/BP1B2P1/7P/6K1 w - - 0 1}
\mainline[level=1]{1. Qxf6+ $1 }
 Black resigned due to: 

\mainline[level=1]{ 1... gxf6 2. Bf8# }
 mate * 




\clearpage 
\newpage 

\section{11.??.??}
\fenboard{4r2k/1b3Q1p/p1q3p1/1p4B1/2pb4/8/PPB3PP/5R1K w - - 0 1}
\begin{center}
\showboard 
\end{center}
\clearpage 
\newpage 
\mainline{1. Be4 $1 { The only drawing move, and easy to find, as Black's mate threat
means White has no other sensible try. } 1... Qxe4 2. Bf6+ Bxf6 3. Qxf6+ { * 1/2-1/2 White has a perpetual on f6 and f7. }}\fenboard{4r2k/1b3Q1p/p1q3p1/1p4B1/2pb4/8/PPB3PP/5R1K w - - 0 1}
\begin{center}
\showboard 
\end{center}
\fenboard{4r2k/1b3Q1p/p1q3p1/1p4B1/2pb4/8/PPB3PP/5R1K w - - 0 1}
\mainline[level=1]{1. Be4 $1 }
 The only drawing move, and easy to find, as Black's mate threat
means White has no other sensible try. 

\variation[level=2]{ 1. Rg1 $4 \xskakcomment{ loses to everything, but is mated most swiftly by }} \variation[level=2]{ 1... Qxg2+ $1 }

\mainline[level=1]{ 1... Qxe4 }

\variation[level=2]{ \xskakcomment{ Obviously not }} \variation[level=2]{ 1... Rxe4 $4 2. Qf8# \xskakcomment{ mate }} 

\mainline[level=1]{ 2. Bf6+ Bxf6 3. Qxf6+ }
 * \aDraw White has a perpetual on f6 and f7. 




\clearpage 
\newpage 

\section{12.??.??}
\fenboard{r1n5/pp2q1kp/2ppr1p1/4p1Q1/8/2N4R/PPP3PP/5RK1 w - - 0 1}
\begin{center}
\showboard 
\end{center}
\clearpage 
\newpage 
\mainline{1. Qh6+ $1 Kg8 2. Rf8+ $1 Qxf8 3. Qxh7# { mate * }}\fenboard{r1n5/pp2q1kp/2ppr1p1/4p1Q1/8/2N4R/PPP3PP/5RK1 w - - 0 1}
\begin{center}
\showboard 
\end{center}
\fenboard{r1n5/pp2q1kp/2ppr1p1/4p1Q1/8/2N4R/PPP3PP/5RK1 w - - 0 1}
\mainline[level=1]{1. Qh6+ $1 Kg8 2. Rf8+ $1 Qxf8 3. Qxh7# }
 mate * 




\clearpage 
\newpage 

\section{13.??.??}
\fenboard{4rrk1/ppp3pp/3p2n1/3Ppqb1/nPP5/6P1/P1NBQP1P/2R1NRK1 b - - 0 1}
\begin{center}
\showinverseboard 
\end{center}
\clearpage 
\newpage 
\mainline{1... Nc3 $1 { The queen can't move and keep the bishop on d2 defended, and } 2. Bxc3 Bxc1 $19 { * lost an exchange (0-1, 39 moves). }}\fenboard{4rrk1/ppp3pp/3p2n1/3Ppqb1/nPP5/6P1/P1NBQP1P/2R1NRK1 b - - 0 1}
\begin{center}
\showinverseboard 
\end{center}
\fenboard{4rrk1/ppp3pp/3p2n1/3Ppqb1/nPP5/6P1/P1NBQP1P/2R1NRK1 b - - 0 1}
\mainline[level=1]{1... Nc3 $1 }
 The queen can't move and keep the bishop on d2 defended, and 

\mainline[level=1]{ 2. Bxc3 Bxc1 $19 }
 * lost an exchange (\blackWins, 39 moves). 




\clearpage 
\newpage 

\section{14.??.??}
\fenboard{2kr3r/p4pp1/2p4p/4p3/2n4q/1NPPnP1P/PP2Q2P/R1K2B1R b - - 0 1}
\begin{center}
\showinverseboard 
\end{center}
\clearpage 
\newpage 
\mainline{1... Rxd3 $1 { Breaking open the king's position to close out the game. } 2. Bg2 Rhd8 { There is nothing
White can do against the threats to penetrate on d1 or d2. The game ended
after: } 3. a4 Rd1+ 4. Rxd1 Rxd1+ 5. Qxd1 Nxd1 { 0-1 }}\fenboard{2kr3r/p4pp1/2p4p/4p3/2n4q/1NPPnP1P/PP2Q2P/R1K2B1R b - - 0 1}
\begin{center}
\showinverseboard 
\end{center}
\fenboard{2kr3r/p4pp1/2p4p/4p3/2n4q/1NPPnP1P/PP2Q2P/R1K2B1R b - - 0 1}
\mainline[level=1]{1... Rxd3 $1 }
 Breaking open the king's position to close out the game. 

\mainline[level=1]{ 2. Bg2 }

\variation[level=2]{ \xskakcomment{ Or }} \variation[level=2]{ 2. Qxd3 Qe1+ \xskakcomment{ * with mate on the next move. }} 

\mainline[level=1]{ 2... Rhd8 }
 There is nothing
White can do against the threats to penetrate on d1 or d2. The game ended
after: 

\mainline[level=1]{ 3. a4 Rd1+ 4. Rxd1 Rxd1+ 5. Qxd1 Nxd1 }
 \blackWins 




\clearpage 
\newpage 

\section{15.??.??}
\fenboard{6k1/5pp1/p1n1r2p/2NQ4/1P1p4/P6P/1B1bqPP1/5RK1 b - - 0 1}
\begin{center}
\showinverseboard 
\end{center}
\clearpage 
\newpage 
\mainline{1... Qxf1+ $1 { * White's previous move, Qc4-d5, was a grave blunder. } 2. Kxf1 Re1# { mate }}\fenboard{6k1/5pp1/p1n1r2p/2NQ4/1P1p4/P6P/1B1bqPP1/5RK1 b - - 0 1}
\begin{center}
\showinverseboard 
\end{center}
\fenboard{6k1/5pp1/p1n1r2p/2NQ4/1P1p4/P6P/1B1bqPP1/5RK1 b - - 0 1}
\mainline[level=1]{1... Qxf1+ $1 }
 * White's previous move, Qc4-d5, was a grave blunder. 

\mainline[level=1]{ 2. Kxf1 Re1# }
 mate 




\clearpage 
\newpage 

\section{16.??.??}
\fenboard{r1bqk1nr/pppp3p/2n2p2/b5p1/2BPPp1P/2P2N2/P5P1/RNBQK2R w KQkq - 0 1}
\begin{center}
\showboard 
\end{center}
\clearpage 
\newpage 
\mainline{1. Nxg5 $1 { Not recapturing would be equivalent to resignation, but Black is
mated if he takes the knight: } 1... fxg5 2. Qh5+ { * } 2... Ke7 { There are several ways to mate or pick up material. The quickest mate is: } 3. Qf7+ Kd6 4. e5+ Nxe5 5. dxe5+ Kxe5 6. Qd5+ Kf6 7. Qxg5# { mate }}\fenboard{r1bqk1nr/pppp3p/2n2p2/b5p1/2BPPp1P/2P2N2/P5P1/RNBQK2R w KQkq - 0 1}
\begin{center}
\showboard 
\end{center}
\fenboard{r1bqk1nr/pppp3p/2n2p2/b5p1/2BPPp1P/2P2N2/P5P1/RNBQK2R w KQkq - 0 1}
\mainline[level=1]{1. Nxg5 $1 }
 Not recapturing would be equivalent to resignation, but Black is
mated if he takes the knight: 

\mainline[level=1]{ 1... fxg5 }

\variation[level=2]{ 1... Qe7 $18 \xskakcomment{ and White won after 26 moves. }} 

\mainline[level=1]{ 2. Qh5+ }
 * 

\mainline[level=1]{ 2... Ke7 }
 There are several ways to mate or pick up material. The quickest mate is: 

\mainline[level=1]{ 3. Qf7+ Kd6 4. e5+ Nxe5 5. dxe5+ Kxe5 6. Qd5+ Kf6 7. Qxg5# }
 mate 




\clearpage 
\newpage 

\section{17.??.??}
\fenboard{2kr1bnr/p1ppqp1p/bpn5/1N4p1/P2PPp2/5N2/1PP2KPP/R1BQ1B1R w - - 0 1}
\begin{center}
\showboard 
\end{center}
\clearpage 
\newpage 
\mainline{1. Nxa7+ $1 { White wins an important pawn after: } 1... Nxa7 2. Bxa6+ { * (1-0, 25 moves) }}\fenboard{2kr1bnr/p1ppqp1p/bpn5/1N4p1/P2PPp2/5N2/1PP2KPP/R1BQ1B1R w - - 0 1}
\begin{center}
\showboard 
\end{center}
\fenboard{2kr1bnr/p1ppqp1p/bpn5/1N4p1/P2PPp2/5N2/1PP2KPP/R1BQ1B1R w - - 0 1}
\mainline[level=1]{1. Nxa7+ $1 }
 White wins an important pawn after: 

\mainline[level=1]{ 1... Nxa7 2. Bxa6+ }
 * (\whiteWins, 25 moves) 




\clearpage 
\newpage 

\section{18.??.??}
\fenboard{rn1qk1nr/ppp2ppp/8/2b1p3/2B1P1b1/5N2/PPPP2PP/RNBQK2R w KQkq - 0 1}
\begin{center}
\showboard 
\end{center}
\clearpage 
\newpage 
\mainline{1. Bxf7+ $1 { Not the only time Steinitz executed this combination. White wins
two pawns after: } 1... Kxf7 2. Nxe5+ $18 { * }}\fenboard{rn1qk1nr/ppp2ppp/8/2b1p3/2B1P1b1/5N2/PPPP2PP/RNBQK2R w KQkq - 0 1}
\begin{center}
\showboard 
\end{center}
\fenboard{rn1qk1nr/ppp2ppp/8/2b1p3/2B1P1b1/5N2/PPPP2PP/RNBQK2R w KQkq - 0 1}
\mainline[level=1]{1. Bxf7+ $1 }
 Not the only time Steinitz executed this combination. White wins
two pawns after: 

\mainline[level=1]{ 1... Kxf7 2. Nxe5+ $18 }
 * 




\clearpage 
\newpage 

\section{19.??.??}
\fenboard{1k2r3/2p3p1/p4p2/1p3q1p/1n6/PQ2P3/1P2B2P/2KR4 b - - 0 1}
\begin{center}
\showinverseboard 
\end{center}
\clearpage 
\newpage 
\mainline{1... Rxe3 $1 2. Qxb4 Rxe2 $19 { * White resigned five moves later. }}\fenboard{1k2r3/2p3p1/p4p2/1p3q1p/1n6/PQ2P3/1P2B2P/2KR4 b - - 0 1}
\begin{center}
\showinverseboard 
\end{center}
\fenboard{1k2r3/2p3p1/p4p2/1p3q1p/1n6/PQ2P3/1P2B2P/2KR4 b - - 0 1}

 Black is two pawns up, but that doesn't stop him from being precise: 

\mainline[level=1]{ 1... Rxe3 $1 2. Qxb4 }

\variation[level=2]{ 2. Qxe3 Qc2# \xskakcomment{ mate * }} 

\mainline[level=1]{ 2... Rxe2 $19 }
 * White resigned five moves later. 




\clearpage 
\newpage 

\section{20.??.??}
\fenboard{r1bqkbnr/pppp3p/2n2p2/6p1/2BPPp2/5N2/PPP3PP/RNBQK2R w KQkq - 0 1}
\begin{center}
\showboard 
\end{center}
\clearpage 
\newpage 
\mainline{1. Nxg5 $1 { Black cannot take back: } 1... fxg5 2. Qh5+ { * } 2... Ke7 3. Qf7+ Kd6 { And for instance: } 4. e5+ Nxe5 5. Qd5+ Ke7 6. Qxe5# { mate }}\fenboard{r1bqkbnr/pppp3p/2n2p2/6p1/2BPPp2/5N2/PPP3PP/RNBQK2R w KQkq - 0 1}
\begin{center}
\showboard 
\end{center}
\fenboard{r1bqkbnr/pppp3p/2n2p2/6p1/2BPPp2/5N2/PPP3PP/RNBQK2R w KQkq - 0 1}
\mainline[level=1]{1. Nxg5 $1 }
 Black cannot take back: 

\mainline[level=1]{ 1... fxg5 }

\variation[level=2]{ \xskakcomment{ Black instead allowed a forced mate after }} \variation[level=2]{ 1... h6 }

\mainline[level=1]{ 2. Qh5+ }
 * 

\mainline[level=1]{ 2... Ke7 3. Qf7+ Kd6 }
 And for instance: 

\mainline[level=1]{ 4. e5+ Nxe5 5. Qd5+ Ke7 6. Qxe5# }
 mate 




\clearpage 
\newpage 

\section{21.??.??}
\fenboard{rnbqkbnr/pppp3p/5p2/6p1/4Pp1P/5N2/PPPP2P1/RNBQKB1R w KQkq - 0 1}
\begin{center}
\showboard 
\end{center}
\clearpage 
\newpage 
\mainline{1. Nxg5 $1 fxg5 2. Qh5+ Ke7 3. Qxg5+ { * } 3... Ke8 4. Qh5+ $1 Ke7 5. Qe5+ $18 { White picks up the rook on h8. }}\fenboard{rnbqkbnr/pppp3p/5p2/6p1/4Pp1P/5N2/PPPP2P1/RNBQKB1R w KQkq - 0 1}
\begin{center}
\showboard 
\end{center}
\fenboard{rnbqkbnr/pppp3p/5p2/6p1/4Pp1P/5N2/PPPP2P1/RNBQKB1R w KQkq - 0 1}
\mainline[level=1]{1. Nxg5 $1 fxg5 2. Qh5+ Ke7 3. Qxg5+ }
 * 

\mainline[level=1]{ 3... Ke8 4. Qh5+ $1 }

\variation[level=2]{ \xskakcomment{ Imprecise is }} \variation[level=2]{ 4. Qe5+ $6 Qe7 5. Qxh8 Qxe4+ $18 \xskakcomment{ with some slight counterplay. }} 

\mainline[level=1]{ 4... Ke7 5. Qe5+ $18 }
 White picks up the rook on h8. 




\clearpage 
\newpage 

\section{22.??.??}
\fenboard{rnbqkb1r/pp1p2pp/2p2p2/4p3/2B5/2P2N2/PPP2PPP/R1BQ1RK1 w kq - 0 1}
\begin{center}
\showboard 
\end{center}
\clearpage 
\newpage 
\mainline{1. Nxe5 $1 { A surprisingly common theme in Lasker's games. } 1... d5 2. Qh5+ g6 3. Nxg6 hxg6 4. Qxh8 { * } 4... dxc4 { There are many ways to win and you don't have to decide in advance. Easiest is: } 5. Re1+ Kf7 6. Bh6 $18}\fenboard{rnbqkb1r/pp1p2pp/2p2p2/4p3/2B5/2P2N2/PPP2PPP/R1BQ1RK1 w kq - 0 1}
\begin{center}
\showboard 
\end{center}
\fenboard{rnbqkb1r/pp1p2pp/2p2p2/4p3/2B5/2P2N2/PPP2PPP/R1BQ1RK1 w kq - 0 1}
\mainline[level=1]{1. Nxe5 $1 }
 A surprisingly common theme in Lasker's games. 

\mainline[level=1]{ 1... d5 }

\variation[level=2]{ 1... fxe5 2. Qh5+ g6 } (\variation[level=3]{ 2... Ke7 3. Qxe5# \xskakcomment{ mate }})
\variation[level=2]{ 3. Qxe5+ Qe7 4. Qxh8 $18 \xskakcomment{ * }} 

\mainline[level=1]{ 2. Qh5+ }

\variation[level=2]{ \xskakcomment{ Or just as good is: }} \variation[level=2]{ 2. Re1 fxe5 3. Rxe5+ Kd7 4. Bg5 $1 $18 }

\mainline[level=1]{ 2... g6 }

\variation[level=2]{ \xskakcomment{ One source gives this game as played in New York 1911, with }} \variation[level=2]{ 2... Ke7 3. Nf7 $2 Qe8 $2 \xskakcomment{ \blackWins (??) as the final moves, none of which makes any sense. }} 

\mainline[level=1]{ 3. Nxg6 hxg6 4. Qxh8 }
 * 

\variation[level=2]{ \xskakcomment{ Or }} \variation[level=2]{ 4. Re1+ \xskakcomment{ first. }} 

\mainline[level=1]{ 4... dxc4 }
 There are many ways to win and you don't have to decide in advance. Easiest is: 




\clearpage 
\newpage 

\section{23.??.??}
\fenboard{2r3k1/p3qppp/2pr4/Q2b4/1P2p3/4P3/P3BPPP/2RR2K1 w - - 0 1}
\begin{center}
\showboard 
\end{center}
\clearpage 
\newpage 
\mainline{1. Rxd5 $1 Rxd5 2. Qxd5 $18 { * White has won a piece, since Black cannot recapture. }}\fenboard{2r3k1/p3qppp/2pr4/Q2b4/1P2p3/4P3/P3BPPP/2RR2K1 w - - 0 1}
\begin{center}
\showboard 
\end{center}
\fenboard{2r3k1/p3qppp/2pr4/Q2b4/1P2p3/4P3/P3BPPP/2RR2K1 w - - 0 1}
\mainline[level=1]{1. Rxd5 $1 Rxd5 2. Qxd5 $18 }
 * White has won a piece, since Black cannot recapture. 




\clearpage 
\newpage 

\section{24.??.??}
\fenboard{6k1/2p3pp/q3pn2/1pp1p3/4P3/1P1P1P2/rNP2P1P/1Q3RK1 w - - 0 1}
\begin{center}
\showboard 
\end{center}
\clearpage 
\newpage 
\mainline{1. Na4 $1 { * The rook is trapped and the c5-pawn is threatened. } 1... Ra3 2. Qb2 b4 3. Qxe5 { Black loses a second pawn. }}\fenboard{6k1/2p3pp/q3pn2/1pp1p3/4P3/1P1P1P2/rNP2P1P/1Q3RK1 w - - 0 1}
\begin{center}
\showboard 
\end{center}
\fenboard{6k1/2p3pp/q3pn2/1pp1p3/4P3/1P1P1P2/rNP2P1P/1Q3RK1 w - - 0 1}
\mainline[level=1]{1. Na4 $1 }
 * The rook is trapped and the c5-pawn is threatened. 

\mainline[level=1]{ 1... Ra3 2. Qb2 }

\variation[level=2]{ \xskakcomment{ Lasker played }} \variation[level=2]{ 2. Nxc5 \xskakcomment{ and won after 22 moves, but the text is better. }} 

\mainline[level=1]{ 2... b4 3. Qxe5 }
 Black loses a second pawn. 




\clearpage 
\newpage 

\section{25.??.??}
\fenboard{8/1p3q1k/2p3pp/4P1r1/8/4Q3/PP6/3R3K b - - 0 1}
\begin{center}
\showinverseboard 
\end{center}
\clearpage 
\newpage 
\mainline{1... Rxe5 $1 2. Qxe5 Qf3+ 3. Kg1 Qxd1+ { * } 4. Kf2 Qd7 $17 { It is probably a
theoretical draw, but that does not change the verdict during a game between
humans (0-1, 55 moves). }}\fenboard{8/1p3q1k/2p3pp/4P1r1/8/4Q3/PP6/3R3K b - - 0 1}
\begin{center}
\showinverseboard 
\end{center}
\fenboard{8/1p3q1k/2p3pp/4P1r1/8/4Q3/PP6/3R3K b - - 0 1}
\mainline[level=1]{1... Rxe5 $1 2. Qxe5 Qf3+ 3. Kg1 Qxd1+ }
 * 

\mainline[level=1]{ 4. Kf2 Qd7 $17 }
 It is probably a
theoretical draw, but that does not change the verdict during a game between
humans (\blackWins, 55 moves). 




\clearpage 
\newpage 

\section{26.??.??}
\fenboard{r5k1/1b1n2r1/p3n2q/1p1pPRN1/2pP3P/2P3P1/PPBQ4/5R1K w - - 0 1}
\begin{center}
\showboard 
\end{center}
\clearpage 
\newpage 
\mainline{1. Rf6 $1 Nxf6 2. Rxf6 { * } 2... Qh5 3. Bd1 { Not necessary, but a luxury White can afford. } 3... Qe8 4. Rxe6 $18 { Black is an exchange up, but since
he has no chance against all the pawns and an invasion on the kingside, he
resigned now. }}\fenboard{r5k1/1b1n2r1/p3n2q/1p1pPRN1/2pP3P/2P3P1/PPBQ4/5R1K w - - 0 1}
\begin{center}
\showboard 
\end{center}
\fenboard{r5k1/1b1n2r1/p3n2q/1p1pPRN1/2pP3P/2P3P1/PPBQ4/5R1K w - - 0 1}

 White has a minor piece less, but can more than make up for it with the
following double threat: 

\mainline[level=1]{ 1. Rf6 $1 Nxf6 2. Rxf6 }
 * 

\mainline[level=1]{ 2... Qh5 3. Bd1 }
 Not necessary, but a luxury White can afford. 

\variation[level=2]{\xskakcomment{\noindent\textbf{a)} } 3. Nxe6 }

\variation[level=2]{\xskakcomment{\noindent\textbf{b)} } \xskakcomment{ and }} \variation[level=2]{ 3. Rxe6 \xskakcomment{ are also winning. }} 

\mainline[level=1]{ 3... Qe8 4. Rxe6 $18 }
 Black is an exchange up, but since
he has no chance against all the pawns and an invasion on the kingside, he
resigned now. 




\clearpage 
\newpage 

\section{27.??.??}
\fenboard{r1b2rk1/p2p1p2/2p5/1p2PPqn/1b1p2N1/1B1P3Q/PPP3PP/R4RK1 w - - 0 1}
\begin{center}
\showboard 
\end{center}
\clearpage 
\newpage 
\mainline{1. Qxh5 $1 { The fork on f6 gains a piece. } 1... Qxh5 2. Nf6+ Kh8 3. Nxh5 $18 { * }}\fenboard{r1b2rk1/p2p1p2/2p5/1p2PPqn/1b1p2N1/1B1P3Q/PPP3PP/R4RK1 w - - 0 1}
\begin{center}
\showboard 
\end{center}
\fenboard{r1b2rk1/p2p1p2/2p5/1p2PPqn/1b1p2N1/1B1P3Q/PPP3PP/R4RK1 w - - 0 1}
\mainline[level=1]{1. Qxh5 $1 }
 The fork on f6 gains a piece. 

\mainline[level=1]{ 1... Qxh5 2. Nf6+ Kh8 3. Nxh5 $18 }
 * 




\clearpage 
\newpage 

\section{28.??.??}
\fenboard{r5r1/p1p1k3/3q3B/5p2/4p3/1P6/P1P1QPP1/R4RK1 b - - 0 1}
\begin{center}
\showinverseboard 
\end{center}
\clearpage 
\newpage 
\mainline{1... Rxg2+ $1 { * White resigned, since he is mated after: } 2. Kxg2 Rg8+ 3. Kh1 Qxh6+ 4. Qh5 Qxh5# { mate }}\fenboard{r5r1/p1p1k3/3q3B/5p2/4p3/1P6/P1P1QPP1/R4RK1 b - - 0 1}
\begin{center}
\showinverseboard 
\end{center}
\fenboard{r5r1/p1p1k3/3q3B/5p2/4p3/1P6/P1P1QPP1/R4RK1 b - - 0 1}
\mainline[level=1]{1... Rxg2+ $1 }
 * White resigned, since he is mated after: 

\mainline[level=1]{ 2. Kxg2 Rg8+ }

\variation[level=2]{ \xskakcomment{ Or }} \variation[level=2]{ 2... Qg6+ }

\mainline[level=1]{ 3. Kh1 Qxh6+ 4. Qh5 Qxh5# }
 mate 




\clearpage 
\newpage 

\section{29.??.??}
\fenboard{6rk/p1q2p2/2p1rb1P/1p2pN2/4P1Q1/2PP4/PPB5/2K4R w - - 0 1}
\begin{center}
\showboard 
\end{center}
\clearpage 
\newpage 
\mainline{1. Qg7+ $1 Rxg7 2. hxg7+ Kg8 3. Rh8# { mate * }}\fenboard{6rk/p1q2p2/2p1rb1P/1p2pN2/4P1Q1/2PP4/PPB5/2K4R w - - 0 1}
\begin{center}
\showboard 
\end{center}
\fenboard{6rk/p1q2p2/2p1rb1P/1p2pN2/4P1Q1/2PP4/PPB5/2K4R w - - 0 1}
\mainline[level=1]{1. Qg7+ $1 Rxg7 2. hxg7+ Kg8 3. Rh8# }
 mate * 




\clearpage 
\newpage 

\section{30.??.??}
\fenboard{4q1k1/2r3pp/1p6/8/1b2N3/4R1P1/PP3P1P/R5K1 w - - 0 1}
\begin{center}
\showboard 
\end{center}
\clearpage 
\newpage 
\mainline{1. Nf6+ $1 { Black could have resigned here, but continued another 15 moves. } 1... gxf6 2. Rxe8+ $18 { * }}\fenboard{4q1k1/2r3pp/1p6/8/1b2N3/4R1P1/PP3P1P/R5K1 w - - 0 1}
\begin{center}
\showboard 
\end{center}
\fenboard{4q1k1/2r3pp/1p6/8/1b2N3/4R1P1/PP3P1P/R5K1 w - - 0 1}
\mainline[level=1]{1. Nf6+ $1 }
 Black could have resigned here, but continued another 15 moves. 

\mainline[level=1]{ 1... gxf6 2. Rxe8+ $18 }
 * 




\clearpage 
\newpage 

\section{31.??.??}
\fenboard{4k3/1bp4r/p7/1p1P4/2P3pN/1P2r1P1/1BP2RPK/8 b - - 0 1}
\begin{center}
\showinverseboard 
\end{center}
\clearpage 
\newpage 
\mainline{1... Rxh4+ $1 { Black can also start by exchanging on c4. } 2. gxh4 g3+ 3. Kg1 gxf2+ { * } 4. Kxf2 $19 { Instead of being an exchange up, Black is a rook up. }}\fenboard{4k3/1bp4r/p7/1p1P4/2P3pN/1P2r1P1/1BP2RPK/8 b - - 0 1}
\begin{center}
\showinverseboard 
\end{center}
\fenboard{4k3/1bp4r/p7/1p1P4/2P3pN/1P2r1P1/1BP2RPK/8 b - - 0 1}
\mainline[level=1]{1... Rxh4+ $1 }
 Black can also start by exchanging on c4. 

\mainline[level=1]{ 2. gxh4 g3+ 3. Kg1 gxf2+ }
 * 

\mainline[level=1]{ 4. Kxf2 $19 }
 Instead of being an exchange up, Black is a rook up. 




\clearpage 
\newpage 

\section{32.??.??}
\fenboard{2q4k/5Qp1/4B2p/p1p5/1P6/6PK/r4P1P/8 b - - 0 1}
\begin{center}
\showinverseboard 
\end{center}
\clearpage 
\newpage 
\mainline{1... Rxf2 $1 { Defending against the double threat and getting a queen ending
with two healthy pawns and a safe king. } 2. Qxf2 Qxe6+ $18 { * }}\fenboard{2q4k/5Qp1/4B2p/p1p5/1P6/6PK/r4P1P/8 b - - 0 1}
\begin{center}
\showinverseboard 
\end{center}
\fenboard{2q4k/5Qp1/4B2p/p1p5/1P6/6PK/r4P1P/8 b - - 0 1}
\mainline[level=1]{1... Rxf2 $1 }
 Defending against the double threat and getting a queen ending
with two healthy pawns and a safe king. 

\mainline[level=1]{ 2. Qxf2 Qxe6+ $18 }
 * 




\clearpage 
\newpage 

\section{33.??.??}
\fenboard{5Rnk/pp1q4/7p/3p2rN/3Pp1Q1/2P5/PP5P/6K1 w - - 0 1}
\begin{center}
\showboard 
\end{center}
\clearpage 
\newpage 
\mainline{1. Qxg5 $1 hxg5 2. Rxg8+ { Black resigned since he will be a piece down: } 2... Kxg8 3. Nf6+ Kf7 4. Nxd7 $18 { * }}\fenboard{5Rnk/pp1q4/7p/3p2rN/3Pp1Q1/2P5/PP5P/6K1 w - - 0 1}
\begin{center}
\showboard 
\end{center}
\fenboard{5Rnk/pp1q4/7p/3p2rN/3Pp1Q1/2P5/PP5P/6K1 w - - 0 1}
\mainline[level=1]{1. Qxg5 $1 }

\variation[level=2]{ 1. Rxg8+ Rxg8 2. Qxg8+ \xskakcomment{ is another way to do the same thing. }} 

\mainline[level=1]{ 1... hxg5 2. Rxg8+ }
 Black resigned since he will be a piece down: 

\mainline[level=1]{ 2... Kxg8 3. Nf6+ Kf7 4. Nxd7 $18 }
 * 




\clearpage 
\newpage 

\section{34.??.??}
\fenboard{r1bq2k1/pp3rpp/2n2b2/3p1p2/3P4/BQPB1N2/P4PPP/R3R1K1 w - - 0 1}
\begin{center}
\showboard 
\end{center}
\clearpage 
\newpage 
\mainline{1. Qxd5 $1 $18 { Black resigned, as he is mated after: } 1... Qxd5 2. Re8+ Rf8 3. Rxf8# { mate * }}\fenboard{r1bq2k1/pp3rpp/2n2b2/3p1p2/3P4/BQPB1N2/P4PPP/R3R1K1 w - - 0 1}
\begin{center}
\showboard 
\end{center}
\fenboard{r1bq2k1/pp3rpp/2n2b2/3p1p2/3P4/BQPB1N2/P4PPP/R3R1K1 w - - 0 1}
\mainline[level=1]{1. Qxd5 $1 $18 }
 Black resigned, as he is mated after: 

\mainline[level=1]{ 1... Qxd5 2. Re8+ Rf8 3. Rxf8# }
 mate * 




\clearpage 
\newpage 

\section{35.??.??}
\fenboard{2r3k1/pb2bp1p/1p2p1p1/8/q1NPP3/3B4/P3QPPP/3R2K1 b - - 0 1}
\begin{center}
\showinverseboard 
\end{center}
\clearpage 
\newpage 
\mainline{1... Bxe4 $1 { White's pieces are overloaded and Black won a pawn after: } 2. Bxe4 Rxc4 $19 { * }}\fenboard{2r3k1/pb2bp1p/1p2p1p1/8/q1NPP3/3B4/P3QPPP/3R2K1 b - - 0 1}
\begin{center}
\showinverseboard 
\end{center}
\fenboard{2r3k1/pb2bp1p/1p2p1p1/8/q1NPP3/3B4/P3QPPP/3R2K1 b - - 0 1}
\mainline[level=1]{1... Bxe4 $1 }
 White's pieces are overloaded and Black won a pawn after: 

\mainline[level=1]{ 2. Bxe4 Rxc4 $19 }
 * 




\clearpage 
\newpage 

\section{36.??.??}
\fenboard{6r1/2r1k3/R3p3/p4pPp/1pPK1P2/1P3B1P/P7/8 w - - 0 1}
\begin{center}
\showboard 
\end{center}
\clearpage 
\newpage 
\mainline{1. Rxe6+ $1 Kxe6 2. Bd5+ Kd6 3. Bxg8 $18 { * Black's rook is unable to fight
against the two passed pawns. The final moves were: } 3... Re7 4. c5+ Kc6 5. Bd5+ Kb5 6. g6}\fenboard{6r1/2r1k3/R3p3/p4pPp/1pPK1P2/1P3B1P/P7/8 w - - 0 1}
\begin{center}
\showboard 
\end{center}
\fenboard{6r1/2r1k3/R3p3/p4pPp/1pPK1P2/1P3B1P/P7/8 w - - 0 1}
\mainline[level=1]{1. Rxe6+ $1 Kxe6 2. Bd5+ Kd6 3. Bxg8 $18 }
 * Black's rook is unable to fight
against the two passed pawns. The final moves were: 




\clearpage 
\newpage 

\section{37.??.??}
\fenboard{r4r1k/pppqNppp/3p1B2/4p3/3nP3/3P1b2/PPPQ1PPP/R4RK1 w - - 0 1}
\begin{center}
\showboard 
\end{center}
\clearpage 
\newpage 
\mainline{1. Bxg7+ $1 Kxg7 2. Qg5+ Kh8 3. Qf6# { mate * }}\fenboard{r4r1k/pppqNppp/3p1B2/4p3/3nP3/3P1b2/PPPQ1PPP/R4RK1 w - - 0 1}
\begin{center}
\showboard 
\end{center}
\fenboard{r4r1k/pppqNppp/3p1B2/4p3/3nP3/3P1b2/PPPQ1PPP/R4RK1 w - - 0 1}
\mainline[level=1]{1. Bxg7+ $1 Kxg7 2. Qg5+ Kh8 3. Qf6# }
 mate * 




\clearpage 
\newpage 

\section{38.??.??}
\fenboard{3r2k1/2p2pp1/p1Q2n1p/7q/8/1P1N2P1/P1P2P2/R3K3 b Q - 0 1}
\begin{center}
\showinverseboard 
\end{center}
\clearpage 
\newpage 
\mainline{1... Rxd3 $1 { The game move wins a piece after: } 2. cxd3 Qe5+ 3. Kd2 Qxa1 $19 { * }}\fenboard{3r2k1/2p2pp1/p1Q2n1p/7q/8/1P1N2P1/P1P2P2/R3K3 b Q - 0 1}
\begin{center}
\showinverseboard 
\end{center}
\fenboard{3r2k1/2p2pp1/p1Q2n1p/7q/8/1P1N2P1/P1P2P2/R3K3 b Q - 0 1}
\mainline[level=1]{1... Rxd3 $1 }
 The game move wins a piece after: 

\variation[level=2]{ 1... Qh1+ $4 2. Qxh1 \xskakcomment{ would be a terrible blunder. }} 

\mainline[level=1]{ 2. cxd3 Qe5+ 3. Kd2 Qxa1 $19 }
 * 




\clearpage 
\newpage 

\section{39.??.??}
\fenboard{r1b2rk1/pp3qpp/2p1p3/2Ppb1PP/5B2/3BP3/PP3Q2/2R1K2R w K - 0 1}
\begin{center}
\showboard 
\end{center}
\clearpage 
\newpage 
\mainline{1. Bxh7+ $1 Kh8 2. Bg6 { And White won. }}\fenboard{r1b2rk1/pp3qpp/2p1p3/2Ppb1PP/5B2/3BP3/PP3Q2/2R1K2R w K - 0 1}
\begin{center}
\showboard 
\end{center}
\fenboard{r1b2rk1/pp3qpp/2p1p3/2Ppb1PP/5B2/3BP3/PP3Q2/2R1K2R w K - 0 1}
\mainline[level=1]{1. Bxh7+ $1 Kh8 }

\variation[level=2]{ \xskakcomment{ Or }} \variation[level=2]{ 1... Kxh7 2. g6+ $18 \xskakcomment{ * with a fork. }} 

\mainline[level=1]{ 2. Bg6 }
 And White won. 




\clearpage 
\newpage 

\section{40.??.??}
\fenboard{2r2rk1/pQ1n1pp1/1p2p2p/3p4/P2P4/4P2P/1qB2PP1/2R2RK1 w - - 0 1}
\begin{center}
\showboard 
\end{center}
\clearpage 
\newpage 
\mainline{1. Bh7+ $1 Kxh7 2. Rxc8 Rxc8 3. Qxc8 $18 { * White is an exchange up. }}\fenboard{2r2rk1/pQ1n1pp1/1p2p2p/3p4/P2P4/4P2P/1qB2PP1/2R2RK1 w - - 0 1}
\begin{center}
\showboard 
\end{center}
\fenboard{2r2rk1/pQ1n1pp1/1p2p2p/3p4/P2P4/4P2P/1qB2PP1/2R2RK1 w - - 0 1}
\mainline[level=1]{1. Bh7+ $1 }

\variation[level=2]{ 1. Qxd7 Rxc2 \xskakcomment{ is equal. }} 

\mainline[level=1]{ 1... Kxh7 2. Rxc8 Rxc8 3. Qxc8 $18 }
 * White is an exchange up. 




\clearpage 
\newpage 

\section{41.??.??}
\fenboard{3k4/p1p2prr/1p5N/3PRPP1/b1P5/4B3/P4K2/8 w - - 0 1}
\begin{center}
\showboard 
\end{center}
\clearpage 
\newpage 
\mainline{1. f6 Rg6 2. Bf4 $1 $18 { when Black's rooks are trapped and he can't
sacrifice the exchange in a good way. }}\fenboard{3k4/p1p2prr/1p5N/3PRPP1/b1P5/4B3/P4K2/8 w - - 0 1}
\begin{center}
\showboard 
\end{center}
\fenboard{3k4/p1p2prr/1p5N/3PRPP1/b1P5/4B3/P4K2/8 w - - 0 1}

 A good choice is 

\mainline[level=1]{ 1. f6 }

\variation[level=2]{ 1. Nxf7+ $6 \xskakcomment{ looks like a combination and was played in the game, but Black had }} \variation[level=2]{ 1... Rxf7 2. g6 Rxf5+ $1 3. Rxf5 Rg7 \xskakcomment{ when he collects the g-pawn with a draw. Instead normal moves give two points. }} 

\mainline[level=1]{ 1... Rg6 2. Bf4 $1 $18 }
 when Black's rooks are trapped and he can't
sacrifice the exchange in a good way. 




\clearpage 
\newpage 

\section{42.??.??}
\fenboard{r3k2r/pbp2qb1/1pn1p2p/3nP1pQ/3PNp2/2PB4/PP1N1BPP/R4RK1 w kq - 0 1}
\begin{center}
\showboard 
\end{center}
\clearpage 
\newpage 
\mainline{1. Nd6+ $1 { Black resigned due to: } 1... cxd6 2. Bg6 $18 { * }}\fenboard{r3k2r/pbp2qb1/1pn1p2p/3nP1pQ/3PNp2/2PB4/PP1N1BPP/R4RK1 w kq - 0 1}
\begin{center}
\showboard 
\end{center}
\fenboard{r3k2r/pbp2qb1/1pn1p2p/3nP1pQ/3PNp2/2PB4/PP1N1BPP/R4RK1 w kq - 0 1}
\mainline[level=1]{1. Nd6+ $1 }
 Black resigned due to: 

\mainline[level=1]{ 1... cxd6 2. Bg6 $18 }
 * 




\clearpage 
\newpage 

\section{43.??.??}
\fenboard{4k3/1r2r1pp/1nR2p2/pp1p4/1N1P2P1/1R2PP2/PP3K1P/8 w - - 0 1}
\begin{center}
\showboard 
\end{center}
\clearpage 
\newpage 
\mainline{1. Rxb6 $1 { Winning a second pawn. } 1... axb4 2. Rxb7 Rxb7 3. Rxb4 { 1-0 The endgame is hopeless for Black. }}\fenboard{4k3/1r2r1pp/1nR2p2/pp1p4/1N1P2P1/1R2PP2/PP3K1P/8 w - - 0 1}
\begin{center}
\showboard 
\end{center}
\fenboard{4k3/1r2r1pp/1nR2p2/pp1p4/1N1P2P1/1R2PP2/PP3K1P/8 w - - 0 1}
\mainline[level=1]{1. Rxb6 $1 }
 Winning a second pawn. 

\mainline[level=1]{ 1... axb4 }

\variation[level=2]{ \xskakcomment{ Or }} \variation[level=2]{ 1... Rxb6 2. Nxd5 $18 \xskakcomment{ * with a fork. }} 

\mainline[level=1]{ 2. Rxb7 Rxb7 3. Rxb4 }
 \whiteWins The endgame is hopeless for Black. 




\clearpage 
\newpage 

\section{44.??.??}
\fenboard{7r/1p3pp1/pn1kb2p/3p4/3N1P1P/PP1BP3/3K2P1/2R5 w - - 0 1}
\begin{center}
\showboard 
\end{center}
\clearpage 
\newpage 
\mainline{1. Bxa6 $1 { White wins a pawn due to: } 1... bxa6 2. Rc6+ Ke7 3. Rxb6 $18 { * }}\fenboard{7r/1p3pp1/pn1kb2p/3p4/3N1P1P/PP1BP3/3K2P1/2R5 w - - 0 1}
\begin{center}
\showboard 
\end{center}
\fenboard{7r/1p3pp1/pn1kb2p/3p4/3N1P1P/PP1BP3/3K2P1/2R5 w - - 0 1}
\mainline[level=1]{1. Bxa6 $1 }
 White wins a pawn due to: 

\mainline[level=1]{ 1... bxa6 2. Rc6+ Ke7 3. Rxb6 $18 }
 * 




\clearpage 
\newpage 

\section{45.??.??}
\fenboard{r3nrk1/1bp2ppp/pp2p3/3q2N1/1b1PNP2/3B2P1/PP2QP1P/2RR2K1 w - - 0 1}
\begin{center}
\showboard 
\end{center}
\clearpage 
\newpage 
\mainline{1. Nxh7 $1 f5 2. Nhg5 { In the game, Black resigned due to: } 2... fxe4 3. Bxe4 { The bishop on b7 is doomed. (White could also go for the
king with 19.Bc4 $18, forcing Black to give up his queen.) }}\fenboard{r3nrk1/1bp2ppp/pp2p3/3q2N1/1b1PNP2/3B2P1/PP2QP1P/2RR2K1 w - - 0 1}
\begin{center}
\showboard 
\end{center}
\fenboard{r3nrk1/1bp2ppp/pp2p3/3q2N1/1b1PNP2/3B2P1/PP2QP1P/2RR2K1 w - - 0 1}
\mainline[level=1]{1. Nxh7 $1 f5 }

\variation[level=2]{ \xskakcomment{ Instead }} \variation[level=2]{ 1... Kxh7 \xskakcomment{ can be met by }} \variation[level=2]{ 2. Nf6+ $18 \xskakcomment{ * with a discovered check }} (\variation[level=3]{ \xskakcomment{ or }} \variation[level=3]{ 2. Nc3+ $18 })

\mainline[level=1]{ 2. Nhg5 }
 In the game, Black resigned due to: 

\variation[level=2]{ 2. Nxf8 fxe4 3. Bc4 \xskakcomment{ is also winning. }} 

\mainline[level=1]{ 2... fxe4 3. Bxe4 }
 The bishop on b7 is doomed. (White could also go for the
king with 19.Bc4 \$18, forcing Black to give up his queen.) 




\clearpage 
\newpage 

\section{46.??.??}
\fenboard{r1b1k2r/2qp1ppp/ppnbpn2/8/2PNP3/P1N1BP2/1P4PP/2RQKB1R w Kkq - 0 1}
\begin{center}
\showboard 
\end{center}
\clearpage 
\newpage 
\mainline{1. Ndb5 $1 axb5 2. Nxb5 Bg3+ { Regaining the pawn
does little to alleviate the real problems of the black position - his
weaknesses on the dark squares (and the b6-pawn) and lack of development. } 3. hxg3 Qxg3+ 4. Bf2 $18 { * }}\fenboard{r1b1k2r/2qp1ppp/ppnbpn2/8/2PNP3/P1N1BP2/1P4PP/2RQKB1R w Kkq - 0 1}
\begin{center}
\showboard 
\end{center}
\fenboard{r1b1k2r/2qp1ppp/ppnbpn2/8/2PNP3/P1N1BP2/1P4PP/2RQKB1R w Kkq - 0 1}
\mainline[level=1]{1. Ndb5 $1 }

\variation[level=2]{ 1. Ncb5 \xskakcomment{ is the same. }} 

\mainline[level=1]{ 1... axb5 2. Nxb5 Bg3+ }
 Regaining the pawn
does little to alleviate the real problems of the black position - his
weaknesses on the dark squares (and the b6-pawn) and lack of development. 

\mainline[level=1]{ 3. hxg3 Qxg3+ 4. Bf2 $18 }
 * 




\clearpage 
\newpage 

\section{47.??.??}
\fenboard{4r1k1/6qp/pp4p1/2pP4/4Pp2/1P6/P1R3PP/4Q2K b - - 0 1}
\begin{center}
\showinverseboard 
\end{center}
\clearpage 
\newpage 
\mainline{1... Rxe4 $1 2. Qxe4 Qa1+ { * With back-rank mate. }}\fenboard{4r1k1/6qp/pp4p1/2pP4/4Pp2/1P6/P1R3PP/4Q2K b - - 0 1}
\begin{center}
\showinverseboard 
\end{center}
\fenboard{4r1k1/6qp/pp4p1/2pP4/4Pp2/1P6/P1R3PP/4Q2K b - - 0 1}
\mainline[level=1]{1... Rxe4 $1 2. Qxe4 Qa1+ }
 * With back-rank mate. 




\clearpage 
\newpage 

\section{48.??.??}
\fenboard{2bq1r2/4bpk1/4pp1N/7Q/1p1p4/3B4/PP3PPP/R5K1 w - - 0 1}
\begin{center}
\showboard 
\end{center}
\clearpage 
\newpage 
\mainline{1. Nxf7 $1 Rxf7 2. Qh7+ $1 Kf8 3. Qh8# { mate * }}\fenboard{2bq1r2/4bpk1/4pp1N/7Q/1p1p4/3B4/PP3PPP/R5K1 w - - 0 1}
\begin{center}
\showboard 
\end{center}
\fenboard{2bq1r2/4bpk1/4pp1N/7Q/1p1p4/3B4/PP3PPP/R5K1 w - - 0 1}
\mainline[level=1]{1. Nxf7 $1 Rxf7 2. Qh7+ $1 Kf8 3. Qh8# }
 mate * 




\clearpage 
\newpage 

\section{49.??.??}
\fenboard{r3r1k1/3b2pp/2p5/p1RpPp2/3Q1P2/1q2P1P1/6BP/R5K1 w - - 0 1}
\begin{center}
\showboard 
\end{center}
\clearpage 
\newpage 
\mainline{1. Rxd5 $1 { White wins a pawn, since } 1... cxd5 2. Bxd5+ $18 { * is a fork. }}\fenboard{r3r1k1/3b2pp/2p5/p1RpPp2/3Q1P2/1q2P1P1/6BP/R5K1 w - - 0 1}
\begin{center}
\showboard 
\end{center}
\fenboard{r3r1k1/3b2pp/2p5/p1RpPp2/3Q1P2/1q2P1P1/6BP/R5K1 w - - 0 1}
\mainline[level=1]{1. Rxd5 $1 }
 White wins a pawn, since 

\mainline[level=1]{ 1... cxd5 2. Bxd5+ $18 }
 * is a fork. 




\clearpage 
\newpage 

\section{50.??.??}
\fenboard{r3br1k/pp5p/4B1p1/4NpP1/P2Pn3/q1PQ3R/7P/3R2K1 w - - 0 1}
\begin{center}
\showboard 
\end{center}
\clearpage 
\newpage 
\mainline{1. Rxh7+ $1 Kxh7 2. Qh3+ Kg7 3. Qh6# { mate * }}\fenboard{r3br1k/pp5p/4B1p1/4NpP1/P2Pn3/q1PQ3R/7P/3R2K1 w - - 0 1}
\begin{center}
\showboard 
\end{center}
\fenboard{r3br1k/pp5p/4B1p1/4NpP1/P2Pn3/q1PQ3R/7P/3R2K1 w - - 0 1}
\mainline[level=1]{1. Rxh7+ $1 Kxh7 2. Qh3+ Kg7 3. Qh6# }
 mate * 




\clearpage 
\newpage 

\section{51.??.??}
\fenboard{5rk1/ppp2ppp/6q1/2b1P3/3r4/2N1BQ1b/PP3PPP/R3R1K1 b - - 0 1}
\begin{center}
\showinverseboard 
\end{center}
\clearpage 
\newpage 
\mainline{1... Bxg2 $1 { Highlighting the usefulness of the active d4-rook. } 2. Qg3 Rg4 { * } 3. Bxc5 Rxg3 4. fxg3 Re8 $19}\fenboard{5rk1/ppp2ppp/6q1/2b1P3/3r4/2N1BQ1b/PP3PPP/R3R1K1 b - - 0 1}
\begin{center}
\showinverseboard 
\end{center}
\fenboard{5rk1/ppp2ppp/6q1/2b1P3/3r4/2N1BQ1b/PP3PPP/R3R1K1 b - - 0 1}
\mainline[level=1]{1... Bxg2 $1 }
 Highlighting the usefulness of the active d4-rook. 

\mainline[level=1]{ 2. Qg3 }

\variation[level=2]{ 2. Qxg2 Rg4 $19 \xskakcomment{ * pins the queen. }} 

\mainline[level=1]{ 2... Rg4 }
 * 




\clearpage 
\newpage 

\section{52.??.??}
\fenboard{4r2k/ppp3pp/8/1PPb1p2/3P1P1b/P1Q2p1P/7R/R4KBq b - - 0 1}
\begin{center}
\showinverseboard 
\end{center}
\clearpage 
\newpage 
\mainline{1... Qg2+ $1 2. Rxg2 fxg2# { mate * }}\fenboard{4r2k/ppp3pp/8/1PPb1p2/3P1P1b/P1Q2p1P/7R/R4KBq b - - 0 1}
\begin{center}
\showinverseboard 
\end{center}
\fenboard{4r2k/ppp3pp/8/1PPb1p2/3P1P1b/P1Q2p1P/7R/R4KBq b - - 0 1}
\mainline[level=1]{1... Qg2+ $1 2. Rxg2 fxg2# }
 mate * 




\clearpage 
\newpage 

\section{53.??.??}
\fenboard{3r1b1k/pp4p1/2p1Qp2/5N2/PP2Pp2/2Pq4/5PKP/5R2 b - - 0 1}
\begin{center}
\showinverseboard 
\end{center}
\clearpage 
\newpage 
\mainline{1... f3+ 2. Kg1 Qxf1+ $1 3. Kxf1 Rd1# { mate * }}\fenboard{3r1b1k/pp4p1/2p1Qp2/5N2/PP2Pp2/2Pq4/5PKP/5R2 b - - 0 1}
\begin{center}
\showinverseboard 
\end{center}
\fenboard{3r1b1k/pp4p1/2p1Qp2/5N2/PP2Pp2/2Pq4/5PKP/5R2 b - - 0 1}
\mainline[level=1]{1... f3+ 2. Kg1 Qxf1+ $1 3. Kxf1 Rd1# }
 mate * 




\clearpage 
\newpage 

\section{54.??.??}
\fenboard{r1b2rk1/pp4pp/2pb4/3p1pq1/2PP4/1N1BPR2/PPQ3PP/4R1K1 b - - 0 1}
\begin{center}
\showinverseboard 
\end{center}
\clearpage 
\newpage 
\mainline{1... Bxh2+ $1 2. Kxh2 Qh4+ 3. Rh3 Qxe1 $17 { * Undefended pieces... }}\fenboard{r1b2rk1/pp4pp/2pb4/3p1pq1/2PP4/1N1BPR2/PPQ3PP/4R1K1 b - - 0 1}
\begin{center}
\showinverseboard 
\end{center}
\fenboard{r1b2rk1/pp4pp/2pb4/3p1pq1/2PP4/1N1BPR2/PPQ3PP/4R1K1 b - - 0 1}
\mainline[level=1]{1... Bxh2+ $1 2. Kxh2 Qh4+ 3. Rh3 Qxe1 $17 }
 * Undefended pieces... 




\clearpage 
\newpage 

\section{55.??.??}
\fenboard{q6r/1b4bp/4k1p1/1p2Pn2/2pPp1Q1/2P5/1P1N2PP/2B2RK1 w - - 0 1}
\begin{center}
\showboard 
\end{center}
\clearpage 
\newpage 
\mainline{1. Rxf5 $1 gxf5 2. Qxg7 $16 { * Considering the weak dark squares and exposed king, his
position is clearly superior. }}\fenboard{q6r/1b4bp/4k1p1/1p2Pn2/2pPp1Q1/2P5/1P1N2PP/2B2RK1 w - - 0 1}
\begin{center}
\showboard 
\end{center}
\fenboard{q6r/1b4bp/4k1p1/1p2Pn2/2pPp1Q1/2P5/1P1N2PP/2B2RK1 w - - 0 1}

 White reduces the material deficit from a piece to an exchange with: 

\mainline[level=1]{ 1. Rxf5 $1 gxf5 2. Qxg7 $16 }
 * Considering the weak dark squares and exposed king, his
position is clearly superior. 




\clearpage 
\newpage 

\section{56.??.??}
\fenboard{5rk1/1b1p1ppp/1qr1p3/p2pP3/P4P2/Q2B4/1PP3PP/R4R1K w - - 0 1}
\begin{center}
\showboard 
\end{center}
\clearpage 
\newpage 
\mainline{1. Bxh7+ $1 { Forcing the king to leave the defence of the rook. } 1... Kxh7 2. Qxf8 $18 { * }}\fenboard{5rk1/1b1p1ppp/1qr1p3/p2pP3/P4P2/Q2B4/1PP3PP/R4R1K w - - 0 1}
\begin{center}
\showboard 
\end{center}
\fenboard{5rk1/1b1p1ppp/1qr1p3/p2pP3/P4P2/Q2B4/1PP3PP/R4R1K w - - 0 1}
\mainline[level=1]{1. Bxh7+ $1 }
 Forcing the king to leave the defence of the rook. 

\mainline[level=1]{ 1... Kxh7 2. Qxf8 $18 }
 * 




\clearpage 
\newpage 

\section{57.??.??}
\fenboard{5r1k/pp4pp/2p5/8/4n3/5NPQ/P3Bq1P/4R2K b - - 0 1}
\begin{center}
\showinverseboard 
\end{center}
\clearpage 
\newpage 
\mainline{1... Qxe1+ $1 2. Nxe1 Nf2+ 3. Kg2 Nxh3 4. Nf3 Rxf3 { Black could also have achieved a winning position by going for the
a-pawn, but this is simple. } 5. Bxf3 Ng5 $19 { * }}\fenboard{5r1k/pp4pp/2p5/8/4n3/5NPQ/P3Bq1P/4R2K b - - 0 1}
\begin{center}
\showinverseboard 
\end{center}
\fenboard{5r1k/pp4pp/2p5/8/4n3/5NPQ/P3Bq1P/4R2K b - - 0 1}
\mainline[level=1]{1... Qxe1+ $1 }

\variation[level=2]{ 1... Rxf3 2. Rf1 Qxf1+ $1 } (\variation[level=3]{ 2... Qxe2 $4 3. Qc8+ Rf8 4. Qxf8# \xskakcomment{ mate }})
\variation[level=2]{ 3. Bxf1 Nf2+ 4. Kg2 Nxh3 5. Kxf3 Ng5+ \xskakcomment{ is also winning for Black,
although it doesn't look so simple with White's king active after }} \variation[level=2]{ 6. Kf4 Nf7 7. Kf5 \xskakcomment{ (one point). }} 

\mainline[level=1]{ 2. Nxe1 Nf2+ 3. Kg2 Nxh3 4. Nf3 }

\variation[level=2]{ 4. Kxh3 Re8 $1 $19 \xskakcomment{ * }} 

\mainline[level=1]{ 4... Rxf3 }
 Black could also have achieved a winning position by going for the
a-pawn, but this is simple. 

\mainline[level=1]{ 5. Bxf3 }

\variation[level=2]{ 5. Kxf3 Ng1+ $18 \xskakcomment{ * }} 

\mainline[level=1]{ 5... Ng5 $19 }
 * 




\clearpage 
\newpage 

\section{58.??.??}
\fenboard{r4rk1/pp2bppp/1qp1p3/4Pb2/Q1P1nB2/2N5/PP1RBPPP/5RK1 w - - 0 1}
\begin{center}
\showboard 
\end{center}
\clearpage 
\newpage 
\mainline{1. Nxe4 $1 Bxe4 2. c5 { A discovered attack, winning the bishop. } 2... Bxc5 3. Qxe4 $18 { * }}\fenboard{r4rk1/pp2bppp/1qp1p3/4Pb2/Q1P1nB2/2N5/PP1RBPPP/5RK1 w - - 0 1}
\begin{center}
\showboard 
\end{center}
\fenboard{r4rk1/pp2bppp/1qp1p3/4Pb2/Q1P1nB2/2N5/PP1RBPPP/5RK1 w - - 0 1}
\mainline[level=1]{1. Nxe4 $1 Bxe4 2. c5 }
 A discovered attack, winning the bishop. 

\mainline[level=1]{ 2... Bxc5 3. Qxe4 $18 }
 * 




\clearpage 
\newpage 

\section{59.??.??}
\fenboard{2r5/pp1bkp1Q/2nbpq2/3p1p2/3P1Pr1/2NBP1N1/PP4PP/2R2RK1 w - - 0 1}
\begin{center}
\showboard 
\end{center}
\clearpage 
\newpage 
\mainline{1. Nxf5+ $1 exf5 2. Nxd5+ $18 { * It's a fork. }}\fenboard{2r5/pp1bkp1Q/2nbpq2/3p1p2/3P1Pr1/2NBP1N1/PP4PP/2R2RK1 w - - 0 1}
\begin{center}
\showboard 
\end{center}
\fenboard{2r5/pp1bkp1Q/2nbpq2/3p1p2/3P1Pr1/2NBP1N1/PP4PP/2R2RK1 w - - 0 1}
\mainline[level=1]{1. Nxf5+ $1 }

\variation[level=2]{ \xskakcomment{ Not }} \variation[level=2]{ 1. Bxf5 $2 Rh4 } (\variation[level=3]{ 1... Rxg3 $17 })
\variation[level=2]{ 2. Nh5 Qh8 \xskakcomment{ and Black wins a piece. }} 

\mainline[level=1]{ 1... exf5 }

\variation[level=2]{ \xskakcomment{ Instead Alekhine tried }} \variation[level=2]{ 1... Kf8 \xskakcomment{ but White has several ways to win, for instance }} \variation[level=2]{ 2. Nxd6 Rh4 \xskakcomment{ trapping the queen, but to no avail. }} \variation[level=2]{ 3. Qxf7+ Qxf7 4. Nxf7 $18 \xskakcomment{ White has won three pawns (\whiteWins, 42 moves). }} 

\mainline[level=1]{ 2. Nxd5+ $18 }
 * It's a fork. 




\clearpage 
\newpage 

\section{60.??.??}
\fenboard{r1bqr1k1/1p1nbpp1/p1p3p1/3p4/3P1B2/2NBP2P/PP3PP1/2RQ1RK1 w - - 0 1}
\begin{center}
\showboard 
\end{center}
\clearpage 
\newpage 
\mainline{1. Nxd5 $1 cxd5 $2 2. Bc7 $18 { * The queen is trapped. }}\fenboard{r1bqr1k1/1p1nbpp1/p1p3p1/3p4/3P1B2/2NBP2P/PP3PP1/2RQ1RK1 w - - 0 1}
\begin{center}
\showboard 
\end{center}
\fenboard{r1bqr1k1/1p1nbpp1/p1p3p1/3p4/3P1B2/2NBP2P/PP3PP1/2RQ1RK1 w - - 0 1}
\mainline[level=1]{1. Nxd5 $1 cxd5 $2 }

\variation[level=2]{ \xskakcomment{ Yates avoided this with }} \variation[level=2]{ 1... Bd6 \xskakcomment{ but after }} \variation[level=2]{ 2. Bxd6 \xskakcomment{ * }} \variation[level=2]{ 2... cxd5 $18 \xskakcomment{ he had no compensation whatsoever for the lost pawn. }} 

\mainline[level=1]{ 2. Bc7 $18 }
 * The queen is trapped. 




\clearpage 
\newpage 

\section{61.??.??}
\fenboard{3r2k1/q1p1nppp/p3n3/1pb1p3/4P2N/1PP3PP/PBB1QPK1/7R b - - 0 1}
\begin{center}
\showinverseboard 
\end{center}
\clearpage 
\newpage 
\mainline{1... Bxf2 $1 $15 { White can't take back due to: } 2. Qxf2 $2 Qxf2+ 3. Kxf2 Rd2+ 4. Ke3 Rxc2 $19 { * }}\fenboard{3r2k1/q1p1nppp/p3n3/1pb1p3/4P2N/1PP3PP/PBB1QPK1/7R b - - 0 1}
\begin{center}
\showinverseboard 
\end{center}
\fenboard{3r2k1/q1p1nppp/p3n3/1pb1p3/4P2N/1PP3PP/PBB1QPK1/7R b - - 0 1}
\mainline[level=1]{1... Bxf2 $1 $15 }
 White can't take back due to: 

\mainline[level=1]{ 2. Qxf2 $2 Qxf2+ 3. Kxf2 Rd2+ 4. Ke3 Rxc2 $19 }
 * 




\clearpage 
\newpage 

\section{62.??.??}
\fenboard{1Q6/p4pkp/3p2p1/3P4/q7/P3rBbP/6P1/5R1K b - - 0 1}
\begin{center}
\showinverseboard 
\end{center}
\clearpage 
\newpage 
\mainline{1... Rxf3 $1 2. gxf3 Qc2 { * 0-1 There is no defence against the mate. }}\fenboard{1Q6/p4pkp/3p2p1/3P4/q7/P3rBbP/6P1/5R1K b - - 0 1}
\begin{center}
\showinverseboard 
\end{center}
\fenboard{1Q6/p4pkp/3p2p1/3P4/q7/P3rBbP/6P1/5R1K b - - 0 1}
\mainline[level=1]{1... Rxf3 $1 2. gxf3 }

\variation[level=2]{ 2. Rxf3 Qd1+ 3. Rf1 Qxf1# \xskakcomment{ mate * }} 

\mainline[level=1]{ 2... Qc2 }
 * \blackWins There is no defence against the mate. 




\clearpage 
\newpage 

\section{63.??.??}
\fenboard{r1bqk2r/p1pn1pp1/1p2pn1p/8/3P4/B1PB4/P1P1QPPP/R3K1NR w KQkq - 0 1}
\begin{center}
\showboard 
\end{center}
\clearpage 
\newpage 
\mainline{1. Qxe6+ $1 fxe6 2. Bg6# { mate * }}\fenboard{r1bqk2r/p1pn1pp1/1p2pn1p/8/3P4/B1PB4/P1P1QPPP/R3K1NR w KQkq - 0 1}
\begin{center}
\showboard 
\end{center}
\fenboard{r1bqk2r/p1pn1pp1/1p2pn1p/8/3P4/B1PB4/P1P1QPPP/R3K1NR w KQkq - 0 1}
\mainline[level=1]{1. Qxe6+ $1 fxe6 2. Bg6# }
 mate * 




\clearpage 
\newpage 

\section{64.??.??}
\fenboard{r2qk2r/1p1b1pp1/p1pBpn1p/2P1N3/1n1P4/3B4/PPQ2PPP/2KR3R w kq - 0 1}
\begin{center}
\showboard 
\end{center}
\clearpage 
\newpage 
\mainline{1. Bg6 $1 fxg6 2. Qxg6# { mate * }}\fenboard{r2qk2r/1p1b1pp1/p1pBpn1p/2P1N3/1n1P4/3B4/PPQ2PPP/2KR3R w kq - 0 1}
\begin{center}
\showboard 
\end{center}
\fenboard{r2qk2r/1p1b1pp1/p1pBpn1p/2P1N3/1n1P4/3B4/PPQ2PPP/2KR3R w kq - 0 1}
\mainline[level=1]{1. Bg6 $1 fxg6 }

\variation[level=2]{ 1... Qa5 \xskakcomment{ avoids mate, but Black is completely lost after }} \variation[level=2]{ 2. Bxf7+ Kd8 \xskakcomment{ followed by any decent queen move. }} 

\mainline[level=1]{ 2. Qxg6# }
 mate * 




\clearpage 
\newpage 

\section{65.??.??}
\fenboard{1r2r1k1/p1pbqppp/Q2b1n2/3p4/P2P4/2P5/1P2BPPP/R1B1KN1R b KQ - 0 1}
\begin{center}
\showinverseboard 
\end{center}
\clearpage 
\newpage 
\mainline{1... Bb5 $1 2. axb5 Qxe2# { mate * }}\fenboard{1r2r1k1/p1pbqppp/Q2b1n2/3p4/P2P4/2P5/1P2BPPP/R1B1KN1R b KQ - 0 1}
\begin{center}
\showinverseboard 
\end{center}
\fenboard{1r2r1k1/p1pbqppp/Q2b1n2/3p4/P2P4/2P5/1P2BPPP/R1B1KN1R b KQ - 0 1}
\mainline[level=1]{1... Bb5 $1 2. axb5 Qxe2# }
 mate * 




\clearpage 
\newpage 

\section{66.??.??}
\fenboard{1r4k1/pqp2pbp/2Q2np1/1N2p3/8/1P5P/PBP2PP1/3R2K1 w - - 0 1}
\begin{center}
\showboard 
\end{center}
\clearpage 
\newpage 
\mainline{1. Rd8+ $1 Bf8 2. Qxf6 $18 { * }}\fenboard{1r4k1/pqp2pbp/2Q2np1/1N2p3/8/1P5P/PBP2PP1/3R2K1 w - - 0 1}
\begin{center}
\showboard 
\end{center}
\fenboard{1r4k1/pqp2pbp/2Q2np1/1N2p3/8/1P5P/PBP2PP1/3R2K1 w - - 0 1}
\mainline[level=1]{1. Rd8+ $1 }

\variation[level=2]{\xskakcomment{\noindent\textbf{a)} } 1. Qxb7 Rxb7 2. Bxe5 \xskakcomment{ is also good (White will soon be two pawns
up) but only the game move forces resignation. }} 

\variation[level=2]{\xskakcomment{\noindent\textbf{b)} } \xskakcomment{ Note that after }} \variation[level=2]{ 1. Qxb7 Rxb7 \xskakcomment{ White should avoid pinning the bishop with }} \variation[level=2]{ 2. Rd8+ Bf8 3. Ba3 \xskakcomment{ since Black can struggle on with: }} \variation[level=2]{ 3... Nd7 $1 \xskakcomment{ Nevertheless, White gets a rook ending a pawn up that looks winning. }} 

\mainline[level=1]{ 1... Bf8 }

\variation[level=2]{ 1... Rxd8 2. Qxb7 $18 \xskakcomment{ * }} 

\mainline[level=1]{ 2. Qxf6 $18 }
 * 




\clearpage 
\newpage 

\section{67.??.??}
\fenboard{rn1qkb1r/pp3p1b/2p1pnpp/4N3/2B4P/6N1/PPPPQPP1/R1B1K2R w KQkq - 0 1}
\begin{center}
\showboard 
\end{center}
\clearpage 
\newpage 
\mainline{1. Nxf7 $1 { Classical destruction of the f7-e6 formation. } 1... Kxf7 2. Qxe6+ { 1-0 }}\fenboard{rn1qkb1r/pp3p1b/2p1pnpp/4N3/2B4P/6N1/PPPPQPP1/R1B1K2R w KQkq - 0 1}
\begin{center}
\showboard 
\end{center}
\fenboard{rn1qkb1r/pp3p1b/2p1pnpp/4N3/2B4P/6N1/PPPPQPP1/R1B1K2R w KQkq - 0 1}
\mainline[level=1]{1. Nxf7 $1 }
 Classical destruction of the f7-e6 formation. 

\mainline[level=1]{ 1... Kxf7 2. Qxe6+ }
 \whiteWins 

\variation[level=2]{ 2. Qxe6+ \xskakcomment{ Black foresaw }} \variation[level=2]{ 2... Kg7 3. Qf7# \xskakcomment{ mate * }} 




\clearpage 
\newpage 

\section{68.??.??}
\fenboard{3n4/2prR1pk/p2r1p1p/1p5P/5P1P/P1B2K2/1PP5/4R3 w - - 0 1}
\begin{center}
\showboard 
\end{center}
\clearpage 
\newpage 
\mainline{1. Bxf6 $1 $16 { * White wins a pawn since Black cannot take back on f6. }}\fenboard{3n4/2prR1pk/p2r1p1p/1p5P/5P1P/P1B2K2/1PP5/4R3 w - - 0 1}
\begin{center}
\showboard 
\end{center}
\fenboard{3n4/2prR1pk/p2r1p1p/1p5P/5P1P/P1B2K2/1PP5/4R3 w - - 0 1}
\mainline[level=1]{1. Bxf6 $1 $16 }
 * White wins a pawn since Black cannot take back on f6. 




\clearpage 
\newpage 

\section{69.??.??}
\fenboard{6r1/1p2R3/p5k1/2p5/4Nr1P/8/PP5P/6K1 b - - 0 1}
\begin{center}
\showinverseboard 
\end{center}
\clearpage 
\newpage 
\mainline{1... Rxe4 $1 { Finishing off a winning position. } 2. Rxe4 Kf5+ 3. Kf2 Kxe4 $19 { * }}\fenboard{6r1/1p2R3/p5k1/2p5/4Nr1P/8/PP5P/6K1 b - - 0 1}
\begin{center}
\showinverseboard 
\end{center}
\fenboard{6r1/1p2R3/p5k1/2p5/4Nr1P/8/PP5P/6K1 b - - 0 1}
\mainline[level=1]{1... Rxe4 $1 }
 Finishing off a winning position. 

\mainline[level=1]{ 2. Rxe4 Kf5+ 3. Kf2 Kxe4 $19 }
 * 




\clearpage 
\newpage 

\section{70.??.??}
\fenboard{r1b2rk1/ppqnbppp/2p1pn2/3p2B1/2PP4/2NBPN2/PPQ2PPP/R3K2R w KQ - 0 1}
\begin{center}
\showboard 
\end{center}
\clearpage 
\newpage 
\mainline{1. Bxh7+ Nxh7 2. Bxe7 { * } 2... Re8 3. Bh4 dxc4 $16 { White was not worried about
giving back the pawn, since he gets a strong centre. If he was worried, he
could have started with cxd5+/-, which is equally strong. (1-0, 40 moves) }}\fenboard{r1b2rk1/ppqnbppp/2p1pn2/3p2B1/2PP4/2NBPN2/PPQ2PPP/R3K2R w KQ - 0 1}
\begin{center}
\showboard 
\end{center}
\fenboard{r1b2rk1/ppqnbppp/2p1pn2/3p2B1/2PP4/2NBPN2/PPQ2PPP/R3K2R w KQ - 0 1}
\mainline[level=1]{1. Bxh7+ Nxh7 2. Bxe7 }
 * 

\mainline[level=1]{ 2... Re8 3. Bh4 dxc4 $16 }
 White was not worried about
giving back the pawn, since he gets a strong centre. If he was worried, he
could have started with cxd5+/-, which is equally strong. (\whiteWins, 40 moves) 




\clearpage 
\newpage 

\section{71.??.??}
\fenboard{r1b1qrk1/pppp1ppp/1bn3n1/3Np1BQ/2B1P3/3P1N2/PPP2PPP/R3K2R w KQ - 0 1}
\begin{center}
\showboard 
\end{center}
\clearpage 
\newpage 
\mainline{1. Nf6+ $1 gxf6 2. Bxf6 { * 1-0 There is no defence against the mate on g7. }}\fenboard{r1b1qrk1/pppp1ppp/1bn3n1/3Np1BQ/2B1P3/3P1N2/PPP2PPP/R3K2R w KQ - 0 1}
\begin{center}
\showboard 
\end{center}
\fenboard{r1b1qrk1/pppp1ppp/1bn3n1/3Np1BQ/2B1P3/3P1N2/PPP2PPP/R3K2R w KQ - 0 1}
\mainline[level=1]{1. Nf6+ $1 gxf6 2. Bxf6 }
 * \whiteWins There is no defence against the mate on g7. 




\clearpage 
\newpage 

\section{72.??.??}
\fenboard{r2qrbk1/1bp2ppp/p2p1n2/2p2NB1/4P3/2N2Q2/PPP2PPP/R3R1K1 w - - 0 1}
\begin{center}
\showboard 
\end{center}
\clearpage 
\newpage 
\mainline{1. Nh6+ $1 gxh6 2. Bxf6 $18 { * White checkmates or wins the queen. }}\fenboard{r2qrbk1/1bp2ppp/p2p1n2/2p2NB1/4P3/2N2Q2/PPP2PPP/R3R1K1 w - - 0 1}
\begin{center}
\showboard 
\end{center}
\fenboard{r2qrbk1/1bp2ppp/p2p1n2/2p2NB1/4P3/2N2Q2/PPP2PPP/R3R1K1 w - - 0 1}
\mainline[level=1]{1. Nh6+ $1 gxh6 2. Bxf6 $18 }
 * White checkmates or wins the queen. 




\clearpage 
\newpage 

\section{73.??.??}
\fenboard{r4r1k/ppn1NBpp/4b3/4P3/3p1R2/1P6/P1P3PP/R5K1 w - - 0 1}
\begin{center}
\showboard 
\end{center}
\clearpage 
\newpage 
\mainline{1. Ng6+ $1 hxg6 2. Rh4# { mate * }}\fenboard{r4r1k/ppn1NBpp/4b3/4P3/3p1R2/1P6/P1P3PP/R5K1 w - - 0 1}
\begin{center}
\showboard 
\end{center}
\fenboard{r4r1k/ppn1NBpp/4b3/4P3/3p1R2/1P6/P1P3PP/R5K1 w - - 0 1}
\mainline[level=1]{1. Ng6+ $1 hxg6 2. Rh4# }
 mate * 




\clearpage 
\newpage 

\section{74.??.??}
\fenboard{2r2rk1/pp1bnp2/3q1n1Q/3p1P2/4p2N/1BPP4/P1P3PP/R4RK1 b - - 0 1}
\begin{center}
\showinverseboard 
\end{center}
\clearpage 
\newpage 
\mainline{1... Qxh2+ $1 { Exchanging queens and consolidating the material advantage. } 2. Kxh2 Ng4+ 3. Kg3 Nxh6 $19 { * }}\fenboard{2r2rk1/pp1bnp2/3q1n1Q/3p1P2/4p2N/1BPP4/P1P3PP/R4RK1 b - - 0 1}
\begin{center}
\showinverseboard 
\end{center}
\fenboard{2r2rk1/pp1bnp2/3q1n1Q/3p1P2/4p2N/1BPP4/P1P3PP/R4RK1 b - - 0 1}
\mainline[level=1]{1... Qxh2+ $1 }
 Exchanging queens and consolidating the material advantage. 

\mainline[level=1]{ 2. Kxh2 Ng4+ 3. Kg3 Nxh6 $19 }
 * 




\clearpage 
\newpage 

\section{75.??.??}
\fenboard{5rk1/pbp2ppp/qr6/8/5Q2/1PP5/P4PP1/R1B2RK1 b - - 0 1}
\begin{center}
\showinverseboard 
\end{center}
\clearpage 
\newpage 
\mainline{1... Bxg2 $1 2. Kxg2 Rg6+ $18 { * Black wins the queen or the rook on f1. }}\fenboard{5rk1/pbp2ppp/qr6/8/5Q2/1PP5/P4PP1/R1B2RK1 b - - 0 1}
\begin{center}
\showinverseboard 
\end{center}
\fenboard{5rk1/pbp2ppp/qr6/8/5Q2/1PP5/P4PP1/R1B2RK1 b - - 0 1}
\mainline[level=1]{1... Bxg2 $1 2. Kxg2 Rg6+ $18 }
 * Black wins the queen or the rook on f1. 




\clearpage 
\newpage 

\section{76.??.??}
\fenboard{r1b2rk1/pp1p1ppp/2n2n2/q7/2P5/P1N2NP1/3QPKBP/R1B4R b - - 0 1}
\begin{center}
\showinverseboard 
\end{center}
\clearpage 
\newpage 
\mainline{1... Qxc3 $1 2. Qxc3 Ne4+ { Black wins back his sacrificed piece, leaving him a pawn up. } 3. Kf1 Nxc3 $15 { * }}\fenboard{r1b2rk1/pp1p1ppp/2n2n2/q7/2P5/P1N2NP1/3QPKBP/R1B4R b - - 0 1}
\begin{center}
\showinverseboard 
\end{center}
\fenboard{r1b2rk1/pp1p1ppp/2n2n2/q7/2P5/P1N2NP1/3QPKBP/R1B4R b - - 0 1}
\mainline[level=1]{1... Qxc3 $1 2. Qxc3 Ne4+ }
 Black wins back his sacrificed piece, leaving him a pawn up. 

\mainline[level=1]{ 3. Kf1 Nxc3 $15 }
 * 




\clearpage 
\newpage 

\section{77.??.??}
\fenboard{3Q4/p4pkp/1p3np1/2q5/4p3/4P1N1/PP3PPP/6K1 w - - 0 1}
\begin{center}
\showboard 
\end{center}
\clearpage 
\newpage 
\mainline{1. Qxf6+ $1 Kxf6 2. Nxe4+ Ke5 3. Nxc5 bxc5 { * The pawn ending is winning,
unless Black gets time to collect the queenside pawns. And he doesn't after: } 4. Kf1}\fenboard{3Q4/p4pkp/1p3np1/2q5/4p3/4P1N1/PP3PPP/6K1 w - - 0 1}
\begin{center}
\showboard 
\end{center}
\fenboard{3Q4/p4pkp/1p3np1/2q5/4p3/4P1N1/PP3PPP/6K1 w - - 0 1}
\mainline[level=1]{1. Qxf6+ $1 Kxf6 2. Nxe4+ Ke5 3. Nxc5 bxc5 }
 * The pawn ending is winning,
unless Black gets time to collect the queenside pawns. And he doesn't after: 

\mainline[level=1]{ 4. Kf1 }

\variation[level=2]{ \xskakcomment{ or }} \variation[level=2]{ 4. f3 }




\clearpage 
\newpage 

\section{78.??.??}
\fenboard{3r4/p2q1pkp/1pn1bnp1/2p1p3/P1N1P3/1PP1Q1PP/5PK1/4RBN1 b - - 0 1}
\begin{center}
\showinverseboard 
\end{center}
\clearpage 
\newpage 
\mainline{1... Nxe4 2. Qxe4 Bd5 $19 { * }}\fenboard{3r4/p2q1pkp/1pn1bnp1/2p1p3/P1N1P3/1PP1Q1PP/5PK1/4RBN1 b - - 0 1}
\begin{center}
\showinverseboard 
\end{center}
\fenboard{3r4/p2q1pkp/1pn1bnp1/2p1p3/P1N1P3/1PP1Q1PP/5PK1/4RBN1 b - - 0 1}
\mainline[level=1]{1... Nxe4 2. Qxe4 }

\variation[level=2]{ \xskakcomment{ If White steps out of the pin with }} \variation[level=2]{ 2. Kh2 \xskakcomment{ there are many moves that keep the pawn: }} \variation[level=2]{ 2... f5 }\\\variation[level=3]{\xskakcomment{\noindent\textbf{a)} } 2... Bd5 } \\\variation[level=3]{\xskakcomment{\noindent\textbf{b)} } \xskakcomment{ or even }} \variation[level=3]{ 2... Nf6 3. Nxe5 Nxe5 4. Qxe5 Bxb3 } 

\mainline[level=1]{ 2... Bd5 $19 }
 * 




\clearpage 
\newpage 

\section{79.??.??}
\fenboard{2kr4/p1p2ppp/3rb3/8/2P5/1R1BR3/P4PPP/5K2 b - - 0 1}
\begin{center}
\showinverseboard 
\end{center}
\clearpage 
\newpage 
\mainline{1... Rxd3 $1 2. Rexd3 Rxd3 3. Rxd3 Bxc4 4. Ke2 $19 { * An extra pawn in a pawn ending wins, unless there are
some special circumstances. }}\fenboard{2kr4/p1p2ppp/3rb3/8/2P5/1R1BR3/P4PPP/5K2 b - - 0 1}
\begin{center}
\showinverseboard 
\end{center}
\fenboard{2kr4/p1p2ppp/3rb3/8/2P5/1R1BR3/P4PPP/5K2 b - - 0 1}
\mainline[level=1]{1... Rxd3 $1 2. Rexd3 Rxd3 }

\variation[level=2]{ \xskakcomment{ Or }} \variation[level=2]{ 2... Bxc4 \xskakcomment{ first makes no difference. }} 

\mainline[level=1]{ 3. Rxd3 Bxc4 4. Ke2 $19 }
 * An extra pawn in a pawn ending wins, unless there are
some special circumstances. 




\clearpage 
\newpage 

\section{80.??.??}
\fenboard{6k1/3q3p/p1p3pQ/1p1p4/3P2RP/1P3P2/r3r1P1/5R1K b - - 0 1}
\begin{center}
\showinverseboard 
\end{center}
\clearpage 
\newpage 
\mainline{1... Rxg2 $1 2. Rxg2 Qh3+ 3. Kg1 Qxg2# { mate * }}\fenboard{6k1/3q3p/p1p3pQ/1p1p4/3P2RP/1P3P2/r3r1P1/5R1K b - - 0 1}
\begin{center}
\showinverseboard 
\end{center}
\fenboard{6k1/3q3p/p1p3pQ/1p1p4/3P2RP/1P3P2/r3r1P1/5R1K b - - 0 1}
\mainline[level=1]{1... Rxg2 $1 2. Rxg2 Qh3+ 3. Kg1 Qxg2# }
 mate * 

\variation[level=2]{ \xskakcomment{ Or }} \variation[level=2]{ 3... Rxg2# \xskakcomment{ mate. }} 




\clearpage 
\newpage 

\section{81.??.??}
\fenboard{6k1/6pp/p1p3r1/3p4/P2P1Pq1/1R2PR2/2Q1K1P1/7r b - - 0 1}
\begin{center}
\showinverseboard 
\end{center}
\clearpage 
\newpage 
\mainline{1... Qxg2+ 2. Rf2 Qxf2+ $1 3. Kxf2 Rh2+ { And Black takes the queen: } 4. Kf3 Rxc2 $19 { * }}\fenboard{6k1/6pp/p1p3r1/3p4/P2P1Pq1/1R2PR2/2Q1K1P1/7r b - - 0 1}
\begin{center}
\showinverseboard 
\end{center}
\fenboard{6k1/6pp/p1p3r1/3p4/P2P1Pq1/1R2PR2/2Q1K1P1/7r b - - 0 1}
\mainline[level=1]{1... Qxg2+ 2. Rf2 Qxf2+ $1 3. Kxf2 Rh2+ }
 And Black takes the queen: 

\mainline[level=1]{ 4. Kf3 Rxc2 $19 }
 * 




\clearpage 
\newpage 

\section{82.??.??}
\fenboard{1r2nrk1/p1p2pp1/4bb1p/3p4/q4B2/P1PB1Q1P/1P2NPP1/1R1R2K1 b - - 0 1}
\begin{center}
\showinverseboard 
\end{center}
\clearpage 
\newpage 
\mainline{1... Rxb2 $1 2. Rxb2 Qxd1+ $15 { * Black has won a pawn. }}\fenboard{1r2nrk1/p1p2pp1/4bb1p/3p4/q4B2/P1PB1Q1P/1P2NPP1/1R1R2K1 b - - 0 1}
\begin{center}
\showinverseboard 
\end{center}
\fenboard{1r2nrk1/p1p2pp1/4bb1p/3p4/q4B2/P1PB1Q1P/1P2NPP1/1R1R2K1 b - - 0 1}
\mainline[level=1]{1... Rxb2 $1 2. Rxb2 Qxd1+ $15 }
 * Black has won a pawn. 




\clearpage 
\newpage 

\section{83.??.??}
\fenboard{5rk1/p4ppp/b3p3/2n1N3/Pp2P3/1P1r4/3N1PPP/R2R2K1 b - - 0 1}
\begin{center}
\showinverseboard 
\end{center}
\clearpage 
\newpage 
\mainline{1... Rxd2 $1 2. Rxd2 Nxb3 $19 { * The extra pawn is decisive. }}\fenboard{5rk1/p4ppp/b3p3/2n1N3/Pp2P3/1P1r4/3N1PPP/R2R2K1 b - - 0 1}
\begin{center}
\showinverseboard 
\end{center}
\fenboard{5rk1/p4ppp/b3p3/2n1N3/Pp2P3/1P1r4/3N1PPP/R2R2K1 b - - 0 1}
\mainline[level=1]{1... Rxd2 $1 2. Rxd2 Nxb3 $19 }
 * The extra pawn is decisive. 




\clearpage 
\newpage 

\section{84.??.??}
\fenboard{2r1r2k/p4pp1/2pBnb1p/q1Pp4/3P4/P2R4/2Q1NPPP/3R2K1 b - - 0 1}
\begin{center}
\showinverseboard 
\end{center}
\clearpage 
\newpage 
\mainline{1... Nxd4 $1 2. Rxd4 Bxd4 { White cannot take in any way due to mate on e1: } 3. Rxd4 Qe1# { mate * }}\fenboard{2r1r2k/p4pp1/2pBnb1p/q1Pp4/3P4/P2R4/2Q1NPPP/3R2K1 b - - 0 1}
\begin{center}
\showinverseboard 
\end{center}
\fenboard{2r1r2k/p4pp1/2pBnb1p/q1Pp4/3P4/P2R4/2Q1NPPP/3R2K1 b - - 0 1}
\mainline[level=1]{1... Nxd4 $1 }

\variation[level=2]{ 1... Bxd4 \xskakcomment{ picks up the pawn, but not the exchange. Black is not
clearly winning yet (zero points). }} 

\mainline[level=1]{ 2. Rxd4 Bxd4 }
 White cannot take in any way due to mate on e1: 

\mainline[level=1]{ 3. Rxd4 Qe1# }
 mate * 




\clearpage 
\newpage 

\section{85.??.??}
\fenboard{r1bn1b1r/pp2k1pp/5p2/1B2p3/5B2/5N2/PPP2PPP/2KR3R w - - 0 1}
\begin{center}
\showboard 
\end{center}
\clearpage 
\newpage 
\mainline{1. Nxe5 $1 Ne6 2. Nd3 $1 $18 { * White is not only a pawn up, Black is also far behind in development. }}\fenboard{r1bn1b1r/pp2k1pp/5p2/1B2p3/5B2/5N2/PPP2PPP/2KR3R w - - 0 1}
\begin{center}
\showboard 
\end{center}
\fenboard{r1bn1b1r/pp2k1pp/5p2/1B2p3/5B2/5N2/PPP2PPP/2KR3R w - - 0 1}
\mainline[level=1]{1. Nxe5 $1 Ne6 }

\variation[level=2]{ 1... fxe5 2. Bg5+ $18 \xskakcomment{ * }} 

\mainline[level=1]{ 2. Nd3 $1 $18 }
 * White is not only a pawn up, Black is also far behind in development. 




\clearpage 
\newpage 

\section{86.??.??}
\fenboard{2r2rk1/5ppp/p1pp4/2p1n1q1/4P2b/1PN4P/PBPRQPP1/R5K1 b - - 0 1}
\begin{center}
\showinverseboard 
\end{center}
\clearpage 
\newpage 
\mainline{1... Nf3+ $1 { Black wins an exchange after: } 2. Qxf3 Qxd2 $19 { * }}\fenboard{2r2rk1/5ppp/p1pp4/2p1n1q1/4P2b/1PN4P/PBPRQPP1/R5K1 b - - 0 1}
\begin{center}
\showinverseboard 
\end{center}
\fenboard{2r2rk1/5ppp/p1pp4/2p1n1q1/4P2b/1PN4P/PBPRQPP1/R5K1 b - - 0 1}
\mainline[level=1]{1... Nf3+ $1 }
 Black wins an exchange after: 

\mainline[level=1]{ 2. Qxf3 Qxd2 $19 }
 * 




\clearpage 
\newpage 

\section{87.??.??}
\fenboard{8/2q3pk/1p2p2p/2n5/2B5/P3PQ2/5PKP/8 w - - 0 1}
\begin{center}
\showboard 
\end{center}
\clearpage 
\newpage 
\mainline{1. Bxe6 $1 Nxe6 { Keeping the knight and setting up a blockade on the e-file
was not realistic, since White has an f-pawn as well. } 2. Qf5+ Kh8 3. Qxe6 $16 { * White's winning chances should be bigger than Black's
drawing chances, although the game ended in a draw. }}\fenboard{8/2q3pk/1p2p2p/2n5/2B5/P3PQ2/5PKP/8 w - - 0 1}
\begin{center}
\showboard 
\end{center}
\fenboard{8/2q3pk/1p2p2p/2n5/2B5/P3PQ2/5PKP/8 w - - 0 1}
\mainline[level=1]{1. Bxe6 $1 Nxe6 }
 Keeping the knight and setting up a blockade on the e-file
was not realistic, since White has an f-pawn as well. 

\mainline[level=1]{ 2. Qf5+ }

\variation[level=2]{ \xskakcomment{ Or }} \variation[level=2]{ 2. Qe4+ }

\mainline[level=1]{ 2... Kh8 3. Qxe6 $16 }
 * White's winning chances should be bigger than Black's
drawing chances, although the game ended in a draw. 




\clearpage 
\newpage 

\section{88.??.??}
\fenboard{r7/3k3p/6p1/N1P2p2/1p2p2P/3bPP2/5KP1/R7 b - - 0 1}
\begin{center}
\showinverseboard 
\end{center}
\clearpage 
\newpage 
\mainline{1... Rxa5 $1 2. Rxa5 b3 $19 { * White loses the rook. }}\fenboard{r7/3k3p/6p1/N1P2p2/1p2p2P/3bPP2/5KP1/R7 b - - 0 1}
\begin{center}
\showinverseboard 
\end{center}
\fenboard{r7/3k3p/6p1/N1P2p2/1p2p2P/3bPP2/5KP1/R7 b - - 0 1}
\mainline[level=1]{1... Rxa5 $1 2. Rxa5 b3 $19 }
 * White loses the rook. 




\clearpage 
\newpage 

\section{89.??.??}
\fenboard{4rrk1/pp2p3/2pqP1p1/4Rp1p/P2P1n1P/6Q1/1P3PP1/1N1R2K1 b - - 0 1}
\begin{center}
\showinverseboard 
\end{center}
\clearpage 
\newpage 
\mainline{1... Qxe5 $1 { Removing the defender of the e2-square. } 2. dxe5 Ne2+ 3. Kh2 Nxg3 $19 { * }}\fenboard{4rrk1/pp2p3/2pqP1p1/4Rp1p/P2P1n1P/6Q1/1P3PP1/1N1R2K1 b - - 0 1}
\begin{center}
\showinverseboard 
\end{center}
\fenboard{4rrk1/pp2p3/2pqP1p1/4Rp1p/P2P1n1P/6Q1/1P3PP1/1N1R2K1 b - - 0 1}
\mainline[level=1]{1... Qxe5 $1 }
 Removing the defender of the e2-square. 

\mainline[level=1]{ 2. dxe5 Ne2+ 3. Kh2 Nxg3 $19 }
 * 




\clearpage 
\newpage 

\section{90.??.??}
\fenboard{3r2k1/2q1b2p/ppnpPpp1/2pB4/2P2PPB/PP1R3P/3Q4/6K1 w - - 0 1}
\begin{center}
\showboard 
\end{center}
\clearpage 
\newpage 
\mainline{1. Bxf6 $1 Bxf6 2. e7+ Kg7 3. exd8=Q $18 { * Opposite-coloured bishops normally
improve the drawing chances of the player who has sacrificed material, but
Black did not sacrifice the exchange here - he lost it. And without active
pieces, he cannot create anything on the dark squares. Instead White's active
pieces and advantage in terrain give him an even greater advantage. }}\fenboard{3r2k1/2q1b2p/ppnpPpp1/2pB4/2P2PPB/PP1R3P/3Q4/6K1 w - - 0 1}
\begin{center}
\showboard 
\end{center}
\fenboard{3r2k1/2q1b2p/ppnpPpp1/2pB4/2P2PPB/PP1R3P/3Q4/6K1 w - - 0 1}
\mainline[level=1]{1. Bxf6 $1 Bxf6 2. e7+ Kg7 3. exd8=Q $18 }
 * Opposite-coloured bishops normally
improve the drawing chances of the player who has sacrificed material, but
Black did not sacrifice the exchange here - he lost it. And without active
pieces, he cannot create anything on the dark squares. Instead White's active
pieces and advantage in terrain give him an even greater advantage. 




\clearpage 
\newpage 

\section{91.??.??}
\fenboard{3r2k1/p1q2pbp/1pn1p1p1/2p5/4P3/4B1P1/PPP1RPBP/2Q4K b - - 0 1}
\begin{center}
\showinverseboard 
\end{center}
\clearpage 
\newpage 
\mainline{1... Bxb2 $1 2. Qxb2 Rd1+ 3. Bf1 Rxf1+ 4. Kg2 Rd1 $17 { * Black has won a pawn (0-1, 35 moves). }}\fenboard{3r2k1/p1q2pbp/1pn1p1p1/2p5/4P3/4B1P1/PPP1RPBP/2Q4K b - - 0 1}
\begin{center}
\showinverseboard 
\end{center}
\fenboard{3r2k1/p1q2pbp/1pn1p1p1/2p5/4P3/4B1P1/PPP1RPBP/2Q4K b - - 0 1}
\mainline[level=1]{1... Bxb2 $1 2. Qxb2 Rd1+ 3. Bf1 Rxf1+ 4. Kg2 Rd1 $17 }
 * Black has won a pawn (\blackWins, 35 moves). 




\clearpage 
\newpage 

\section{92.??.??}
\fenboard{1k6/pp3pp1/rr6/3p2Np/2pPnP2/q1P1P2P/P1R3P1/K1QR4 b - - 0 1}
\begin{center}
\showinverseboard 
\end{center}
\clearpage 
\newpage 
\mainline{1... Nxc3 $1 2. Qxa3 Rxa3 $19 { * (0-1, 37 moves) }}\fenboard{1k6/pp3pp1/rr6/3p2Np/2pPnP2/q1P1P2P/P1R3P1/K1QR4 b - - 0 1}
\begin{center}
\showinverseboard 
\end{center}
\fenboard{1k6/pp3pp1/rr6/3p2Np/2pPnP2/q1P1P2P/P1R3P1/K1QR4 b - - 0 1}
\mainline[level=1]{1... Nxc3 $1 2. Qxa3 }

\variation[level=2]{ 2. Rxc3 Qxa2# \xskakcomment{ mate * }} 

\mainline[level=1]{ 2... Rxa3 $19 }
 * (\blackWins, 37 moves) 




\clearpage 
\newpage 

\section{93.??.??}
\fenboard{r3r1k1/ppp2ppp/2nb1q2/6Rn/2BP4/P1NQBP1P/1P3P2/2KR4 b - - 0 1}
\begin{center}
\showinverseboard 
\end{center}
\clearpage 
\newpage 
\mainline{1... Rxe3 $1 2. Qxe3 Bf4 $19 { * There is both a fork and a pin. }}\fenboard{r3r1k1/ppp2ppp/2nb1q2/6Rn/2BP4/P1NQBP1P/1P3P2/2KR4 b - - 0 1}
\begin{center}
\showinverseboard 
\end{center}
\fenboard{r3r1k1/ppp2ppp/2nb1q2/6Rn/2BP4/P1NQBP1P/1P3P2/2KR4 b - - 0 1}
\mainline[level=1]{1... Rxe3 $1 2. Qxe3 }

\variation[level=2]{ \xskakcomment{ White played }} \variation[level=2]{ 2. Rxg7+ $18 \xskakcomment{ and resigned after 45 moves. }} 

\mainline[level=1]{ 2... Bf4 $19 }
 * There is both a fork and a pin. 




\clearpage 
\newpage 

\section{94.??.??}
\fenboard{7k/p6p/6p1/2b2b2/2P5/2R1pBP1/P2rR1KP/8 b - - 0 1}
\begin{center}
\showinverseboard 
\end{center}
\clearpage 
\newpage 
\mainline{1... Be4 { White resigned. } 2. Bxe4 Rxe2+ $18 { * The bishop will have to sacrifice itself for the e-pawn. }}\fenboard{7k/p6p/6p1/2b2b2/2P5/2R1pBP1/P2rR1KP/8 b - - 0 1}
\begin{center}
\showinverseboard 
\end{center}
\fenboard{7k/p6p/6p1/2b2b2/2P5/2R1pBP1/P2rR1KP/8 b - - 0 1}
\mainline[level=1]{1... Be4 }
 White resigned. 

\mainline[level=1]{ 2. Bxe4 }

\variation[level=2]{\xskakcomment{\noindent\textbf{a)} } 2. Rxd2 exd2 $19 \xskakcomment{ * and the pawn queens }} 

\variation[level=2]{\xskakcomment{\noindent\textbf{b)} } \xskakcomment{ or }} \variation[level=2]{ 2. Rxe3 Bxe3 3. Rxd2 Bxf3+ 4. Kxf3 Bxd2 \xskakcomment{ * and Black is winning. }} 

\mainline[level=1]{ 2... Rxe2+ $18 }
 * The bishop will have to sacrifice itself for the e-pawn. 




\clearpage 
\newpage 

\section{95.??.??}
\fenboard{3q3k/1pp3pp/5p2/1P6/4PQ2/3B2P1/1r3b1P/R6K w - - 0 1}
\begin{center}
\showboard 
\end{center}
\clearpage 
\newpage 
\mainline{1. Qxc7 $1 $18 { * White wins two pawns due to the back-rank mate (1-0, 47 moves). }}\fenboard{3q3k/1pp3pp/5p2/1P6/4PQ2/3B2P1/1r3b1P/R6K w - - 0 1}
\begin{center}
\showboard 
\end{center}
\fenboard{3q3k/1pp3pp/5p2/1P6/4PQ2/3B2P1/1r3b1P/R6K w - - 0 1}
\mainline[level=1]{1. Qxc7 $1 $18 }
 * White wins two pawns due to the back-rank mate (\whiteWins, 47 moves). 

\variation[level=2]{ 1. e5 \xskakcomment{ also looks promising, as it opens up for an attack against h7, but }} \variation[level=2]{ 1... Rd2 $1 2. Qf5 g6 3. Qxf6+ Qxf6 4. exf6 $18 \xskakcomment{ limits White's advantage to only a winning endgame. }} 




\clearpage 
\newpage 

\section{96.??.??}
\fenboard{r4rk1/p3qpp1/2p2n2/3pb2p/3Q2B1/1PN1P3/PB3PPP/2R2RK1 w - - 0 1}
\begin{center}
\showboard 
\end{center}
\clearpage 
\newpage 
\mainline{1. Nxd5 $1 { Threatening the queen with check. } 1... cxd5 2. Qxe5 Qxe5 3. Bxe5 $18 { * White has won a pawn with a dominant position (1-0, 24 moves). }}\fenboard{r4rk1/p3qpp1/2p2n2/3pb2p/3Q2B1/1PN1P3/PB3PPP/2R2RK1 w - - 0 1}
\begin{center}
\showboard 
\end{center}
\fenboard{r4rk1/p3qpp1/2p2n2/3pb2p/3Q2B1/1PN1P3/PB3PPP/2R2RK1 w - - 0 1}
\mainline[level=1]{1. Nxd5 $1 }
 Threatening the queen with check. 

\mainline[level=1]{ 1... cxd5 2. Qxe5 Qxe5 3. Bxe5 $18 }
 * White has won a pawn with a dominant position (\whiteWins, 24 moves). 




\clearpage 
\newpage 

\section{97.??.??}
\fenboard{r4rk1/pp1b3p/2p4q/3p1p2/3P2n1/2NBP1PR/PP3PK1/R2Q4 b - - 0 1}
\begin{center}
\showinverseboard 
\end{center}
\clearpage 
\newpage 
\mainline{1... Qxh3+ $1 2. Kxh3 Nxf2+ 3. Kg2 Nxd1 4. Rxd1 $19 { * It's an exchange and a pawn. }}\fenboard{r4rk1/pp1b3p/2p4q/3p1p2/3P2n1/2NBP1PR/PP3PK1/R2Q4 b - - 0 1}
\begin{center}
\showinverseboard 
\end{center}
\fenboard{r4rk1/pp1b3p/2p4q/3p1p2/3P2n1/2NBP1PR/PP3PK1/R2Q4 b - - 0 1}
\mainline[level=1]{1... Qxh3+ $1 2. Kxh3 Nxf2+ 3. Kg2 Nxd1 4. Rxd1 $19 }
 * It's an exchange and a pawn. 




\clearpage 
\newpage 

\section{98.??.??}
\fenboard{r3kr2/1p1b1pp1/p1n1p2p/8/4q3/1N6/PPP1BKPP/R2Q1R2 w q - 0 1}
\begin{center}
\showboard 
\end{center}
\clearpage 
\newpage 
\mainline{1. Qxd7+ $1 Kxd7 2. Nc5+ Ke7 3. Nxe4 $18 { * }}\fenboard{r3kr2/1p1b1pp1/p1n1p2p/8/4q3/1N6/PPP1BKPP/R2Q1R2 w q - 0 1}
\begin{center}
\showboard 
\end{center}
\fenboard{r3kr2/1p1b1pp1/p1n1p2p/8/4q3/1N6/PPP1BKPP/R2Q1R2 w q - 0 1}
\mainline[level=1]{1. Qxd7+ $1 Kxd7 2. Nc5+ Ke7 3. Nxe4 $18 }
 * 




\clearpage 
\newpage 

\section{99.??.??}
\fenboard{4r1k1/5p1p/2Qb2p1/3P4/6Pn/2N1B2P/1P3P1q/3R1K2 b - - 0 1}
\begin{center}
\showinverseboard 
\end{center}
\clearpage 
\newpage 
\mainline{1... Rxe3 $1 2. fxe3 Qg2+ 3. Ke1 Nf3# { mate * }}\fenboard{4r1k1/5p1p/2Qb2p1/3P4/6Pn/2N1B2P/1P3P1q/3R1K2 b - - 0 1}
\begin{center}
\showinverseboard 
\end{center}
\fenboard{4r1k1/5p1p/2Qb2p1/3P4/6Pn/2N1B2P/1P3P1q/3R1K2 b - - 0 1}
\mainline[level=1]{1... Rxe3 $1 2. fxe3 Qg2+ 3. Ke1 Nf3# }
 mate * 




\clearpage 
\newpage 

\section{100.??.??}
\fenboard{r1b1k2r/1p3pp1/p3pn2/2p1q1N1/8/1B2P3/PPP1Q2P/2KR3R w kq - 0 1}
\begin{center}
\showboard 
\end{center}
\clearpage 
\newpage 
\mainline{1. Rd8+ $1 Ke7 2. Rxh8 $18 { * The game finished with: } 2... Qxg5 3. Qd2}\fenboard{r1b1k2r/1p3pp1/p3pn2/2p1q1N1/8/1B2P3/PPP1Q2P/2KR3R w kq - 0 1}
\begin{center}
\showboard 
\end{center}
\fenboard{r1b1k2r/1p3pp1/p3pn2/2p1q1N1/8/1B2P3/PPP1Q2P/2KR3R w kq - 0 1}
\mainline[level=1]{1. Rd8+ $1 Ke7 }

\variation[level=2]{ \xskakcomment{ The point is }} \variation[level=2]{ 1... Kxd8 2. Nxf7+ Ke7 3. Nxe5 $18 \xskakcomment{ * }} 

\mainline[level=1]{ 2. Rxh8 $18 }
 * The game finished with: 




\clearpage 
\newpage 

\section{101.??.??}
\fenboard{6k1/1pqrnp2/3p2p1/2pn2b1/P1Q3Pp/2B4P/1PP1RP2/4R1K1 w - - 0 1}
\begin{center}
\showboard 
\end{center}
\clearpage 
\newpage 
\mainline{1. Qxd5 $1 Nxd5 2. Re8+ Kh7 3. Rh8# { mate * }}\fenboard{6k1/1pqrnp2/3p2p1/2pn2b1/P1Q3Pp/2B4P/1PP1RP2/4R1K1 w - - 0 1}
\begin{center}
\showboard 
\end{center}
\fenboard{6k1/1pqrnp2/3p2p1/2pn2b1/P1Q3Pp/2B4P/1PP1RP2/4R1K1 w - - 0 1}
\mainline[level=1]{1. Qxd5 $1 Nxd5 }

\variation[level=2]{ \xskakcomment{ The game move }} \variation[level=2]{ 1... Bf4 \xskakcomment{ is of course hopeless as well (\whiteWins, 41 moves). }} 

\mainline[level=1]{ 2. Re8+ Kh7 3. Rh8# }
 mate * 




\clearpage 
\newpage 

\section{102.??.??}
\fenboard{6k1/1p3pp1/p7/P2n1PQ1/8/8/1r3r1P/3R3K b - - 0 1}
\begin{center}
\showinverseboard 
\end{center}
\clearpage 
\newpage 
\mainline{1... Rxh2+ $1 2. Kg1 Rbg2+ $1 3. Qxg2 Rxg2+ 4. Kxg2 Ne3+ 5. Kf3 Nxd1 $19 { * }}\fenboard{6k1/1p3pp1/p7/P2n1PQ1/8/8/1r3r1P/3R3K b - - 0 1}
\begin{center}
\showinverseboard 
\end{center}
\fenboard{6k1/1p3pp1/p7/P2n1PQ1/8/8/1r3r1P/3R3K b - - 0 1}
\mainline[level=1]{1... Rxh2+ $1 2. Kg1 Rbg2+ $1 }

\variation[level=2]{ \xskakcomment{ Or }} \variation[level=2]{ 2... Rhg2+ }

\mainline[level=1]{ 3. Qxg2 }

\variation[level=2]{ \xskakcomment{ The game went }} \variation[level=2]{ 3. Kf1 Rxg5 \xskakcomment{ \blackWins. }} 

\mainline[level=1]{ 3... Rxg2+ 4. Kxg2 Ne3+ 5. Kf3 Nxd1 $19 }
 * 




\clearpage 
\newpage 

\section{103.??.??}
\fenboard{1r4k1/2qpn1pp/p1p1pr2/2b5/2P2P2/3B4/PP4PP/R1BQ1R1K w - - 0 1}
\begin{center}
\showboard 
\end{center}
\clearpage 
\newpage 
\mainline{1. Bxh7+ $1 Kxh7 2. Qh5+ Rh6 3. Qxc5 $16 { * White has won a pawn, but it is only a flank pawn. }}\fenboard{1r4k1/2qpn1pp/p1p1pr2/2b5/2P2P2/3B4/PP4PP/R1BQ1R1K w - - 0 1}
\begin{center}
\showboard 
\end{center}
\fenboard{1r4k1/2qpn1pp/p1p1pr2/2b5/2P2P2/3B4/PP4PP/R1BQ1R1K w - - 0 1}
\mainline[level=1]{1. Bxh7+ $1 }

\variation[level=2]{ 1. Qh5 Nf5 2. g4 \xskakcomment{ does not win a piece, and even loses after }} \variation[level=2]{ 2... Rh6 }

\mainline[level=1]{ 1... Kxh7 2. Qh5+ Rh6 3. Qxc5 $16 }
 * White has won a pawn, but it is only a flank pawn. 




\clearpage 
\newpage 

\section{104.??.??}
\fenboard{1rq1r2k/5Rbp/p2p1p1B/2p1p3/2P1P2Q/1P6/P5PP/3b2K1 w - - 0 1}
\begin{center}
\showboard 
\end{center}
\clearpage 
\newpage 
\mainline{1. Bxg7+ $1 Kg8 2. Bh8 $1 Kxf7 3. Qxf6+ Kg8 4. Qg7# { mate * }}\fenboard{1rq1r2k/5Rbp/p2p1p1B/2p1p3/2P1P2Q/1P6/P5PP/3b2K1 w - - 0 1}
\begin{center}
\showboard 
\end{center}
\fenboard{1rq1r2k/5Rbp/p2p1p1B/2p1p3/2P1P2Q/1P6/P5PP/3b2K1 w - - 0 1}
\mainline[level=1]{1. Bxg7+ $1 Kg8 2. Bh8 $1 Kxf7 3. Qxf6+ Kg8 4. Qg7# }
 mate * 




\clearpage 
\newpage 

\section{105.??.??}
\fenboard{5bk1/p2Q1p2/q4p2/4r2p/3pr3/8/PPRRNPPP/5K2 b - - 0 1}
\begin{center}
\showinverseboard 
\end{center}
\clearpage 
\newpage 
\mainline{1... Rxe2 $1 2. Rxe2 d3 3. Rxe5 dxc2+ $18 { * Black gets a second queen. }}\fenboard{5bk1/p2Q1p2/q4p2/4r2p/3pr3/8/PPRRNPPP/5K2 b - - 0 1}
\begin{center}
\showinverseboard 
\end{center}
\fenboard{5bk1/p2Q1p2/q4p2/4r2p/3pr3/8/PPRRNPPP/5K2 b - - 0 1}
\mainline[level=1]{1... Rxe2 $1 2. Rxe2 d3 3. Rxe5 }

\variation[level=2]{ \xskakcomment{ White resigned after }} \variation[level=2]{ 3. Rc3 dxe2+ 4. Ke1 Bb4 }

\mainline[level=1]{ 3... dxc2+ $18 }
 * Black gets a second queen. 




\clearpage 
\newpage 

\section{106.??.??}
\fenboard{6k1/q4pp1/4p2p/1p1r4/1PpPQ3/r1P1R1P1/4RP1P/6K1 b - - 0 1}
\begin{center}
\showinverseboard 
\end{center}
\clearpage 
\newpage 
\mainline{1... Rxc3 $1 2. Rxc3 Qa1+ 3. Kg2 Qxc3 $19 { * (0-1, 38 moves) }}\fenboard{6k1/q4pp1/4p2p/1p1r4/1PpPQ3/r1P1R1P1/4RP1P/6K1 b - - 0 1}
\begin{center}
\showinverseboard 
\end{center}
\fenboard{6k1/q4pp1/4p2p/1p1r4/1PpPQ3/r1P1R1P1/4RP1P/6K1 b - - 0 1}
\mainline[level=1]{1... Rxc3 $1 2. Rxc3 Qa1+ 3. Kg2 Qxc3 $19 }
 * (\blackWins, 38 moves) 




\clearpage 
\newpage 

\section{107.??.??}
\fenboard{r2qk2r/pp1n1ppb/2pbpn1p/4N3/2BP1P1P/6N1/PPP1Q1P1/R1B1K2R w KQkq - 0 1}
\begin{center}
\showboard 
\end{center}
\clearpage 
\newpage 
\mainline{1. Nxf7 $1 Kxf7 2. Qxe6+ Kf8 3. Qf7# { mate * }}\fenboard{r2qk2r/pp1n1ppb/2pbpn1p/4N3/2BP1P1P/6N1/PPP1Q1P1/R1B1K2R w KQkq - 0 1}
\begin{center}
\showboard 
\end{center}
\fenboard{r2qk2r/pp1n1ppb/2pbpn1p/4N3/2BP1P1P/6N1/PPP1Q1P1/R1B1K2R w KQkq - 0 1}
\mainline[level=1]{1. Nxf7 $1 Kxf7 }

\variation[level=2]{ 1... Qe7 \xskakcomment{ and White won after 30 moves. }} 

\mainline[level=1]{ 2. Qxe6+ Kf8 3. Qf7# }
 mate * 




\clearpage 
\newpage 

\section{108.??.??}
\fenboard{1r4k1/5n1p/5qp1/1p6/3Q4/1P4PP/P3rPB1/R2R2K1 b - - 0 1}
\begin{center}
\showinverseboard 
\end{center}
\clearpage 
\newpage 
\mainline{1... Re1+ $1 { 0-1 White resigned due to: } 2. Kh2 Qxd4 3. Rxd4 Rxa1 $19 { * }}\fenboard{1r4k1/5n1p/5qp1/1p6/3Q4/1P4PP/P3rPB1/R2R2K1 b - - 0 1}
\begin{center}
\showinverseboard 
\end{center}
\fenboard{1r4k1/5n1p/5qp1/1p6/3Q4/1P4PP/P3rPB1/R2R2K1 b - - 0 1}
\mainline[level=1]{1... Re1+ $1 }
 \blackWins White resigned due to: 

\mainline[level=1]{ 2. Kh2 }

\variation[level=2]{ 2. Rxe1 Qxd4 $19 \xskakcomment{ * }} 

\mainline[level=1]{ 2... Qxd4 3. Rxd4 Rxa1 $19 }
 * 




\clearpage 
\newpage 

\section{109.??.??}
\fenboard{r3r1k1/1p5p/p1pqn1p1/3p1p2/PP1P1P2/1Q1RP3/4B1PP/1R4K1 b - - 0 1}
\begin{center}
\showinverseboard 
\end{center}
\clearpage 
\newpage 
\mainline{1... Nxf4 $1 2. exf4 Rxe2 { * Black is clearly better, but White managed to hold (41 moves). }}\fenboard{r3r1k1/1p5p/p1pqn1p1/3p1p2/PP1P1P2/1Q1RP3/4B1PP/1R4K1 b - - 0 1}
\begin{center}
\showinverseboard 
\end{center}
\fenboard{r3r1k1/1p5p/p1pqn1p1/3p1p2/PP1P1P2/1Q1RP3/4B1PP/1R4K1 b - - 0 1}
\mainline[level=1]{1... Nxf4 $1 2. exf4 Rxe2 }
 * Black is clearly better, but White managed to hold (41 moves). 




\clearpage 
\newpage 

\section{110.??.??}
\fenboard{3r2k1/p2r1p2/4b1p1/qPp1R2p/P1p4P/8/5PP1/Q3RBK1 w - - 0 1}
\begin{center}
\showboard 
\end{center}
\clearpage 
\newpage 
\mainline{1. Bxc4 $1 $18 { Black cannot take the bishop due to: } 1... Bxc4 2. Re8+ Rxe8 3. Rxe8+ Kh7 4. Rh8# { mate * }}\fenboard{3r2k1/p2r1p2/4b1p1/qPp1R2p/P1p4P/8/5PP1/Q3RBK1 w - - 0 1}
\begin{center}
\showboard 
\end{center}
\fenboard{3r2k1/p2r1p2/4b1p1/qPp1R2p/P1p4P/8/5PP1/Q3RBK1 w - - 0 1}
\mainline[level=1]{1. Bxc4 $1 $18 }
 Black cannot take the bishop due to: 

\variation[level=2]{ 1. Rxe6 fxe6 2. Bxc4 $2 } (\variation[level=3]{ 2. Rxe6 Kh7 })
\variation[level=2]{ 2... Qxe1+ $18 }

\mainline[level=1]{ 1... Bxc4 }

\variation[level=2]{ 1... Rd6 2. Bxe6 \xskakcomment{ was just hopeless (\whiteWins, 41 moves). }} 

\mainline[level=1]{ 2. Re8+ Rxe8 3. Rxe8+ Kh7 4. Rh8# }
 mate * 




\clearpage 
\newpage 

\section{111.??.??}
\fenboard{3r2k1/5qpp/pp2r3/2p2b2/nPPp1PP1/P4Q2/3N3P/R3RBK1 b - - 0 1}
\begin{center}
\showinverseboard 
\end{center}
\clearpage 
\newpage 
\mainline{1... Bxg4 $1 2. Qxg4 Rg6 $19 { * }}\fenboard{3r2k1/5qpp/pp2r3/2p2b2/nPPp1PP1/P4Q2/3N3P/R3RBK1 b - - 0 1}
\begin{center}
\showinverseboard 
\end{center}
\fenboard{3r2k1/5qpp/pp2r3/2p2b2/nPPp1PP1/P4Q2/3N3P/R3RBK1 b - - 0 1}
\mainline[level=1]{1... Bxg4 $1 2. Qxg4 }

\variation[level=2]{ \xskakcomment{ White fought on with }} \variation[level=2]{ 2. Qg3 $19 \xskakcomment{ but he regretted his previous move g4? for sure (\blackWins, 38 moves). }} 

\mainline[level=1]{ 2... Rg6 $19 }
 * 




\clearpage 
\newpage 

\section{112.??.??}
\fenboard{6k1/1q1rbpp1/7p/1p1p1P2/1P2p1P1/P1Q5/4B2P/3R2K1 b - - 0 1}
\begin{center}
\showinverseboard 
\end{center}
\clearpage 
\newpage 
\mainline{1... d4 $1 2. Rxd4 Qb6 { * 0-1  If you
chose a slower way to advance the pawn on the 1st move, such as Bf6 or Qa7+,
you also get full points. }}\fenboard{6k1/1q1rbpp1/7p/1p1p1P2/1P2p1P1/P1Q5/4B2P/3R2K1 b - - 0 1}
\begin{center}
\showinverseboard 
\end{center}
\fenboard{6k1/1q1rbpp1/7p/1p1p1P2/1P2p1P1/P1Q5/4B2P/3R2K1 b - - 0 1}
\mainline[level=1]{1... d4 $1 2. Rxd4 }

\variation[level=2]{ 2. Qe1 d3 $19 \xskakcomment{ and the pawn supported by pieces is too strong. }} 

\mainline[level=1]{ 2... Qb6 }
 * \blackWins If you
chose a slower way to advance the pawn on the 1st move, such as Bf6 or Qa7+,
you also get full points. 

\variation[level=2]{\xskakcomment{\noindent\textbf{a)} } \xskakcomment{ Or }} \variation[level=2]{ 2... Qa7 $19 }

\variation[level=2]{\xskakcomment{\noindent\textbf{b)} } \xskakcomment{ but not }} \variation[level=2]{ 2... Bf6 $2 3. Rxd7 $1 $17 }




\clearpage 
\newpage 

\section{113.??.??}
\fenboard{2r1b1k1/r1N2p1p/1p2p1pn/p2pP3/1b1P2P1/1P3N1P/P1R2P2/2R2BK1 w - - 0 1}
\begin{center}
\showboard 
\end{center}
\clearpage 
\newpage 
\mainline{1. Nxd5 $1 Rxc2 2. Nf6+ { Getting out of Dodge before recapturing the rook. } 2... Kf8 3. Rxc2 $18 { * In additi
on to an extra pawn, White's pieces are much stronger, for instance the bishop
on e8 is dominated (1-0, 33 moves). }}\fenboard{2r1b1k1/r1N2p1p/1p2p1pn/p2pP3/1b1P2P1/1P3N1P/P1R2P2/2R2BK1 w - - 0 1}
\begin{center}
\showboard 
\end{center}
\fenboard{2r1b1k1/r1N2p1p/1p2p1pn/p2pP3/1b1P2P1/1P3N1P/P1R2P2/2R2BK1 w - - 0 1}
\mainline[level=1]{1. Nxd5 $1 }

\variation[level=2]{ 1. Ba6 $6 Rd8 $14 }

\mainline[level=1]{ 1... Rxc2 2. Nf6+ }
 Getting out of Dodge before recapturing the rook. 

\mainline[level=1]{ 2... Kf8 3. Rxc2 $18 }
 * In additi
on to an extra pawn, White's pieces are much stronger, for instance the bishop
on e8 is dominated (\whiteWins, 33 moves). 




\clearpage 
\newpage 

\section{114.??.??}
\fenboard{r4rk1/2n1q2p/b1n1p1p1/pp1pPpN1/P1pP1N1P/2P3P1/1P3PB1/R2QR1K1 w - - 0 1}
\begin{center}
\showboard 
\end{center}
\clearpage 
\newpage 
\mainline{1. Nfxe6 $1 $18 Nxe6 2. Bxd5 $18 { * Since the
c6-knight cannot retreat on account of the a8-rook, White is regaining at
least the piece, with a winning position (1-0, 30 moves). }}\fenboard{r4rk1/2n1q2p/b1n1p1p1/pp1pPpN1/P1pP1N1P/2P3P1/1P3PB1/R2QR1K1 w - - 0 1}
\begin{center}
\showboard 
\end{center}
\fenboard{r4rk1/2n1q2p/b1n1p1p1/pp1pPpN1/P1pP1N1P/2P3P1/1P3PB1/R2QR1K1 w - - 0 1}
\mainline[level=1]{1. Nfxe6 $1 $18 }

\variation[level=2]{\xskakcomment{\noindent\textbf{a)} } \xskakcomment{ There are a few alternatives with the same idea: }} \variation[level=2]{ 1. Ngxe6 $18 }

\variation[level=2]{\xskakcomment{\noindent\textbf{b)} } \xskakcomment{ or }} \variation[level=2]{ 1. axb5 Bxb5 2. Ngxe6 $18 }

\mainline[level=1]{ 1... Nxe6 2. Bxd5 $18 }
 * Since the
c6-knight cannot retreat on account of the a8-rook, White is regaining at
least the piece, with a winning position (\whiteWins, 30 moves). 




\clearpage 
\newpage 

\section{115.??.??}
\fenboard{3r1n1k/3P3p/pp3q2/2pQp3/P1P3B1/3b2R1/1P5P/6K1 w - - 0 1}
\begin{center}
\showboard 
\end{center}
\clearpage 
\newpage 
\mainline{1. Qg8+ $1 { A magnet sacrifice leading to a quick mate. Black resigned, rather than face: } 1... Kxg8 2. Be6+ Kh8 3. Rg8# { mate * }}\fenboard{3r1n1k/3P3p/pp3q2/2pQp3/P1P3B1/3b2R1/1P5P/6K1 w - - 0 1}
\begin{center}
\showboard 
\end{center}
\fenboard{3r1n1k/3P3p/pp3q2/2pQp3/P1P3B1/3b2R1/1P5P/6K1 w - - 0 1}
\mainline[level=1]{1. Qg8+ $1 }
 A magnet sacrifice leading to a quick mate. Black resigned, rather than face: 

\mainline[level=1]{ 1... Kxg8 2. Be6+ Kh8 3. Rg8# }
 mate * 




\clearpage 
\newpage 

\section{116.??.??}
\fenboard{6k1/5r1p/p2N4/nppP2q1/2P5/1P2N3/PQ5P/7K w - - 0 1}
\begin{center}
\showboard 
\end{center}
\clearpage 
\newpage 
\mainline{1. Qh8+ $1 { In the game, Black resigned. He is a piece down after: } 1... Kxh8 2. Nxf7+ Kg7 3. Nxg5 $18 { * }}\fenboard{6k1/5r1p/p2N4/nppP2q1/2P5/1P2N3/PQ5P/7K w - - 0 1}
\begin{center}
\showboard 
\end{center}
\fenboard{6k1/5r1p/p2N4/nppP2q1/2P5/1P2N3/PQ5P/7K w - - 0 1}
\mainline[level=1]{1. Qh8+ $1 }
 In the game, Black resigned. He is a piece down after: 

\variation[level=2]{ 1. Nxf7 $6 Qxe3 $1 $14 }

\mainline[level=1]{ 1... Kxh8 2. Nxf7+ Kg7 3. Nxg5 $18 }
 * 




\clearpage 
\newpage 

\section{117.??.??}
\fenboard{Q7/2r2rpk/2p4p/7N/3PpN2/1p2P3/1K4R1/5q2 w - - 0 1}
\begin{center}
\showboard 
\end{center}
\clearpage 
\newpage 
\mainline{1. Rxg7+ $1 Rxg7 2. Nf6# { mate * }}\fenboard{Q7/2r2rpk/2p4p/7N/3PpN2/1p2P3/1K4R1/5q2 w - - 0 1}
\begin{center}
\showboard 
\end{center}
\fenboard{Q7/2r2rpk/2p4p/7N/3PpN2/1p2P3/1K4R1/5q2 w - - 0 1}
\mainline[level=1]{1. Rxg7+ $1 }

\variation[level=2]{\xskakcomment{\noindent\textbf{a)} } 1. Qe8 \xskakcomment{ eyes the pawn on e4, but wins only because White has the
same rook sacrifice later. }} 

\variation[level=2]{\xskakcomment{\noindent\textbf{b)} } \xskakcomment{ But not: }} \variation[level=2]{ 1. Nf6+ $4 Rxf6 $19 }

\mainline[level=1]{ 1... Rxg7 2. Nf6# }
 mate * 




\clearpage 
\newpage 

\section{118.??.??}
\fenboard{6rk/p3p2p/1p2Pp2/2p2P2/2P1nBr1/1P6/P6P/3R1R1K b - - 0 1}
\begin{center}
\showinverseboard 
\end{center}
\clearpage 
\newpage 
\mainline{1... Rg1+ $1 { White resigned due to: } 2. Rxg1 Nf2# { mate * }}\fenboard{6rk/p3p2p/1p2Pp2/2p2P2/2P1nBr1/1P6/P6P/3R1R1K b - - 0 1}
\begin{center}
\showinverseboard 
\end{center}
\fenboard{6rk/p3p2p/1p2Pp2/2p2P2/2P1nBr1/1P6/P6P/3R1R1K b - - 0 1}
\mainline[level=1]{1... Rg1+ $1 }
 White resigned due to: 

\mainline[level=1]{ 2. Rxg1 Nf2# }
 mate * 




\clearpage 
\newpage 

\section{119.??.??}
\fenboard{8/pr1r3p/6p1/2p1pk2/3b2N1/1P4P1/P2R1PKP/4R3 w - - 0 1}
\begin{center}
\showboard 
\end{center}
\clearpage 
\newpage 
\mainline{1. Rxd4 $1 { Black resigned as it's mate: } 1... Rxd4 2. Rxe5+ Kxg4 3. f3# { mate * }}\fenboard{8/pr1r3p/6p1/2p1pk2/3b2N1/1P4P1/P2R1PKP/4R3 w - - 0 1}
\begin{center}
\showboard 
\end{center}
\fenboard{8/pr1r3p/6p1/2p1pk2/3b2N1/1P4P1/P2R1PKP/4R3 w - - 0 1}
\mainline[level=1]{1. Rxd4 $1 }
 Black resigned as it's mate: 

\mainline[level=1]{ 1... Rxd4 2. Rxe5+ Kxg4 3. f3# }
 mate * 

\variation[level=2]{ \xskakcomment{ Or }} \variation[level=2]{ 3. h3# \xskakcomment{ mate. }} 




\clearpage 
\newpage 

\section{120.??.??}
\fenboard{r5k1/pp1b2pp/4q3/n7/3Rp3/P7/5QPP/2B2RK1 w - - 0 1}
\begin{center}
\showboard 
\end{center}
\clearpage 
\newpage 
\mainline{1. Rxe4 $1 Qxe4 2. Qf7+ Kh8 3. Qf8+ Rxf8 4. Rxf8# { mate * }}\fenboard{r5k1/pp1b2pp/4q3/n7/3Rp3/P7/5QPP/2B2RK1 w - - 0 1}
\begin{center}
\showboard 
\end{center}
\fenboard{r5k1/pp1b2pp/4q3/n7/3Rp3/P7/5QPP/2B2RK1 w - - 0 1}
\mainline[level=1]{1. Rxe4 $1 Qxe4 }

\variation[level=2]{ 1... Qg6 $18 \xskakcomment{ is hopeless for Black when he has lost his only
trump, the passed e-pawn (\whiteWins, 28 moves). }} 

\mainline[level=1]{ 2. Qf7+ Kh8 3. Qf8+ Rxf8 4. Rxf8# }
 mate * 




\clearpage 
\newpage 

\section{121.??.??}
\fenboard{r1bq1rk1/pp2ppbp/2n3p1/2pn4/3p1P2/NP2PN2/PBPPB1PP/3RQRK1 b - - 0 1}
\begin{center}
\showinverseboard 
\end{center}
\clearpage 
\newpage 
\mainline{1... d3 $1 { The bishop on b2 is en prise and } 2. Bxg7 dxe2 $19 { * is an intermediate move that wins a piece. }}\fenboard{r1bq1rk1/pp2ppbp/2n3p1/2pn4/3p1P2/NP2PN2/PBPPB1PP/3RQRK1 b - - 0 1}
\begin{center}
\showinverseboard 
\end{center}
\fenboard{r1bq1rk1/pp2ppbp/2n3p1/2pn4/3p1P2/NP2PN2/PBPPB1PP/3RQRK1 b - - 0 1}
\mainline[level=1]{1... d3 $1 }
 The bishop on b2 is en prise and 

\mainline[level=1]{ 2. Bxg7 dxe2 $19 }
 * is an intermediate move that wins a piece. 




\clearpage 
\newpage 

\section{122.??.??}
\fenboard{8/4r1k1/4qp2/1p6/3R4/P1N2QPp/1PP2K1P/4r3 b - - 0 1}
\begin{center}
\showinverseboard 
\end{center}
\clearpage 
\newpage 
\mainline{1... Rf1+ $1 { Other moves are obviously also winning, but mate-in-two should
be seen and played here. } 2. Kxf1 Qe1# { mate * }}\fenboard{8/4r1k1/4qp2/1p6/3R4/P1N2QPp/1PP2K1P/4r3 b - - 0 1}
\begin{center}
\showinverseboard 
\end{center}
\fenboard{8/4r1k1/4qp2/1p6/3R4/P1N2QPp/1PP2K1P/4r3 b - - 0 1}
\mainline[level=1]{1... Rf1+ $1 }
 Other moves are obviously also winning, but mate-in-two should
be seen and played here. 

\mainline[level=1]{ 2. Kxf1 Qe1# }
 mate * 




\clearpage 
\newpage 

\section{123.??.??}
\fenboard{2kr3r/pbpq4/1p5p/4PpP1/P2p4/2P2Q2/2P2P2/R1B1KB1R w KQ - 0 1}
\begin{center}
\showboard 
\end{center}
\clearpage 
\newpage 
\mainline{1. Ba6 $1 Bxa6 2. Qa8# { mate * }}\fenboard{2kr3r/pbpq4/1p5p/4PpP1/P2p4/2P2Q2/2P2P2/R1B1KB1R w KQ - 0 1}
\begin{center}
\showboard 
\end{center}
\fenboard{2kr3r/pbpq4/1p5p/4PpP1/P2p4/2P2Q2/2P2P2/R1B1KB1R w KQ - 0 1}
\mainline[level=1]{1. Ba6 $1 Bxa6 }

\variation[level=2]{ \xskakcomment{ Black played }} \variation[level=2]{ 1... c6 \xskakcomment{ but was simply a piece down after }} \variation[level=2]{ 2. Bxb7+ }

\mainline[level=1]{ 2. Qa8# }
 mate * 




\clearpage 
\newpage 

\section{124.??.??}
\fenboard{8/4r1p1/p2k1p1p/1bNp3P/3P1K2/2P2P2/6P1/1R6 w - - 0 1}
\begin{center}
\showboard 
\end{center}
\clearpage 
\newpage 
\mainline{1. Nxa6 $1 { Neatly picking up a pawn due to: } 1... Bxa6 2. Rb6+ Kd7 3. Rxa6 $18 { * }}\fenboard{8/4r1p1/p2k1p1p/1bNp3P/3P1K2/2P2P2/6P1/1R6 w - - 0 1}
\begin{center}
\showboard 
\end{center}
\fenboard{8/4r1p1/p2k1p1p/1bNp3P/3P1K2/2P2P2/6P1/1R6 w - - 0 1}
\mainline[level=1]{1. Nxa6 $1 }
 Neatly picking up a pawn due to: 

\mainline[level=1]{ 1... Bxa6 2. Rb6+ Kd7 3. Rxa6 $18 }
 * 




\clearpage 
\newpage 

\section{125.??.??}
\fenboard{4kb1r/1br4p/p3p1p1/1p3R2/1n2p3/PNN1B3/1PP3PP/2R3K1 w k - 0 1}
\begin{center}
\showboard 
\end{center}
\clearpage 
\newpage 
\mainline{1. Rxf8+ $1 Rxf8 2. axb4 $18 { * The two pieces easily outshine the rook. }}\fenboard{4kb1r/1br4p/p3p1p1/1p3R2/1n2p3/PNN1B3/1PP3PP/2R3K1 w k - 0 1}
\begin{center}
\showboard 
\end{center}
\fenboard{4kb1r/1br4p/p3p1p1/1p3R2/1n2p3/PNN1B3/1PP3PP/2R3K1 w k - 0 1}
\mainline[level=1]{1. Rxf8+ $1 Rxf8 2. axb4 $18 }
 * The two pieces easily outshine the rook. 




\clearpage 
\newpage 

\section{126.??.??}
\fenboard{8/4k1p1/1p2pp1p/p6P/2n1PN2/4qPP1/1P4K1/3Q4 b - - 0 1}
\begin{center}
\showinverseboard 
\end{center}
\clearpage 
\newpage 
\mainline{1... Qxf4 $1 { There is a fork on e3 coming up. } 2. gxf4 Ne3+ 3. Kg3 Nxd1 $19 { * }}\fenboard{8/4k1p1/1p2pp1p/p6P/2n1PN2/4qPP1/1P4K1/3Q4 b - - 0 1}
\begin{center}
\showinverseboard 
\end{center}
\fenboard{8/4k1p1/1p2pp1p/p6P/2n1PN2/4qPP1/1P4K1/3Q4 b - - 0 1}
\mainline[level=1]{1... Qxf4 $1 }
 There is a fork on e3 coming up. 

\mainline[level=1]{ 2. gxf4 Ne3+ 3. Kg3 Nxd1 $19 }
 * 




\clearpage 
\newpage 

\section{127.??.??}
\fenboard{8/5r1k/p3Npp1/4p3/3nP3/4QP2/PP2q1P1/1KR5 w - - 0 1}
\begin{center}
\showboard 
\end{center}
\clearpage 
\newpage 
\mainline{1. Qh6+ $1 { Black did not let his opponent execute the mate: } 1... Kxh6 2. Rh1# { mate * }}\fenboard{8/5r1k/p3Npp1/4p3/3nP3/4QP2/PP2q1P1/1KR5 w - - 0 1}
\begin{center}
\showboard 
\end{center}
\fenboard{8/5r1k/p3Npp1/4p3/3nP3/4QP2/PP2q1P1/1KR5 w - - 0 1}
\mainline[level=1]{1. Qh6+ $1 }
 Black did not let his opponent execute the mate: 

\mainline[level=1]{ 1... Kxh6 }

\variation[level=2]{ \xskakcomment{ Or }} \variation[level=2]{ 1... Kg8 2. Rc8+ Rf8 3. Rxf8# \xskakcomment{ mate. }} 

\mainline[level=1]{ 2. Rh1# }
 mate * 




\clearpage 
\newpage 

\section{128.??.??}
\fenboard{3Rb3/6kp/6p1/3P1p2/3q4/Q5PP/P1r1r1B1/6RK b - - 0 1}
\begin{center}
\showinverseboard 
\end{center}
\clearpage 
\newpage 
\mainline{1... Qxg1+ 2. Kxg1 Rxg2+ 3. Kh1 Rh2+ 4. Kg1 Rcg2+ 5. Kf1 Bb5+ 6. Ke1 Rh1# { mate * }}\fenboard{3Rb3/6kp/6p1/3P1p2/3q4/Q5PP/P1r1r1B1/6RK b - - 0 1}
\begin{center}
\showinverseboard 
\end{center}
\fenboard{3Rb3/6kp/6p1/3P1p2/3q4/Q5PP/P1r1r1B1/6RK b - - 0 1}

 Black is winning with many moves, but only one is a forced mate: 

\mainline[level=1]{ 1... Qxg1+ 2. Kxg1 Rxg2+ 3. Kh1 Rh2+ 4. Kg1 Rcg2+ 5. Kf1 Bb5+ 6. Ke1 Rh1# }
 mate * 

\variation[level=2]{ \xskakcomment{ Or }} \variation[level=2]{ 6... Rg1# \xskakcomment{ mate }} 




\clearpage 
\newpage 

\section{129.??.??}
\fenboard{2r2nk1/p2q1pp1/1p2b2p/3p4/3P1P2/4R1P1/PP1Q3P/1B2R1K1 w - - 0 1}
\begin{center}
\showboard 
\end{center}
\clearpage 
\newpage 
\mainline{1. f5 $1 Bxf5 2. Re7 $18 { * The queen can no longer defend the bishop. }}\fenboard{2r2nk1/p2q1pp1/1p2b2p/3p4/3P1P2/4R1P1/PP1Q3P/1B2R1K1 w - - 0 1}
\begin{center}
\showboard 
\end{center}
\fenboard{2r2nk1/p2q1pp1/1p2b2p/3p4/3P1P2/4R1P1/PP1Q3P/1B2R1K1 w - - 0 1}
\mainline[level=1]{1. f5 $1 Bxf5 2. Re7 $18 }
 * The queen can no longer defend the bishop. 




\clearpage 
\newpage 

\section{130.??.??}
\fenboard{b2r4/p4Brk/1pqp1Qp1/2p2pPp/5P2/3P4/PP5P/2B1RRK1 w - - 0 1}
\begin{center}
\showboard 
\end{center}
\clearpage 
\newpage 
\mainline{1. Bxg6+ $1 Rxg6 2. Re7+ Kg8 3. Qxg6+ Kf8 4. Qg7# { mate * }}\fenboard{b2r4/p4Brk/1pqp1Qp1/2p2pPp/5P2/3P4/PP5P/2B1RRK1 w - - 0 1}
\begin{center}
\showboard 
\end{center}
\fenboard{b2r4/p4Brk/1pqp1Qp1/2p2pPp/5P2/3P4/PP5P/2B1RRK1 w - - 0 1}

 It's mate in five moves: 

\mainline[level=1]{ 1. Bxg6+ $1 }

\variation[level=2]{ 1. Re2 \xskakcomment{ is winning as well thanks to the continued threat of Bxg6+ }} 

\mainline[level=1]{ 1... Rxg6 2. Re7+ Kg8 3. Qxg6+ }

\variation[level=2]{ \xskakcomment{ Or }} \variation[level=2]{ 3. Qf7+ Kh8 4. Qh7# \xskakcomment{ mate. }} 

\mainline[level=1]{ 3... Kf8 4. Qg7# }
 mate * 

\variation[level=2]{ \xskakcomment{ Or }} \variation[level=2]{ 4. Qf7# \xskakcomment{ mate. }} 




\clearpage 
\newpage 

\section{131.??.??}
\fenboard{2kn3r/pp1b4/2n1p1p1/q2pP1P1/2pN1P2/P1P3K1/2PB4/R2Q3B b - - 0 1}
\begin{center}
\showinverseboard 
\end{center}
\clearpage 
\newpage 
\mainline{1... Rxh1 $1 { Black gets two pieces for the rook after: } 2. Qxh1 Nxd4 3. cxd4 Qxd2 $19 { * }}\fenboard{2kn3r/pp1b4/2n1p1p1/q2pP1P1/2pN1P2/P1P3K1/2PB4/R2Q3B b - - 0 1}
\begin{center}
\showinverseboard 
\end{center}
\fenboard{2kn3r/pp1b4/2n1p1p1/q2pP1P1/2pN1P2/P1P3K1/2PB4/R2Q3B b - - 0 1}
\mainline[level=1]{1... Rxh1 $1 }
 Black gets two pieces for the rook after: 

\mainline[level=1]{ 2. Qxh1 Nxd4 3. cxd4 Qxd2 $19 }
 * 




\clearpage 
\newpage 

\section{132.??.??}
\fenboard{7k/8/7P/6K1/6Q1/6P1/7q/8 b - - 0 1}
\begin{center}
\showinverseboard 
\end{center}
\clearpage 
\newpage 
\mainline{1... Qxh6+ $1 2. Kxh6 { * 1/2-1/2 Stalemate! }}\fenboard{7k/8/7P/6K1/6Q1/6P1/7q/8 b - - 0 1}
\begin{center}
\showinverseboard 
\end{center}
\fenboard{7k/8/7P/6K1/6Q1/6P1/7q/8 b - - 0 1}
\mainline[level=1]{1... Qxh6+ $1 }

\variation[level=2]{\xskakcomment{\noindent\textbf{a)} } 1... Qd2+ \xskakcomment{ is the complicated way to draw (zero points). The
queen endgame with g- and h-pawns is generally drawn with the defending king
in front of the pawns. It surprised the whole Swedish team when we learned
this at the 2016 Olympiad in Baku. }} 

\variation[level=2]{\xskakcomment{\noindent\textbf{b)} } \xskakcomment{ However }} \variation[level=2]{ 1... Qd2+ \xskakcomment{ should lose in a practical game. First, Black has to find }} \variation[level=2]{ 2. Qf4 Qd8+ 3. Qf6+ Kh7 $1 4. Qxd8 \xskakcomment{ with stalemate. }} 

\mainline[level=1]{ 2. Kxh6 }
 * \aDraw Stalemate! 

\variation[level=2]{ 2. Kf5 \xskakcomment{ keeps the game going, but it's an easy draw anyway. }} 




\clearpage 
\newpage 

\section{133.??.??}
\fenboard{r4rk1/pp5p/2pp4/2n3q1/2b2p1R/2N3P1/PP1QPPB1/2KR4 b - - 0 1}
\begin{center}
\showinverseboard 
\end{center}
\clearpage 
\newpage 
\mainline{1... Bxa2 $1 { White is lost due to: } 2. Nxa2 Nb3+ { * }}\fenboard{r4rk1/pp5p/2pp4/2n3q1/2b2p1R/2N3P1/PP1QPPB1/2KR4 b - - 0 1}
\begin{center}
\showinverseboard 
\end{center}
\fenboard{r4rk1/pp5p/2pp4/2n3q1/2b2p1R/2N3P1/PP1QPPB1/2KR4 b - - 0 1}
\mainline[level=1]{1... Bxa2 $1 }
 White is lost due to: 

\mainline[level=1]{ 2. Nxa2 Nb3+ }
 * 




\clearpage 
\newpage 

\section{134.??.??}
\fenboard{2q4k/p6p/6p1/5p2/5P1P/1Bb2QP1/Pr6/3R2K1 w - - 0 1}
\begin{center}
\showboard 
\end{center}
\clearpage 
\newpage 
\mainline{1. Rd8+ $1 { Deflection. } 1... Qxd8 2. Qxc3+ Qf6 3. Qxf6# { mate * }}\fenboard{2q4k/p6p/6p1/5p2/5P1P/1Bb2QP1/Pr6/3R2K1 w - - 0 1}
\begin{center}
\showboard 
\end{center}
\fenboard{2q4k/p6p/6p1/5p2/5P1P/1Bb2QP1/Pr6/3R2K1 w - - 0 1}
\mainline[level=1]{1. Rd8+ $1 }
 Deflection. 

\variation[level=2]{ \xskakcomment{ But not }} \variation[level=2]{ 1. Rc1 $4 Bd4+ $18 }

\mainline[level=1]{ 1... Qxd8 2. Qxc3+ Qf6 3. Qxf6# }
 mate * 




\clearpage 
\newpage 

\section{135.??.??}
\fenboard{3r2k1/5rbp/pp3np1/3P4/1B1N1R2/8/P5PP/5RK1 b - - 0 1}
\begin{center}
\showinverseboard 
\end{center}
\clearpage 
\newpage 
\mainline{1... Nxd5 $1 2. Ne6 Nxf4 $19 { White resigned after: } 3. Rxf4 Rxf4 { * }}\fenboard{3r2k1/5rbp/pp3np1/3P4/1B1N1R2/8/P5PP/5RK1 b - - 0 1}
\begin{center}
\showinverseboard 
\end{center}
\fenboard{3r2k1/5rbp/pp3np1/3P4/1B1N1R2/8/P5PP/5RK1 b - - 0 1}
\mainline[level=1]{1... Nxd5 $1 2. Ne6 }

\variation[level=2]{ 2. Rxf7 Bxd4+ 3. Kh1 Nxb4 $19 \xskakcomment{ * }} 

\mainline[level=1]{ 2... Nxf4 $19 }
 White resigned after: 

\mainline[level=1]{ 3. Rxf4 Rxf4 }
 * 




\clearpage 
\newpage 

\section{136.??.??}
\fenboard{r1bqr1k1/p1p2ppp/3b1n2/2P3B1/8/2NQ1N2/PP3PPP/R4RK1 b - - 0 1}
\begin{center}
\showinverseboard 
\end{center}
\clearpage 
\newpage 
\mainline{1... Bxh2+ $1 { * Discovered attack. }}\fenboard{r1bqr1k1/p1p2ppp/3b1n2/2P3B1/8/2NQ1N2/PP3PPP/R4RK1 b - - 0 1}
\begin{center}
\showinverseboard 
\end{center}
\fenboard{r1bqr1k1/p1p2ppp/3b1n2/2P3B1/8/2NQ1N2/PP3PPP/R4RK1 b - - 0 1}
\mainline[level=1]{1... Bxh2+ $1 }
 * Discovered attack. 




\clearpage 
\newpage 

\section{137.??.??}
\fenboard{r4rk1/2p3b1/3p2qp/p1PPp3/4Rn2/P5Q1/3B1NPP/2R3K1 b - - 0 1}
\begin{center}
\showinverseboard 
\end{center}
\clearpage 
\newpage 
\mainline{1... Qxe4 $1 2. Nxe4 Ne2+ 3. Kh1 Nxg3+ { * Black should be winning with the
extra exchange, but failed to convert (1/2-1/2, 57 moves). }}\fenboard{r4rk1/2p3b1/3p2qp/p1PPp3/4Rn2/P5Q1/3B1NPP/2R3K1 b - - 0 1}
\begin{center}
\showinverseboard 
\end{center}
\fenboard{r4rk1/2p3b1/3p2qp/p1PPp3/4Rn2/P5Q1/3B1NPP/2R3K1 b - - 0 1}
\mainline[level=1]{1... Qxe4 $1 2. Nxe4 Ne2+ 3. Kh1 Nxg3+ }
 * Black should be winning with the
extra exchange, but failed to convert (\aDraw, 57 moves). 




\clearpage 
\newpage 

\section{138.??.??}
\fenboard{2rq1rk1/pp3pb1/3p2pB/n4P2/4pP2/1BNQ4/PKP3P1/3R3R w - - 0 1}
\begin{center}
\showboard 
\end{center}
\clearpage 
\newpage 
\mainline{1. Bxg7 $1 exd3 2. f6 dxc2 3. Rh8# { mate * }}\fenboard{2rq1rk1/pp3pb1/3p2pB/n4P2/4pP2/1BNQ4/PKP3P1/3R3R w - - 0 1}
\begin{center}
\showboard 
\end{center}
\fenboard{2rq1rk1/pp3pb1/3p2pB/n4P2/4pP2/1BNQ4/PKP3P1/3R3R w - - 0 1}
\mainline[level=1]{1. Bxg7 $1 exd3 }

\variation[level=2]{ \xskakcomment{ The game ended after }} \variation[level=2]{ 1... Kxg7 2. Nxe4 \xskakcomment{ when either White's
attack or his extra piece would have been enough on their own. }} 

\mainline[level=1]{ 2. f6 }

\variation[level=2]{ \xskakcomment{ Or }} \variation[level=2]{ 2. Bd4 $18 }

\mainline[level=1]{ 2... dxc2 3. Rh8# }
 mate * 




\clearpage 
\newpage 

\section{139.??.??}
\fenboard{rr4k1/1q3p2/4b2p/p2n2p1/2B5/Q3P3/PP1R1PPP/3R2K1 b - - 0 1}
\begin{center}
\showinverseboard 
\end{center}
\clearpage 
\newpage 
\mainline{1... Nxe3 $1 { Black is a piece up, but there is still work to be done. Black
decided the game on the spot. } 2. Qxe3 Bxc4 $19 { * (0-1, 31 moves) }}\fenboard{rr4k1/1q3p2/4b2p/p2n2p1/2B5/Q3P3/PP1R1PPP/3R2K1 b - - 0 1}
\begin{center}
\showinverseboard 
\end{center}
\fenboard{rr4k1/1q3p2/4b2p/p2n2p1/2B5/Q3P3/PP1R1PPP/3R2K1 b - - 0 1}
\mainline[level=1]{1... Nxe3 $1 }
 Black is a piece up, but there is still work to be done. Black
decided the game on the spot. 

\mainline[level=1]{ 2. Qxe3 }

\variation[level=2]{ 2. Bxe6 Qxg2# \xskakcomment{ mate * }} 

\mainline[level=1]{ 2... Bxc4 $19 }
 * (\blackWins, 31 moves) 




\clearpage 
\newpage 

\section{140.??.??}
\fenboard{3r1rk1/ppp1qpp1/2p1b2p/4P3/3nNQ1P/5N2/PPP2PP1/R4RK1 w - - 0 1}
\begin{center}
\showboard 
\end{center}
\clearpage 
\newpage 
\mainline{1. Nxd4 Rxd4 2. Nf6+ $1 Qxf6 3. Qxd4 { * White has a clear advantage and the game ended abruptly
after a further blunder by Black: } 3... Rd8 4. Qe4 Bd5 $2 5. Qxd5}\fenboard{3r1rk1/ppp1qpp1/2p1b2p/4P3/3nNQ1P/5N2/PPP2PP1/R4RK1 w - - 0 1}
\begin{center}
\showboard 
\end{center}
\fenboard{3r1rk1/ppp1qpp1/2p1b2p/4P3/3nNQ1P/5N2/PPP2PP1/R4RK1 w - - 0 1}

 White exploits Black's last move (Nd4) with a simple discovered attack. 

\mainline[level=1]{ 1. Nxd4 }

\variation[level=2]{ 1. Nf6+ $2 Qxf6 $1 \xskakcomment{ and White has to play }} \variation[level=2]{ 2. Qxd4 $15 }

\mainline[level=1]{ 1... Rxd4 2. Nf6+ $1 Qxf6 3. Qxd4 }
 * White has a clear advantage and the game ended abruptly
after a further blunder by Black: 




\clearpage 
\newpage 

\section{141.??.??}
\fenboard{2brrbk1/2q2p1p/p4np1/1ppPp3/4P3/BP2N1P1/P3QPBP/2RR2K1 w - - 0 1}
\begin{center}
\showboard 
\end{center}
\clearpage 
\newpage 
\mainline{1. Bxc5 $1 Bxc5 2. b4 $18 { * } 2... Nd7 $2 { The only critical move, but it is simply bad: } 3. bxc5 Nxc5 $2 4. Qc2 $18}\fenboard{2brrbk1/2q2p1p/p4np1/1ppPp3/4P3/BP2N1P1/P3QPBP/2RR2K1 w - - 0 1}
\begin{center}
\showboard 
\end{center}
\fenboard{2brrbk1/2q2p1p/p4np1/1ppPp3/4P3/BP2N1P1/P3QPBP/2RR2K1 w - - 0 1}
\mainline[level=1]{1. Bxc5 $1 Bxc5 }

\variation[level=2]{ \xskakcomment{ The game saw }} \variation[level=2]{ 1... Qb8 $18 }

\mainline[level=1]{ 2. b4 $18 }
 * 

\mainline[level=1]{ 2... Nd7 $2 }
 The only critical move, but it is simply bad: 




\clearpage 
\newpage 

\section{142.??.??}
\fenboard{r4rk1/ppp1bppp/2N2n2/7q/6b1/2P5/PP1NBPPP/R1B1QRK1 b - - 0 1}
\begin{center}
\showinverseboard 
\end{center}
\clearpage 
\newpage 
\mainline{1... Bd6 $1 { Moving the threatened piece out of danger with a dangerous threat. } 2. h3 Bxe2 3. Nd4 Bxf1 $19 { * (0-1, 36 moves) }}\fenboard{r4rk1/ppp1bppp/2N2n2/7q/6b1/2P5/PP1NBPPP/R1B1QRK1 b - - 0 1}
\begin{center}
\showinverseboard 
\end{center}
\fenboard{r4rk1/ppp1bppp/2N2n2/7q/6b1/2P5/PP1NBPPP/R1B1QRK1 b - - 0 1}
\mainline[level=1]{1... Bd6 $1 }
 Moving the threatened piece out of danger with a dangerous threat. 

\variation[level=2]{\xskakcomment{\noindent\textbf{a)} } \xskakcomment{ Not }} \variation[level=2]{ 1... bxc6 2. Bxg4 }

\variation[level=2]{\xskakcomment{\noindent\textbf{b)} } \xskakcomment{ or }} \variation[level=2]{ 1... Bxe2 $2 2. Nxe7+ $16 }

\mainline[level=1]{ 2. h3 Bxe2 3. Nd4 Bxf1 $19 }
 * (\blackWins, 36 moves) 




\clearpage 
\newpage 

\section{143.??.??}
\fenboard{r1nq1n2/3b2k1/p2p1p1R/Pp1Pp1p1/1Pp1P1P1/2P1BPN1/2BQ2K1/8 w - - 0 1}
\begin{center}
\showboard 
\end{center}
\clearpage 
\newpage 
\mainline{1. Rxf6 $1 { Taking the rook loses the queen, so Black resigned. } 1... Qxf6 2. Nh5+ $18 { * }}\fenboard{r1nq1n2/3b2k1/p2p1p1R/Pp1Pp1p1/1Pp1P1P1/2P1BPN1/2BQ2K1/8 w - - 0 1}
\begin{center}
\showboard 
\end{center}
\fenboard{r1nq1n2/3b2k1/p2p1p1R/Pp1Pp1p1/1Pp1P1P1/2P1BPN1/2BQ2K1/8 w - - 0 1}
\mainline[level=1]{1. Rxf6 $1 }
 Taking the rook loses the queen, so Black resigned. 

\variation[level=2]{ \xskakcomment{ Instead if White had retreated the rook with, for example, }} \variation[level=2]{ 1. Rh3 \xskakcomment{ then he wou
ld still have some work to do, although \whiteWins does seem the most probable result
(no points). }} 

\mainline[level=1]{ 1... Qxf6 }

\variation[level=2]{ 1... Kxf6 2. Bxg5+ $18 \xskakcomment{ * }} 

\mainline[level=1]{ 2. Nh5+ $18 }
 * 




\clearpage 
\newpage 

\section{144.??.??}
\fenboard{4rbk1/3Q1ppp/3p4/3P4/5q2/B7/P5PP/5RK1 b - - 0 1}
\begin{center}
\showinverseboard 
\end{center}
\clearpage 
\newpage 
\mainline{1... Qd4+ $1 2. Kh1 Qf2 $1 $19 { * }}\fenboard{4rbk1/3Q1ppp/3p4/3P4/5q2/B7/P5PP/5RK1 b - - 0 1}
\begin{center}
\showinverseboard 
\end{center}
\fenboard{4rbk1/3Q1ppp/3p4/3P4/5q2/B7/P5PP/5RK1 b - - 0 1}
\mainline[level=1]{1... Qd4+ $1 }

\variation[level=2]{ 1... Qe3+ 2. Rf2 } (\variation[level=3]{ 2. Kh1 Qf2 $19 })
\variation[level=2]{ 2... Re7 $1 3. Qa4 Ra7 $1 \xskakcomment{ will also win. }} 

\mainline[level=1]{ 2. Kh1 }

\variation[level=2]{ 2. Rf2 Re1# \xskakcomment{ mate * }} 

\mainline[level=1]{ 2... Qf2 $1 $19 }
 * 




\clearpage 
\newpage 

\section{145.??.??}
\fenboard{r1b5/5pk1/1p1p3p/3Pq1p1/PQ2Pn2/5P2/5RPP/3R3K b - - 0 1}
\begin{center}
\showinverseboard 
\end{center}
\clearpage 
\newpage 
\mainline{1... Nd3 $1 { An unexpected fork after White's last move Rf1-f2? } 2. Qxb6 Nxf2+ $18 { * Black is a piece up and has the more
active heavy pieces. White resigned a few moves later. }}\fenboard{r1b5/5pk1/1p1p3p/3Pq1p1/PQ2Pn2/5P2/5RPP/3R3K b - - 0 1}
\begin{center}
\showinverseboard 
\end{center}
\fenboard{r1b5/5pk1/1p1p3p/3Pq1p1/PQ2Pn2/5P2/5RPP/3R3K b - - 0 1}
\mainline[level=1]{1... Nd3 $1 }
 An unexpected fork after White's last move Rf1-f2? 

\mainline[level=1]{ 2. Qxb6 }

\variation[level=2]{ 2. Rxd3 Qa1+ \xskakcomment{ * mating. }} 

\mainline[level=1]{ 2... Nxf2+ $18 }
 * Black is a piece up and has the more
active heavy pieces. White resigned a few moves later. 




\clearpage 
\newpage 

\section{146.??.??}
\fenboard{6rk/7p/p2Q4/1p2r3/8/4NP1q/PPP4P/5R1K b - - 0 1}
\begin{center}
\showinverseboard 
\end{center}
\clearpage 
\newpage 
\mainline{1... Qxh2+ $1 { The only winning move. 0-1 } 2. Kxh2 Rh5# { mate * }}\fenboard{6rk/7p/p2Q4/1p2r3/8/4NP1q/PPP4P/5R1K b - - 0 1}
\begin{center}
\showinverseboard 
\end{center}
\fenboard{6rk/7p/p2Q4/1p2r3/8/4NP1q/PPP4P/5R1K b - - 0 1}
\mainline[level=1]{1... Qxh2+ $1 }
 The only winning move. \blackWins 

\mainline[level=1]{ 2. Kxh2 Rh5# }
 mate * 




\clearpage 
\newpage 

\section{147.??.??}
\fenboard{1r2b1k1/R4nqp/4p3/1pR5/1P1P1r2/5N2/1Q2B1P1/6K1 w - - 0 1}
\begin{center}
\showboard 
\end{center}
\clearpage 
\newpage 
\mainline{1. Rg5 $1 { Using the fact that the knight on f7 is pinned. } 1... Qxg5 2. Nxg5 { * } 2... Nxg5 3. d5}\fenboard{1r2b1k1/R4nqp/4p3/1pR5/1P1P1r2/5N2/1Q2B1P1/6K1 w - - 0 1}
\begin{center}
\showboard 
\end{center}
\fenboard{1r2b1k1/R4nqp/4p3/1pR5/1P1P1r2/5N2/1Q2B1P1/6K1 w - - 0 1}
\mainline[level=1]{1. Rg5 $1 }
 Using the fact that the knight on f7 is pinned. 

\mainline[level=1]{ 1... Qxg5 }

\variation[level=2]{ 1... Nxg5 2. Rxg7+ Kxg7 3. Nxg5 $18 \xskakcomment{ * and White is up too much material. }} 

\mainline[level=1]{ 2. Nxg5 }
 * 




\clearpage 
\newpage 

\section{148.??.??}
\fenboard{2rb1qk1/1b4pp/3p4/1P2pr2/2n2PB1/2NQ2P1/P1N4P/R2R2K1 b - - 0 1}
\begin{center}
\showinverseboard 
\end{center}
\clearpage 
\newpage 
\mainline{1... Bb6+ $1 2. Kf1 Rxf4+ $1 3. gxf4 Qxf4+ { * 0-1 Black has a mating attack. }}\fenboard{2rb1qk1/1b4pp/3p4/1P2pr2/2n2PB1/2NQ2P1/P1N4P/R2R2K1 b - - 0 1}
\begin{center}
\showinverseboard 
\end{center}
\fenboard{2rb1qk1/1b4pp/3p4/1P2pr2/2n2PB1/2NQ2P1/P1N4P/R2R2K1 b - - 0 1}
\mainline[level=1]{1... Bb6+ $1 }

\variation[level=2]{\xskakcomment{\noindent\textbf{a)} } \xskakcomment{ The move order }} \variation[level=2]{ 1... Rxf4 \xskakcomment{ doesn't work as well: }} \variation[level=2]{ 2. Be6+ $1 Kh8 3. Nd5 $17 }

\variation[level=2]{\xskakcomment{\noindent\textbf{b)} } \xskakcomment{ Even worse is: }} \variation[level=2]{ 1... Nb2 $2 2. Qxf5 $16 }

\mainline[level=1]{ 2. Kf1 Rxf4+ $1 3. gxf4 Qxf4+ }
 * \blackWins Black has a mating attack. 




\clearpage 
\newpage 

\section{149.??.??}
\fenboard{8/2b2k1p/pp1P3r/2p3p1/6P1/1P2R2P/5P2/6K1 w - - 0 1}
\begin{center}
\showboard 
\end{center}
\clearpage 
\newpage 
\mainline{1. Re7+ $1 { The only way to promote the pawn. } 1... Kf8 2. dxc7 $18 Rc6 3. Rd7 Ke8 4. Rd8+ Ke7 5. c8=Q Rxc8 6. Rxc8 $18 { * }}\fenboard{8/2b2k1p/pp1P3r/2p3p1/6P1/1P2R2P/5P2/6K1 w - - 0 1}
\begin{center}
\showboard 
\end{center}
\fenboard{8/2b2k1p/pp1P3r/2p3p1/6P1/1P2R2P/5P2/6K1 w - - 0 1}
\mainline[level=1]{1. Re7+ $1 }
 The only way to promote the pawn. 

\variation[level=2]{ \xskakcomment{ And not: }} \variation[level=2]{ 1. dxc7 $2 Rc6 $17 }

\mainline[level=1]{ 1... Kf8 }

\variation[level=2]{ 1... Kf6 2. dxc7 $18 \xskakcomment{ * }} 

\mainline[level=1]{ 2. dxc7 $18 Rc6 3. Rd7 }

\variation[level=2]{ \xskakcomment{ White chose a slower way: }} \variation[level=2]{ 3. Rxh7 Ke8 4. h4 $18 }

\mainline[level=1]{ 3... Ke8 4. Rd8+ Ke7 5. c8=Q Rxc8 6. Rxc8 $18 }
 * 




\clearpage 
\newpage 

\section{150.??.??}
\fenboard{r3rnk1/1N3ppp/1p6/p1Pp2q1/8/6P1/PP2PPBP/2R1Q1K1 b - - 0 1}
\begin{center}
\showinverseboard 
\end{center}
\clearpage 
\newpage 
\mainline{1... Rxe2 $1 2. Qxe2 Qxc1+ { * Black has opened White's first rank and won a
pawn, leaving him up a full exchange, and winning. The game had a quick finish: } 3. Qf1 Qd2 4. cxb6 $2 Rc8}\fenboard{r3rnk1/1N3ppp/1p6/p1Pp2q1/8/6P1/PP2PPBP/2R1Q1K1 b - - 0 1}
\begin{center}
\showinverseboard 
\end{center}
\fenboard{r3rnk1/1N3ppp/1p6/p1Pp2q1/8/6P1/PP2PPBP/2R1Q1K1 b - - 0 1}
\mainline[level=1]{1... Rxe2 $1 2. Qxe2 Qxc1+ }
 * Black has opened White's first rank and won a
pawn, leaving him up a full exchange, and winning. The game had a quick finish: 




\clearpage 
\newpage 

\section{151.??.??}
\fenboard{8/6k1/1P1p2p1/3Ppn2/3q4/8/6PP/rR2QB1K b - - 0 1}
\begin{center}
\showinverseboard 
\end{center}
\clearpage 
\newpage 
\mainline{1... Ng3+ $1 { White resigned in view of } 2. hxg3 Ra8 $1 { * with mate. }}\fenboard{8/6k1/1P1p2p1/3Ppn2/3q4/8/6PP/rR2QB1K b - - 0 1}
\begin{center}
\showinverseboard 
\end{center}
\fenboard{8/6k1/1P1p2p1/3Ppn2/3q4/8/6PP/rR2QB1K b - - 0 1}
\mainline[level=1]{1... Ng3+ $1 }
 White resigned in view of 

\mainline[level=1]{ 2. hxg3 Ra8 $1 }
 * with mate. 




\clearpage 
\newpage 

\section{152.??.??}
\fenboard{8/7R/2r5/8/P3n3/8/3nk1PP/R5K1 b - - 0 1}
\begin{center}
\showinverseboard 
\end{center}
\clearpage 
\newpage 
\mainline{1... Nf3+ $1 { 0-1 Mate is coming up: } 2. gxf3 Rg6+ 3. Kh1 Nf2# { mate * }}\fenboard{8/7R/2r5/8/P3n3/8/3nk1PP/R5K1 b - - 0 1}
\begin{center}
\showinverseboard 
\end{center}
\fenboard{8/7R/2r5/8/P3n3/8/3nk1PP/R5K1 b - - 0 1}
\mainline[level=1]{1... Nf3+ $1 }
 \blackWins Mate is coming up: 

\mainline[level=1]{ 2. gxf3 Rg6+ 3. Kh1 Nf2# }
 mate * 




\clearpage 
\newpage 

\section{153.??.??}
\fenboard{3rrk2/R4ppB/7p/1p2N1q1/nPbP1p2/2PQ1P2/6PP/4R1K1 w - - 0 1}
\begin{center}
\showboard 
\end{center}
\clearpage 
\newpage 
\mainline{1. Qxc4 $1 { Black resigned instead of permitting: } 1... bxc4 2. Rxf7# { mate * }}\fenboard{3rrk2/R4ppB/7p/1p2N1q1/nPbP1p2/2PQ1P2/6PP/4R1K1 w - - 0 1}
\begin{center}
\showboard 
\end{center}
\fenboard{3rrk2/R4ppB/7p/1p2N1q1/nPbP1p2/2PQ1P2/6PP/4R1K1 w - - 0 1}
\mainline[level=1]{1. Qxc4 $1 }
 Black resigned instead of permitting: 

\mainline[level=1]{ 1... bxc4 2. Rxf7# }
 mate * 




\clearpage 
\newpage 

\section{154.??.??}
\fenboard{5kr1/5p2/8/1Q1bPp2/8/P5P1/q1rBBK1P/3R4 w - - 0 1}
\begin{center}
\showboard 
\end{center}
\clearpage 
\newpage 
\mainline{1. Qxd5 $1 { Move order is important here. } 1... Qxd5 2. Bh6+ Ke7 3. Rxd5 $18 { * (1-0, 38 moves) }}\fenboard{5kr1/5p2/8/1Q1bPp2/8/P5P1/q1rBBK1P/3R4 w - - 0 1}
\begin{center}
\showboard 
\end{center}
\fenboard{5kr1/5p2/8/1Q1bPp2/8/P5P1/q1rBBK1P/3R4 w - - 0 1}
\mainline[level=1]{1. Qxd5 $1 }
 Move order is important here. 

\variation[level=2]{\xskakcomment{\noindent\textbf{a)} } 1. Bh6+ Ke7 $14 2. Rxd5 $2 } (\variation[level=3]{ 2. Qb4+ Kd7 3. Rd2 Rxd2 4. Bxd2 Rg6 $14 })
\variation[level=2]{ 2... Rxe2+ $1 }

\variation[level=2]{\xskakcomment{\noindent\textbf{b)} } \xskakcomment{ or }} \variation[level=2]{ 1. Qb4+ $5 Kg7 2. Qh4 Re8 $1 $14 \xskakcomment{ still with great chances against Black's exposed king. }} 

\mainline[level=1]{ 1... Qxd5 2. Bh6+ }

\variation[level=2]{ 2. Bb4+ $2 Qc5+ \xskakcomment{ = }} 

\mainline[level=1]{ 2... Ke7 3. Rxd5 $18 }
 * (\whiteWins, 38 moves) 




\clearpage 
\newpage 

\section{155.??.??}
\fenboard{3rrnk1/1bq2pp1/ppnppb1p/8/2P1PPQ1/1PN2N2/PB4PP/1B1R1R1K w - - 0 1}
\begin{center}
\showboard 
\end{center}
\clearpage 
\newpage 
\mainline{1. Nd5 $1 Bxb2 2. Nxc7 $18 { * (1-0, 40 moves) }}\fenboard{3rrnk1/1bq2pp1/ppnppb1p/8/2P1PPQ1/1PN2N2/PB4PP/1B1R1R1K w - - 0 1}
\begin{center}
\showboard 
\end{center}
\fenboard{3rrnk1/1bq2pp1/ppnppb1p/8/2P1PPQ1/1PN2N2/PB4PP/1B1R1R1K w - - 0 1}
\mainline[level=1]{1. Nd5 $1 }

\variation[level=2]{ 1. Nb5 \xskakcomment{ is a worse variant since the queen can escape to e7, but it
still gives a winning advantage: }} \variation[level=2]{ 1... Qe7 2. Bxf6 Qxf6 3. Nxd6 $18 \xskakcomment{ (full points) }} 

\mainline[level=1]{ 1... Bxb2 }

\variation[level=2]{ 1... exd5 2. Bxf6 Ng6 3. Bxd8 $18 \xskakcomment{ * }} 

\mainline[level=1]{ 2. Nxc7 $18 }
 * (\whiteWins, 40 moves) 




\clearpage 
\newpage 

\section{156.??.??}
\fenboard{1n3rk1/p2pQpp1/4q2p/1r6/4B3/4P1P1/P4P1P/2RR2K1 w - - 0 1}
\begin{center}
\showboard 
\end{center}
\clearpage 
\newpage 
\mainline{1. Bh7+ $1 { Deflection. } 1... Kxh7 2. Qxf8 $18 { * (1-0, 40 moves) }}\fenboard{1n3rk1/p2pQpp1/4q2p/1r6/4B3/4P1P1/P4P1P/2RR2K1 w - - 0 1}
\begin{center}
\showboard 
\end{center}
\fenboard{1n3rk1/p2pQpp1/4q2p/1r6/4B3/4P1P1/P4P1P/2RR2K1 w - - 0 1}
\mainline[level=1]{1. Bh7+ $1 }
 Deflection. 

\mainline[level=1]{ 1... Kxh7 2. Qxf8 $18 }
 * (\whiteWins, 40 moves) 




\clearpage 
\newpage 

\section{157.??.??}
\fenboard{4R3/p4pk1/5qpp/1PnP4/3Q3P/6P1/r1r1BPK1/3R4 w - - 0 1}
\begin{center}
\showboard 
\end{center}
\clearpage 
\newpage 
\mainline{1. Rg8+ $1 { Deflection. } 1... Kxg8 2. Qxf6 { * (1-0, 45 moves) }}\fenboard{4R3/p4pk1/5qpp/1PnP4/3Q3P/6P1/r1r1BPK1/3R4 w - - 0 1}
\begin{center}
\showboard 
\end{center}
\fenboard{4R3/p4pk1/5qpp/1PnP4/3Q3P/6P1/r1r1BPK1/3R4 w - - 0 1}
\mainline[level=1]{1. Rg8+ $1 }
 Deflection. 

\mainline[level=1]{ 1... Kxg8 2. Qxf6 }
 * (\whiteWins, 45 moves) 




\clearpage 
\newpage 

\section{158.??.??}
\fenboard{4rrk1/1p1n1pb1/2ppq1p1/p3p1Bp/P1P1P2P/2NP2P1/1P2QPK1/3R1R2 w - - 0 1}
\begin{center}
\showboard 
\end{center}
\clearpage 
\newpage 
\mainline{1. Nd5 $1 { Exploiting the claustrophobic queen on e6. } 1... Nc5 2. Nc7 Qd7 3. Nxe8 $16 { * Black saved a draw (42 moves). }}\fenboard{4rrk1/1p1n1pb1/2ppq1p1/p3p1Bp/P1P1P2P/2NP2P1/1P2QPK1/3R1R2 w - - 0 1}
\begin{center}
\showboard 
\end{center}
\fenboard{4rrk1/1p1n1pb1/2ppq1p1/p3p1Bp/P1P1P2P/2NP2P1/1P2QPK1/3R1R2 w - - 0 1}
\mainline[level=1]{1. Nd5 $1 }
 Exploiting the claustrophobic queen on e6. 

\mainline[level=1]{ 1... Nc5 }

\variation[level=2]{\xskakcomment{\noindent\textbf{a)} } \xskakcomment{ Black's alternatives are no better: }} \variation[level=2]{ 1... cxd5 2. cxd5 $18 \xskakcomment{ * }} 

\variation[level=2]{\xskakcomment{\noindent\textbf{b)} } 1... Rb8 2. Nc7 $18 \xskakcomment{ * }} 

\variation[level=2]{\xskakcomment{\noindent\textbf{c)} } \xskakcomment{ or }} \variation[level=2]{ 1... Rc8 2. Ne7+ $16 \xskakcomment{ * }} 

\mainline[level=1]{ 2. Nc7 Qd7 3. Nxe8 $16 }
 * Black saved a draw (42 moves). 




\clearpage 
\newpage 

\section{159.??.??}
\fenboard{r3rbk1/1b3Npp/1q1p2n1/2pP4/5R2/1PN3Q1/4P1BP/5R1K w - - 0 1}
\begin{center}
\showboard 
\end{center}
\clearpage 
\newpage 
\mainline{1. Qxg6 $1 { 1-0 White was winning anyway, but this is too nice to pass up. } 1... hxg6 2. Rh4 { And on next move Rh8 mate * is unavoidable. }}\fenboard{r3rbk1/1b3Npp/1q1p2n1/2pP4/5R2/1PN3Q1/4P1BP/5R1K w - - 0 1}
\begin{center}
\showboard 
\end{center}
\fenboard{r3rbk1/1b3Npp/1q1p2n1/2pP4/5R2/1PN3Q1/4P1BP/5R1K w - - 0 1}
\mainline[level=1]{1. Qxg6 $1 }
 \whiteWins White was winning anyway, but this is too nice to pass up. 

\mainline[level=1]{ 1... hxg6 2. Rh4 }
 And on next move Rh8 mate * is unavoidable. 




\clearpage 
\newpage 

\section{160.??.??}
\fenboard{5nk1/1b1qQpp1/p3p3/3pP1N1/3P3P/1p1B4/1P3PP1/6K1 w - - 0 1}
\begin{center}
\showboard 
\end{center}
\clearpage 
\newpage 
\mainline{1. Bh7+ $1 { Deflecting the knight or king. } 1... Nxh7 2. Qxd7 $18 { * }}\fenboard{5nk1/1b1qQpp1/p3p3/3pP1N1/3P3P/1p1B4/1P3PP1/6K1 w - - 0 1}
\begin{center}
\showboard 
\end{center}
\fenboard{5nk1/1b1qQpp1/p3p3/3pP1N1/3P3P/1p1B4/1P3PP1/6K1 w - - 0 1}
\mainline[level=1]{1. Bh7+ $1 }
 Deflecting the knight or king. 

\mainline[level=1]{ 1... Nxh7 }

\variation[level=2]{ 1... Kh8 2. Qxf8# \xskakcomment{ mate * }} 

\mainline[level=1]{ 2. Qxd7 $18 }
 * 




\clearpage 
\newpage 

\section{161.??.??}
\fenboard{4r2k/3n2bB/2p1NqQ1/2Pp4/5B2/r6P/5PP1/4R1K1 b - - 0 1}
\begin{center}
\showinverseboard 
\end{center}
\clearpage 
\newpage 
\mainline{1... Rxe6 $1 { Clearing the back rank in order to force the king to h2, so the
bishop falls with check. } 2. Rxe6 Ra1+ 3. Kh2 Qxf4+ $18 { * (0-1, 46 moves) }}\fenboard{4r2k/3n2bB/2p1NqQ1/2Pp4/5B2/r6P/5PP1/4R1K1 b - - 0 1}
\begin{center}
\showinverseboard 
\end{center}
\fenboard{4r2k/3n2bB/2p1NqQ1/2Pp4/5B2/r6P/5PP1/4R1K1 b - - 0 1}
\mainline[level=1]{1... Rxe6 $1 }
 Clearing the back rank in order to force the king to h2, so the
bishop falls with check. 

\mainline[level=1]{ 2. Rxe6 Ra1+ }

\variation[level=2]{ 2... Qxf4 $6 3. Re8+ Nf8 4. Qh5 $15 }

\mainline[level=1]{ 3. Kh2 Qxf4+ $18 }
 * (\blackWins, 46 moves) 




\clearpage 
\newpage 

\section{162.??.??}
\fenboard{2b4k/p3q2p/2p1p3/2PpPpr1/3P3Q/4P2B/3b1B1P/R6K b - - 0 1}
\begin{center}
\showinverseboard 
\end{center}
\clearpage 
\newpage 
\mainline{1... Bxe3 $1 { Using the fact that pins can sometimes become discovered attacks instead. } 2. Bxe3 $6 Rg1+ 3. Rxg1 Qxh4 4. Bg5 Qe4+ $1 5. Bg2 Qxd4 { * White has no time to move the bishop from g2, since the e5-pawn is
hanging. After } 6. Bf6+ Kg8 $19 { there is no good discovered check (0-1, 50 moves). }}\fenboard{2b4k/p3q2p/2p1p3/2PpPpr1/3P3Q/4P2B/3b1B1P/R6K b - - 0 1}
\begin{center}
\showinverseboard 
\end{center}
\fenboard{2b4k/p3q2p/2p1p3/2PpPpr1/3P3Q/4P2B/3b1B1P/R6K b - - 0 1}
\mainline[level=1]{1... Bxe3 $1 }
 Using the fact that pins can sometimes become discovered attacks instead. 

\mainline[level=1]{ 2. Bxe3 $6 }

\variation[level=2]{ 2. Bg2 Bxf2 3. Qxf2 $17 }

\mainline[level=1]{ 2... Rg1+ 3. Rxg1 Qxh4 4. Bg5 Qe4+ $1 5. Bg2 Qxd4 }
 * White has no time to move the bishop from g2, since the e5-pawn is
hanging. After 

\mainline[level=1]{ 6. Bf6+ Kg8 $19 }
 there is no good discovered check (\blackWins, 50 moves). 




\clearpage 
\newpage 

\section{163.??.??}
\fenboard{7r/3b1pk1/2p2qp1/8/1P1bNP2/r2B4/3Q2PP/2R2R1K b - - 0 1}
\begin{center}
\showinverseboard 
\end{center}
\clearpage 
\newpage 
\mainline{1... Rxh2+ { White resigned due to: } 2. Kxh2 Qh4# { mate * }}\fenboard{7r/3b1pk1/2p2qp1/8/1P1bNP2/r2B4/3Q2PP/2R2R1K b - - 0 1}
\begin{center}
\showinverseboard 
\end{center}
\fenboard{7r/3b1pk1/2p2qp1/8/1P1bNP2/r2B4/3Q2PP/2R2R1K b - - 0 1}
\mainline[level=1]{1... Rxh2+ }
 White resigned due to: 

\mainline[level=1]{ 2. Kxh2 Qh4# }
 mate * 




\clearpage 
\newpage 

\section{164.??.??}
\fenboard{1r1r4/5ppk/1p5p/2b1Pp1Q/8/q7/3B1PPP/2RR2K1 b - - 0 1}
\begin{center}
\showinverseboard 
\end{center}
\clearpage 
\newpage 
\mainline{1... Rxd2 $1 2. Rxd2 Qxc1+ $18 { * }}\fenboard{1r1r4/5ppk/1p5p/2b1Pp1Q/8/q7/3B1PPP/2RR2K1 b - - 0 1}
\begin{center}
\showinverseboard 
\end{center}
\fenboard{1r1r4/5ppk/1p5p/2b1Pp1Q/8/q7/3B1PPP/2RR2K1 b - - 0 1}
\mainline[level=1]{1... Rxd2 $1 2. Rxd2 }

\variation[level=2]{ \xskakcomment{ After the game move }} \variation[level=2]{ 2. Qh4 $19 \xskakcomment{ Black has simply won a piece, so White resigned in a couple of moves. }} 

\mainline[level=1]{ 2... Qxc1+ $18 }
 * 




\clearpage 
\newpage 

\section{165.??.??}
\fenboard{4kr2/p1Bbpp2/1p4p1/n6p/2B1P2P/2K2P2/P2R2P1/8 w - - 0 1}
\begin{center}
\showboard 
\end{center}
\clearpage 
\newpage 
\mainline{1. Bb5 $1 { The X-ray mate threat wins the bishop. Black resigned instead of allowing: } 1... Bxb5 2. Rd8# { mate * }}\fenboard{4kr2/p1Bbpp2/1p4p1/n6p/2B1P2P/2K2P2/P2R2P1/8 w - - 0 1}
\begin{center}
\showboard 
\end{center}
\fenboard{4kr2/p1Bbpp2/1p4p1/n6p/2B1P2P/2K2P2/P2R2P1/8 w - - 0 1}
\mainline[level=1]{1. Bb5 $1 }
 The X-ray mate threat wins the bishop. Black resigned instead of allowing: 

\mainline[level=1]{ 1... Bxb5 2. Rd8# }
 mate * 




\clearpage 
\newpage 

\section{166.??.??}
\fenboard{2n2k2/2b2pp1/2p1pnp1/2Pp2N1/3P1PP1/3BP2P/3BK3/8 w - - 0 1}
\begin{center}
\showboard 
\end{center}
\clearpage 
\newpage 
\mainline{1. Bxg6 $1 Bxf4 2. exf4 fxg6 3. Nxe6+ $18 { (1-0, 69 moves) }}\fenboard{2n2k2/2b2pp1/2p1pnp1/2Pp2N1/3P1PP1/3BP2P/3BK3/8 w - - 0 1}
\begin{center}
\showboard 
\end{center}
\fenboard{2n2k2/2b2pp1/2p1pnp1/2Pp2N1/3P1PP1/3BP2P/3BK3/8 w - - 0 1}
\mainline[level=1]{1. Bxg6 $1 Bxf4 }

\variation[level=2]{ 1... fxg6 2. Nxe6+ Ke7 3. Nxc7 $18 \xskakcomment{ * }} 

\mainline[level=1]{ 2. exf4 }

\variation[level=2]{\xskakcomment{\noindent\textbf{a)} } \xskakcomment{ Or }} \variation[level=2]{ 2. Nxf7 $18 }

\variation[level=2]{\xskakcomment{\noindent\textbf{b)} } \xskakcomment{ or }} \variation[level=2]{ 2. Nxe6+ fxe6 3. exf4 $18 }

\mainline[level=1]{ 2... fxg6 3. Nxe6+ $18 }
 (\whiteWins, 69 moves) 




\clearpage 
\newpage 

\section{167.??.??}
\fenboard{r5k1/pb1p1Rpq/2n1p1Q1/1p2P3/3P2P1/P7/1P1K4/5R2 w - - 0 1}
\begin{center}
\showboard 
\end{center}
\clearpage 
\newpage 
\mainline{1. Rf8+ $1 { Deflection. } 1... Rxf8 2. Rxf8+ Kxf8 3. Qxh7 $18 { * (1-0, 42 moves) }}\fenboard{r5k1/pb1p1Rpq/2n1p1Q1/1p2P3/3P2P1/P7/1P1K4/5R2 w - - 0 1}
\begin{center}
\showboard 
\end{center}
\fenboard{r5k1/pb1p1Rpq/2n1p1Q1/1p2P3/3P2P1/P7/1P1K4/5R2 w - - 0 1}
\mainline[level=1]{1. Rf8+ $1 }
 Deflection. 

\variation[level=2]{ \xskakcomment{ Worse is }} \variation[level=2]{ 1. Qxh7+ Kxh7 2. Rxd7 $16 }

\mainline[level=1]{ 1... Rxf8 2. Rxf8+ Kxf8 3. Qxh7 $18 }
 * (\whiteWins, 42 moves) 




\clearpage 
\newpage 

\section{168.??.??}
\fenboard{1k1r1q1r/2p2ppp/Qp6/3b1N2/8/P5P1/1P3P1P/2R2RK1 w - - 0 1}
\begin{center}
\showboard 
\end{center}
\clearpage 
\newpage 
\mainline{1. Rxc7 $1 Kxc7 2. Qa7+ { * 1-0 With the rook coming to c1, the attack will be deadly. }}\fenboard{1k1r1q1r/2p2ppp/Qp6/3b1N2/8/P5P1/1P3P1P/2R2RK1 w - - 0 1}
\begin{center}
\showboard 
\end{center}
\fenboard{1k1r1q1r/2p2ppp/Qp6/3b1N2/8/P5P1/1P3P1P/2R2RK1 w - - 0 1}
\mainline[level=1]{1. Rxc7 $1 }

\variation[level=2]{ 1. Nd4 \xskakcomment{ threatens Nb5 with mate, but Black can defend after }} \variation[level=2]{ 1... Qe7 2. Nb5 c6 $16 }

\mainline[level=1]{ 1... Kxc7 2. Qa7+ }
 * \whiteWins With the rook coming to c1, the attack will be deadly. 




\clearpage 
\newpage 

\section{169.??.??}
\fenboard{6k1/4qp2/1B4pp/2R1b3/P2Nb3/1P2Q3/r4PPP/6K1 b - - 0 1}
\begin{center}
\showinverseboard 
\end{center}
\clearpage 
\newpage 
\mainline{1... Qxc5 $1 { * The back-rank mate threat means Black has simply won a rook,
leaving him easily winning. }}\fenboard{6k1/4qp2/1B4pp/2R1b3/P2Nb3/1P2Q3/r4PPP/6K1 b - - 0 1}
\begin{center}
\showinverseboard 
\end{center}
\fenboard{6k1/4qp2/1B4pp/2R1b3/P2Nb3/1P2Q3/r4PPP/6K1 b - - 0 1}
\mainline[level=1]{1... Qxc5 $1 }
 * The back-rank mate threat means Black has simply won a rook,
leaving him easily winning. 




\clearpage 
\newpage 

\section{170.??.??}
\fenboard{R3n1k1/7p/3prppB/1p2p3/1P2P2Q/7P/5PPK/1q6 w - - 0 1}
\begin{center}
\showboard 
\end{center}
\clearpage 
\newpage 
\mainline{1. Rxe8+ Rxe8 2. Qxf6 { * White is mating. }}\fenboard{R3n1k1/7p/3prppB/1p2p3/1P2P2Q/7P/5PPK/1q6 w - - 0 1}
\begin{center}
\showboard 
\end{center}
\fenboard{R3n1k1/7p/3prppB/1p2p3/1P2P2Q/7P/5PPK/1q6 w - - 0 1}
\mainline[level=1]{1. Rxe8+ Rxe8 }

\variation[level=2]{ \xskakcomment{ Instead Black soon lost after }} \variation[level=2]{ 1... Kf7 $18 }

\mainline[level=1]{ 2. Qxf6 }
 * White is mating. 




\clearpage 
\newpage 

\section{171.??.??}
\fenboard{2rr4/pp4kp/6p1/3nNp2/2B5/2n3P1/P2R1PKP/R7 b - - 0 1}
\begin{center}
\showinverseboard 
\end{center}
\clearpage 
\newpage 
\mainline{1... Ne3+ { * 0-1 }}\fenboard{2rr4/pp4kp/6p1/3nNp2/2B5/2n3P1/P2R1PKP/R7 b - - 0 1}
\begin{center}
\showinverseboard 
\end{center}
\fenboard{2rr4/pp4kp/6p1/3nNp2/2B5/2n3P1/P2R1PKP/R7 b - - 0 1}
\mainline[level=1]{1... Ne3+ }
 * \blackWins 

\variation[level=2]{ \xskakcomment{ A discovered attack, which could also be executed with }} \variation[level=2]{ 1... Nf4+ $18 }




\clearpage 
\newpage 

\section{172.??.??}
\fenboard{5rk1/ppr2Npp/2q1Pp2/3n4/8/6Q1/P5PP/3R1R1K w - - 0 1}
\begin{center}
\showboard 
\end{center}
\clearpage 
\newpage 
\mainline{1. Rxd5 $1 Qxd5 2. Qxc7 $18 { * }}\fenboard{5rk1/ppr2Npp/2q1Pp2/3n4/8/6Q1/P5PP/3R1R1K w - - 0 1}
\begin{center}
\showboard 
\end{center}
\fenboard{5rk1/ppr2Npp/2q1Pp2/3n4/8/6Q1/P5PP/3R1R1K w - - 0 1}
\mainline[level=1]{1. Rxd5 $1 Qxd5 2. Qxc7 $18 }
 * 




\clearpage 
\newpage 

\section{173.??.??}
\fenboard{r7/5kpp/2p1pq2/1bP2p2/p4Q2/5BP1/PP3P1P/4R1K1 w - - 0 1}
\begin{center}
\showboard 
\end{center}
\clearpage 
\newpage 
\mainline{1. Bxc6 $1 Bxc6 2. Qc7+ Qe7 3. Qxc6 $16 { * White is a pawn up with good winning chances. }}\fenboard{r7/5kpp/2p1pq2/1bP2p2/p4Q2/5BP1/PP3P1P/4R1K1 w - - 0 1}
\begin{center}
\showboard 
\end{center}
\fenboard{r7/5kpp/2p1pq2/1bP2p2/p4Q2/5BP1/PP3P1P/4R1K1 w - - 0 1}
\mainline[level=1]{1. Bxc6 $1 Bxc6 2. Qc7+ Qe7 3. Qxc6 $16 }
 * White is a pawn up with good winning chances. 




\clearpage 
\newpage 

\section{174.??.??}
\fenboard{8/1p3pkp/pb2p1p1/8/4P2P/1N4P1/P1R1KP2/7r b - - 0 1}
\begin{center}
\showinverseboard 
\end{center}
\clearpage 
\newpage 
\mainline{1... Bxf2 $1 { Winning a second pawn due to: } 2. Kxf2 Rh2+ $18 { * }}\fenboard{8/1p3pkp/pb2p1p1/8/4P2P/1N4P1/P1R1KP2/7r b - - 0 1}
\begin{center}
\showinverseboard 
\end{center}
\fenboard{8/1p3pkp/pb2p1p1/8/4P2P/1N4P1/P1R1KP2/7r b - - 0 1}
\mainline[level=1]{1... Bxf2 $1 }
 Winning a second pawn due to: 

\mainline[level=1]{ 2. Kxf2 Rh2+ $18 }
 * 




\clearpage 
\newpage 

\section{175.??.??}
\fenboard{2r5/pp1b4/4p3/3pPk2/5P2/8/1K6/3R2R1 w - - 0 1}
\begin{center}
\showboard 
\end{center}
\clearpage 
\newpage 
\mainline{1. Rd4 $1 { * 1-0 Mating. }}\fenboard{2r5/pp1b4/4p3/3pPk2/5P2/8/1K6/3R2R1 w - - 0 1}
\begin{center}
\showboard 
\end{center}
\fenboard{2r5/pp1b4/4p3/3pPk2/5P2/8/1K6/3R2R1 w - - 0 1}
\mainline[level=1]{1. Rd4 $1 }
 * \whiteWins Mating. 

\variation[level=2]{ 1. Rg7 \xskakcomment{ is a useless intermediate move that allows Black to defend with }} \variation[level=2]{ 1... Be8 2. Rd4 Bg6 $16 }




\clearpage 
\newpage 

\section{176.??.??}
\fenboard{r2r2k1/1p1n1bpp/1pp1pp2/8/Pb1PP1P1/1BNRBP1P/1P3K2/2R5 b - - 0 1}
\begin{center}
\showinverseboard 
\end{center}
\clearpage 
\newpage 
\mainline{1... Nc5 $1 { The pin allows this fork. } 2. dxc5 Rxd3 $17 { * (0-1, 36 moves) }}\fenboard{r2r2k1/1p1n1bpp/1pp1pp2/8/Pb1PP1P1/1BNRBP1P/1P3K2/2R5 b - - 0 1}
\begin{center}
\showinverseboard 
\end{center}
\fenboard{r2r2k1/1p1n1bpp/1pp1pp2/8/Pb1PP1P1/1BNRBP1P/1P3K2/2R5 b - - 0 1}
\mainline[level=1]{1... Nc5 $1 }
 The pin allows this fork. 

\mainline[level=1]{ 2. dxc5 Rxd3 $17 }
 * (\blackWins, 36 moves) 




\clearpage 
\newpage 

\section{177.??.??}
\fenboard{8/1R3pbp/4p1k1/5p2/8/2N2P2/1P1r1P1P/4K3 b - - 0 1}
\begin{center}
\showinverseboard 
\end{center}
\clearpage 
\newpage 
\mainline{1... Rxb2 { Picking up a pawn with a small tactic gives Black a winning endgame. } 2. Rxb2 Bxc3+ $18 { * }}\fenboard{8/1R3pbp/4p1k1/5p2/8/2N2P2/1P1r1P1P/4K3 b - - 0 1}
\begin{center}
\showinverseboard 
\end{center}
\fenboard{8/1R3pbp/4p1k1/5p2/8/2N2P2/1P1r1P1P/4K3 b - - 0 1}
\mainline[level=1]{1... Rxb2 }
 Picking up a pawn with a small tactic gives Black a winning endgame. 

\mainline[level=1]{ 2. Rxb2 }

\variation[level=2]{ \xskakcomment{ Instead, White tried to fight with }} \variation[level=2]{ 2. Nb5 $19 \xskakcomment{ but in the end it proved fruitless (\blackWins, 53 moves). }} 

\mainline[level=1]{ 2... Bxc3+ $18 }
 * 




\clearpage 
\newpage 

\section{178.??.??}
\fenboard{5k1r/ppq3b1/8/3Pp2p/3n4/7Q/PP3PPP/3RR1K1 w - - 0 1}
\begin{center}
\showboard 
\end{center}
\clearpage 
\newpage 
\mainline{1. Rxd4 $1 { Black resigned in view of: } 1... exd4 2. Qf5+ Qf7 3. Qc8+ Qe8 4. Qxe8# { mate * }}\fenboard{5k1r/ppq3b1/8/3Pp2p/3n4/7Q/PP3PPP/3RR1K1 w - - 0 1}
\begin{center}
\showboard 
\end{center}
\fenboard{5k1r/ppq3b1/8/3Pp2p/3n4/7Q/PP3PPP/3RR1K1 w - - 0 1}

 White is obviously much better, but cleanest is: 

\mainline[level=1]{ 1. Rxd4 $1 }
 Black resigned in view of: 

\mainline[level=1]{ 1... exd4 2. Qf5+ Qf7 3. Qc8+ Qe8 4. Qxe8# }
 mate * 




\clearpage 
\newpage 

\section{179.??.??}
\fenboard{r3qn1k/pp3pp1/2b1p2p/4P1BP/2B3Q1/8/PPP3P1/2K1R3 w - - 0 1}
\begin{center}
\showboard 
\end{center}
\clearpage 
\newpage 
\mainline{1. Bf6 $1 { The king is too exposed after: } 1... gxf6 2. exf6 { * }}\fenboard{r3qn1k/pp3pp1/2b1p2p/4P1BP/2B3Q1/8/PPP3P1/2K1R3 w - - 0 1}
\begin{center}
\showboard 
\end{center}
\fenboard{r3qn1k/pp3pp1/2b1p2p/4P1BP/2B3Q1/8/PPP3P1/2K1R3 w - - 0 1}
\mainline[level=1]{1. Bf6 $1 }
 The king is too exposed after: 

\mainline[level=1]{ 1... gxf6 2. exf6 }
 * 




\clearpage 
\newpage 

\section{180.??.??}
\fenboard{3r2k1/1b3p1p/pp4p1/4PqP1/3NpP2/1P2Q2P/P6K/2R5 b - - 0 1}
\begin{center}
\showinverseboard 
\end{center}
\clearpage 
\newpage 
\mainline{1... Rxd4 $1 $19 { * Black wins back the rook on c1, so he has just won a piece,
and will break through easily. }}\fenboard{3r2k1/1b3p1p/pp4p1/4PqP1/3NpP2/1P2Q2P/P6K/2R5 b - - 0 1}
\begin{center}
\showinverseboard 
\end{center}
\fenboard{3r2k1/1b3p1p/pp4p1/4PqP1/3NpP2/1P2Q2P/P6K/2R5 b - - 0 1}
\mainline[level=1]{1... Rxd4 $1 $19 }
 * Black wins back the rook on c1, so he has just won a piece,
and will break through easily. 




\clearpage 
\newpage 

\section{181.??.??}
\fenboard{4r1rk/p3q2p/3p1n1Q/3P4/Pp1n1P2/3BN2R/1PPK1P2/R7 w - - 0 1}
\begin{center}
\showboard 
\end{center}
\clearpage 
\newpage 
\mainline{1. Qxf6+ $1 { Other moves take longer to win. } 1... Qxf6 2. Rxh7# { mate * }}\fenboard{4r1rk/p3q2p/3p1n1Q/3P4/Pp1n1P2/3BN2R/1PPK1P2/R7 w - - 0 1}
\begin{center}
\showboard 
\end{center}
\fenboard{4r1rk/p3q2p/3p1n1Q/3P4/Pp1n1P2/3BN2R/1PPK1P2/R7 w - - 0 1}
\mainline[level=1]{1. Qxf6+ $1 }
 Other moves take longer to win. 

\mainline[level=1]{ 1... Qxf6 2. Rxh7# }
 mate * 




\clearpage 
\newpage 

\section{182.??.??}
\fenboard{2kr3r/1pq3p1/p1p3Qp/P1n1p3/8/RbN1P1P1/1P2P1BP/5RK1 w - - 0 1}
\begin{center}
\showboard 
\end{center}
\clearpage 
\newpage 
\mainline{1. Rxb3 $1 { Black resigned since } 1... Nxb3 2. Qe6+ Kb8 3. Qxb3 $18 { * is winning, although it wouldn't have hurt to fight on. }}\fenboard{2kr3r/1pq3p1/p1p3Qp/P1n1p3/8/RbN1P1P1/1P2P1BP/5RK1 w - - 0 1}
\begin{center}
\showboard 
\end{center}
\fenboard{2kr3r/1pq3p1/p1p3Qp/P1n1p3/8/RbN1P1P1/1P2P1BP/5RK1 w - - 0 1}
\mainline[level=1]{1. Rxb3 $1 }
 Black resigned since 

\mainline[level=1]{ 1... Nxb3 2. Qe6+ Kb8 3. Qxb3 $18 }
 * is winning, although it wouldn't have hurt to fight on. 




\clearpage 
\newpage 

\section{183.??.??}
\fenboard{2rq1rk1/1p4bp/4p1p1/p2p1bP1/P2N4/2PnB3/QP4BP/R5RK b - - 0 1}
\begin{center}
\showinverseboard 
\end{center}
\clearpage 
\newpage 
\mainline{1... Qxg5 $1 { Picking up this pawn increases the advantage considerably. } 2. Bxg5 Nf2# { mate * }}\fenboard{2rq1rk1/1p4bp/4p1p1/p2p1bP1/P2N4/2PnB3/QP4BP/R5RK b - - 0 1}
\begin{center}
\showinverseboard 
\end{center}
\fenboard{2rq1rk1/1p4bp/4p1p1/p2p1bP1/P2N4/2PnB3/QP4BP/R5RK b - - 0 1}
\mainline[level=1]{1... Qxg5 $1 }
 Picking up this pawn increases the advantage considerably. 

\mainline[level=1]{ 2. Bxg5 }

\variation[level=2]{ \xskakcomment{ White tried to fight on with }} \variation[level=2]{ 2. Rge1 \xskakcomment{ but resigned a few moves later. }} 

\mainline[level=1]{ 2... Nf2# }
 mate * 




\clearpage 
\newpage 

\section{184.??.??}
\fenboard{2k4r/pp3p2/1np3q1/2Q3p1/3R4/1P2P1P1/P4PB1/6K1 w - - 0 1}
\begin{center}
\showboard 
\end{center}
\clearpage 
\newpage 
\mainline{1. Bh3+ $1 { * 1-0 If the king moves, Qe5+ picks up the rook on h8 (and mates).
If the bishop is taken then Qf8+ mates. And finally, anything put in the way
will just be taken. }}\fenboard{2k4r/pp3p2/1np3q1/2Q3p1/3R4/1P2P1P1/P4PB1/6K1 w - - 0 1}
\begin{center}
\showboard 
\end{center}
\fenboard{2k4r/pp3p2/1np3q1/2Q3p1/3R4/1P2P1P1/P4PB1/6K1 w - - 0 1}
\mainline[level=1]{1. Bh3+ $1 }
 * \whiteWins If the king moves, Qe5+ picks up the rook on h8 (and mates).
If the bishop is taken then Qf8+ mates. And finally, anything put in the way
will just be taken. 

\variation[level=2]{ \xskakcomment{ Not }} \variation[level=2]{ 1. Rd6 $2 Qb1+ 2. Bf1 Qh7 3. Bg2 Qb1+ \xskakcomment{ = }} 




\clearpage 
\newpage 

\section{185.??.??}
\fenboard{r1b2rk1/p1q2n1p/6p1/2Ppp3/8/2Q2NPP/P4PB1/1R3RK1 w - - 0 1}
\begin{center}
\showboard 
\end{center}
\clearpage 
\newpage 
\mainline{1. Nxe5 $1 Nxe5 2. Bxd5+ $18 { * Winning the rook on a8. }}\fenboard{r1b2rk1/p1q2n1p/6p1/2Ppp3/8/2Q2NPP/P4PB1/1R3RK1 w - - 0 1}
\begin{center}
\showboard 
\end{center}
\fenboard{r1b2rk1/p1q2n1p/6p1/2Ppp3/8/2Q2NPP/P4PB1/1R3RK1 w - - 0 1}
\mainline[level=1]{1. Nxe5 $1 Nxe5 2. Bxd5+ $18 }
 * Winning the rook on a8. 




\clearpage 
\newpage 

\section{186.??.??}
\fenboard{1r6/RP5p/P1kp4/2n2p2/8/4PKP1/5P1P/8 w - - 0 1}
\begin{center}
\showboard 
\end{center}
\clearpage 
\newpage 
\mainline{1. Ra8 $1 Kc7 2. a7 Rxb7 3. Rc8+ $1 { * 1-0 The pawn promotes. }}\fenboard{1r6/RP5p/P1kp4/2n2p2/8/4PKP1/5P1P/8 w - - 0 1}
\begin{center}
\showboard 
\end{center}
\fenboard{1r6/RP5p/P1kp4/2n2p2/8/4PKP1/5P1P/8 w - - 0 1}
\mainline[level=1]{1. Ra8 $1 Kc7 }

\variation[level=2]{ 1... Nxa6 2. Rxa6+ \xskakcomment{ leaves White with an easily winning endgame. }} 

\mainline[level=1]{ 2. a7 Rxb7 3. Rc8+ $1 }
 * \whiteWins The pawn promotes. 




\clearpage 
\newpage 

\section{187.??.??}
\fenboard{r2qk2r/3bb1pp/5p2/1p1Q4/1pNP4/4P3/PB4PP/R4RK1 w kq - 0 1}
\begin{center}
\showboard 
\end{center}
\clearpage 
\newpage 
\mainline{1. Ne5 $1 Rf8 2. Nxd7 $1 $18 { * Black can't take back since it would leave the rook on a8 unprotected (1-0,
23 moves). }}\fenboard{r2qk2r/3bb1pp/5p2/1p1Q4/1pNP4/4P3/PB4PP/R4RK1 w kq - 0 1}
\begin{center}
\showboard 
\end{center}
\fenboard{r2qk2r/3bb1pp/5p2/1p1Q4/1pNP4/4P3/PB4PP/R4RK1 w kq - 0 1}

 The knight seems to be forced back, but can instead go forward to seemingly
protected squares. 

\mainline[level=1]{ 1. Ne5 $1 Rf8 }

\variation[level=2]{ 1... fxe5 $2 2. Qf7# \xskakcomment{ mate }} 

\mainline[level=1]{ 2. Nxd7 $1 $18 }
 * Black can't take back since it would leave the rook on a8 unprotected (\whiteWins,
23 moves). 




\clearpage 
\newpage 

\section{188.??.??}
\fenboard{r3k1r1/p4p1p/3bpBp1/3p1b2/5P2/3BP2P/PP4P1/1K1R3R w q - 0 1}
\begin{center}
\showboard 
\end{center}
\clearpage 
\newpage 
\mainline{1. e4 $1 { The bishop on d6 becomes exposed. } 1... Bxe4 2. Bxe4 $18 { * }}\fenboard{r3k1r1/p4p1p/3bpBp1/3p1b2/5P2/3BP2P/PP4P1/1K1R3R w q - 0 1}
\begin{center}
\showboard 
\end{center}
\fenboard{r3k1r1/p4p1p/3bpBp1/3p1b2/5P2/3BP2P/PP4P1/1K1R3R w q - 0 1}
\mainline[level=1]{1. e4 $1 }
 The bishop on d6 becomes exposed. 

\mainline[level=1]{ 1... Bxe4 }

\variation[level=2]{ 1... dxe4 2. Bb5+ Kf8 3. Rxd6 $18 \xskakcomment{ * }} 

\mainline[level=1]{ 2. Bxe4 $18 }
 * 




\clearpage 
\newpage 

\section{189.??.??}
\fenboard{4r2k/6b1/8/p2R3p/1pP1p1Nq/8/PP3P1P/3Q3K w - - 0 1}
\begin{center}
\showboard 
\end{center}
\clearpage 
\newpage 
\mainline{1. Nf6 $1 { 1-0 With winning threats on h5. } 1... Qxf6 2. Qxh5+ Kg8 3. Qxe8+ $18 { * }}\fenboard{4r2k/6b1/8/p2R3p/1pP1p1Nq/8/PP3P1P/3Q3K w - - 0 1}
\begin{center}
\showboard 
\end{center}
\fenboard{4r2k/6b1/8/p2R3p/1pP1p1Nq/8/PP3P1P/3Q3K w - - 0 1}
\mainline[level=1]{1. Nf6 $1 }
 \whiteWins With winning threats on h5. 

\variation[level=2]{ \xskakcomment{ But not }} \variation[level=2]{ 1. Ne3 $6 Be5 $16 }

\mainline[level=1]{ 1... Qxf6 }

\variation[level=2]{ 1... Bxf6 2. Rxh5+ $18 }

\mainline[level=1]{ 2. Qxh5+ Kg8 3. Qxe8+ $18 }
 * 




\clearpage 
\newpage 

\section{190.??.??}
\fenboard{2r4k/2q4p/p2bPp2/1p6/6QP/3B4/PPP5/1K4R1 w - - 0 1}
\begin{center}
\showboard 
\end{center}
\clearpage 
\newpage 
\mainline{1. e7 $1 { * 1-0 The only defence against Qg7 mate leaves the rook on c8 undefended. }}\fenboard{2r4k/2q4p/p2bPp2/1p6/6QP/3B4/PPP5/1K4R1 w - - 0 1}
\begin{center}
\showboard 
\end{center}
\fenboard{2r4k/2q4p/p2bPp2/1p6/6QP/3B4/PPP5/1K4R1 w - - 0 1}
\mainline[level=1]{1. e7 $1 }
 * \whiteWins The only defence against Qg7 mate leaves the rook on c8 undefended. 




\clearpage 
\newpage 

\section{191.??.??}
\fenboard{3n4/2p2pk1/3r2p1/6Np/1P1q3P/4QR2/5PP1/6K1 w - - 0 1}
\begin{center}
\showboard 
\end{center}
\clearpage 
\newpage 
\mainline{1. Rxf7+ $1 { 1-0 The variation goes: } 1... Nxf7 2. Ne6+ Rxe6 3. Qxd4+ $18 { * }}\fenboard{3n4/2p2pk1/3r2p1/6Np/1P1q3P/4QR2/5PP1/6K1 w - - 0 1}
\begin{center}
\showboard 
\end{center}
\fenboard{3n4/2p2pk1/3r2p1/6Np/1P1q3P/4QR2/5PP1/6K1 w - - 0 1}
\mainline[level=1]{1. Rxf7+ $1 }
 \whiteWins The variation goes: 

\mainline[level=1]{ 1... Nxf7 2. Ne6+ Rxe6 3. Qxd4+ $18 }
 * 




\clearpage 
\newpage 

\section{192.??.??}
\fenboard{r1q2rk1/3bpp1p/3p2p1/3P4/Pnn1P3/2N2NP1/1R1Q1PB1/4R1K1 w - - 0 1}
\begin{center}
\showboard 
\end{center}
\clearpage 
\newpage 
\mainline{1. Qh6 $1 { White removes the queen from danger with a mating threat, leaving
the knight on b4 to face the gallows. } 1... f6 2. Rxb4 $18 { * }}\fenboard{r1q2rk1/3bpp1p/3p2p1/3P4/Pnn1P3/2N2NP1/1R1Q1PB1/4R1K1 w - - 0 1}
\begin{center}
\showboard 
\end{center}
\fenboard{r1q2rk1/3bpp1p/3p2p1/3P4/Pnn1P3/2N2NP1/1R1Q1PB1/4R1K1 w - - 0 1}
\mainline[level=1]{1. Qh6 $1 }
 White removes the queen from danger with a mating threat, leaving
the knight on b4 to face the gallows. 

\mainline[level=1]{ 1... f6 }

\variation[level=2]{ 1... Nxb2 2. Ng5 $18 \xskakcomment{ * with mate. }} 

\mainline[level=1]{ 2. Rxb4 $18 }
 * 




\clearpage 
\newpage 

\section{193.??.??}
\fenboard{2r2rk1/p4ppp/1p6/2b5/5Pn1/1Q2PR1P/P5P1/R1B1qBK1 b - - 0 22}
\begin{center}
\showinverseboard 
\end{center}
\clearpage 
\newpage 
\mainline{22... Nxe3 $1 { * } 23. Rxe3 Rfe8}\fenboard{2r2rk1/p4ppp/1p6/2b5/5Pn1/1Q2PR1P/P5P1/R1B1qBK1 b - - 0 22}
\begin{center}
\showinverseboard 
\end{center}
\fenboard{2r2rk1/p4ppp/1p6/2b5/5Pn1/1Q2PR1P/P5P1/R1B1qBK1 b - - 0 22}
\mainline[level=1]{22... Nxe3 $1 }
 * 

\variation[level=2]{ 22... Bxe3+ $2 23. Bxe3 \xskakcomment{ wins for White. }} 

\mainline[level=1]{ 23. Rxe3 }

\variation[level=2]{ \xskakcomment{ Other moves are no better: }} \variation[level=2]{ 23. Bxe3 Qxa1 $19 } (\variation[level=3]{ \xskakcomment{ or }} \variation[level=3]{ 23... Bxe3+ 24. Qxe3 Qxa1 $19 \xskakcomment{ White is so tied up that basically any move wins by eventually bringing
one rook into the action. }})




\clearpage 
\newpage 

\section{194.??.??}
\fenboard{5r1k/2p5/5qp1/1Pb1p2p/4B2P/3P2P1/4QPK1/5R2 w - - 0 1}
\begin{center}
\showboard 
\end{center}
\clearpage 
\newpage 
\mainline{1. Bxg6 $1 { Winning two more pawns, bringing the total to an overwhelming three. } 1... Qxg6 2. Qxe5+ Kg8 3. Qxc5 { * }}\fenboard{5r1k/2p5/5qp1/1Pb1p2p/4B2P/3P2P1/4QPK1/5R2 w - - 0 1}
\begin{center}
\showboard 
\end{center}
\fenboard{5r1k/2p5/5qp1/1Pb1p2p/4B2P/3P2P1/4QPK1/5R2 w - - 0 1}
\mainline[level=1]{1. Bxg6 $1 }
 Winning two more pawns, bringing the total to an overwhelming three. 

\mainline[level=1]{ 1... Qxg6 2. Qxe5+ Kg8 3. Qxc5 }
 * 




\clearpage 
\newpage 

\section{195.??.??}
\fenboard{2kB2r1/p3R3/q1p3r1/bp2Qp2/2PP3p/1P4P1/P4P2/5K2 w - - 0 1}
\begin{center}
\showboard 
\end{center}
\clearpage 
\newpage 
\mainline{1. Rc7+ $1 Kb8 2. Rxc6+ { * }}\fenboard{2kB2r1/p3R3/q1p3r1/bp2Qp2/2PP3p/1P4P1/P4P2/5K2 w - - 0 1}
\begin{center}
\showboard 
\end{center}
\fenboard{2kB2r1/p3R3/q1p3r1/bp2Qp2/2PP3p/1P4P1/P4P2/5K2 w - - 0 1}
\mainline[level=1]{1. Rc7+ $1 }

\variation[level=2]{ 1. Qxf5+ Kb8 2. Bxa5 Ka8 $1 $13 }

\mainline[level=1]{ 1... Kb8 }

\variation[level=2]{\xskakcomment{\noindent\textbf{a)} } 1... Kxd8 2. Qe7# \xskakcomment{ mate * }} 

\variation[level=2]{\xskakcomment{\noindent\textbf{b)} } \xskakcomment{ or }} \variation[level=2]{ 1... Bxc7 2. Qxc7# \xskakcomment{ mate * }} 

\mainline[level=1]{ 2. Rxc6+ }
 * 




\clearpage 
\newpage 

\section{196.??.??}
\fenboard{1r3k2/2p3Rp/2b5/p1Pp2p1/P2B1pq1/2PB2P1/7P/5RK1 w - - 0 1}
\begin{center}
\showboard 
\end{center}
\clearpage 
\newpage 
\mainline{1. Rxf4+ $1 { * 1-0 Winning the queen. }}\fenboard{1r3k2/2p3Rp/2b5/p1Pp2p1/P2B1pq1/2PB2P1/7P/5RK1 w - - 0 1}
\begin{center}
\showboard 
\end{center}
\fenboard{1r3k2/2p3Rp/2b5/p1Pp2p1/P2B1pq1/2PB2P1/7P/5RK1 w - - 0 1}
\mainline[level=1]{1. Rxf4+ $1 }
 * \whiteWins Winning the queen. 

\variation[level=2]{\xskakcomment{\noindent\textbf{a)} } 1. Bxh7 \xskakcomment{ keeps the threat of Rxf4+, but Black can fight on with }} \variation[level=2]{ 1... Qh3 $16 }

\variation[level=2]{\xskakcomment{\noindent\textbf{b)} } 1. Rxh7 \xskakcomment{ allows }} \variation[level=2]{ 1... Rb2 $1 }




\clearpage 
\newpage 

\section{197.??.??}
\fenboard{6k1/5rpp/4Qn2/3p4/1q1P4/1p2P2P/6P1/3B1RK1 w - - 0 1}
\begin{center}
\showboard 
\end{center}
\clearpage 
\newpage 
\mainline{1. Bh5 $1 { Exploiting all the pins! } 1... g6 2. Rxf6 { *  In the game, Black resigned in a few moves: } 2... Qb7 3. Bd1 b2 4. Qe8+ Kg7 5. Qxf7+}\fenboard{6k1/5rpp/4Qn2/3p4/1q1P4/1p2P2P/6P1/3B1RK1 w - - 0 1}
\begin{center}
\showboard 
\end{center}
\fenboard{6k1/5rpp/4Qn2/3p4/1q1P4/1p2P2P/6P1/3B1RK1 w - - 0 1}
\mainline[level=1]{1. Bh5 $1 }
 Exploiting all the pins! 

\variation[level=2]{ 1. Rxf6 $5 gxf6 2. Bh5 Qb7 3. Bxf7+ Qxf7 4. Qb6 $16 \xskakcomment{ gives Black good drawing chances in a queen ending a pawn down. }} 

\mainline[level=1]{ 1... g6 }

\variation[level=2]{\xskakcomment{\noindent\textbf{a)} } 1... Nxh5 2. Qxf7+ \xskakcomment{ * }} \variation[level=2]{ 2... Kh8 3. Qe8+ Qf8 4. Rxf8# \xskakcomment{ mate }} 

\variation[level=2]{\xskakcomment{\noindent\textbf{b)} } \xskakcomment{ or }} \variation[level=2]{ 1... Qf8 2. Bxf7+ $18 }

\mainline[level=1]{ 2. Rxf6 }
 * In the game, Black resigned in a few moves: 

\variation[level=2]{ \xskakcomment{ Or }} \variation[level=2]{ 2. Bxg6 hxg6 3. Rxf6 $18 }




\clearpage 
\newpage 

\section{198.??.??}
\fenboard{6rk/7p/8/1p3R2/pN6/P1n2B1P/4brPK/4R3 b - - 0 1}
\begin{center}
\showinverseboard 
\end{center}
\clearpage 
\newpage 
\mainline{1... Bxf3 $1 { Winning the g2-pawn and an exchange. } 2. Rxf3 Rgxg2+ 3. Kh1 Rh2+ 4. Kg1 Ne2+ { * }}\fenboard{6rk/7p/8/1p3R2/pN6/P1n2B1P/4brPK/4R3 b - - 0 1}
\begin{center}
\showinverseboard 
\end{center}
\fenboard{6rk/7p/8/1p3R2/pN6/P1n2B1P/4brPK/4R3 b - - 0 1}
\mainline[level=1]{1... Bxf3 $1 }
 Winning the g2-pawn and an exchange. 

\variation[level=2]{ 1... Rgxg2+ 2. Bxg2 Rxf5 \xskakcomment{ is too kind: }} \variation[level=2]{ 3. Rc1 $15 }

\mainline[level=1]{ 2. Rxf3 Rgxg2+ 3. Kh1 Rh2+ 4. Kg1 Ne2+ }
 * 




\clearpage 
\newpage 

\section{199.??.??}
\fenboard{2r3k1/p2b2pp/1pqr4/2p1R3/4P3/P1Q3PP/1P4BK/3R4 w - - 0 1}
\begin{center}
\showboard 
\end{center}
\clearpage 
\newpage 
\mainline{1. Re7 $1 $18 { * Threatening mate, and both defences end up losing the bishop. } 1... Rd4 2. Rxd4 cxd4 3. Qxd4 { 1-0 Double attack. }}\fenboard{2r3k1/p2b2pp/1pqr4/2p1R3/4P3/P1Q3PP/1P4BK/3R4 w - - 0 1}
\begin{center}
\showboard 
\end{center}
\fenboard{2r3k1/p2b2pp/1pqr4/2p1R3/4P3/P1Q3PP/1P4BK/3R4 w - - 0 1}
\mainline[level=1]{1. Re7 $1 $18 }
 * Threatening mate, and both defences end up losing the bishop. 

\mainline[level=1]{ 1... Rd4 }

\variation[level=2]{ 1... Rg6 2. Rdxd7 $18 }

\mainline[level=1]{ 2. Rxd4 cxd4 3. Qxd4 }
 \whiteWins Double attack. 




\clearpage 
\newpage 

\section{200.??.??}
\fenboard{r3r1k1/p2q1pbp/6p1/3Q2B1/8/5P2/PP4PP/R2R2K1 b - - 0 1}
\begin{center}
\showinverseboard 
\end{center}
\clearpage 
\newpage 
\mainline{1... Re1+ $1 { Deflection. } 2. Rxe1 Qxd5 $19 { * }}\fenboard{r3r1k1/p2q1pbp/6p1/3Q2B1/8/5P2/PP4PP/R2R2K1 b - - 0 1}
\begin{center}
\showinverseboard 
\end{center}
\fenboard{r3r1k1/p2q1pbp/6p1/3Q2B1/8/5P2/PP4PP/R2R2K1 b - - 0 1}
\mainline[level=1]{1... Re1+ $1 }
 Deflection. 

\mainline[level=1]{ 2. Rxe1 }

\variation[level=2]{ 2. Kf2 Qxd5 3. Rxd5 Rxa1 $19 \xskakcomment{ * }} 

\mainline[level=1]{ 2... Qxd5 $19 }
 * 




\clearpage 
\newpage 

\section{201.??.??}
\fenboard{3r2k1/pp4qn/3p1ppQ/2pPp3/4P1N1/1P2P2R/1PP3PP/6K1 w - - 0 1}
\begin{center}
\showboard 
\end{center}
\clearpage 
\newpage 
\mainline{1. Qxh7+ Qxh7 2. Nxf6+ $18 { * 1-0 White wins a piece. }}\fenboard{3r2k1/pp4qn/3p1ppQ/2pPp3/4P1N1/1P2P2R/1PP3PP/6K1 w - - 0 1}
\begin{center}
\showboard 
\end{center}
\fenboard{3r2k1/pp4qn/3p1ppQ/2pPp3/4P1N1/1P2P2R/1PP3PP/6K1 w - - 0 1}
\mainline[level=1]{1. Qxh7+ Qxh7 2. Nxf6+ $18 }
 * \whiteWins White wins a piece. 




\clearpage 
\newpage 

\section{202.??.??}
\fenboard{4r2k/p4rp1/1pb3qp/3NB3/2P1Q2n/1P1R4/1P4PP/4R1K1 b - - 0 1}
\begin{center}
\showinverseboard 
\end{center}
\clearpage 
\newpage 
\mainline{1... Rxe5 $1 { The white queen is doubly pinned and is needed to protect g2. } 2. Qxh4 Rxe1+ 3. Qxe1 Qxd3 { * }}\fenboard{4r2k/p4rp1/1pb3qp/3NB3/2P1Q2n/1P1R4/1P4PP/4R1K1 b - - 0 1}
\begin{center}
\showinverseboard 
\end{center}
\fenboard{4r2k/p4rp1/1pb3qp/3NB3/2P1Q2n/1P1R4/1P4PP/4R1K1 b - - 0 1}
\mainline[level=1]{1... Rxe5 $1 }
 The white queen is doubly pinned and is needed to protect g2. 

\mainline[level=1]{ 2. Qxh4 }

\variation[level=2]{\xskakcomment{\noindent\textbf{a)} } 2. Qxe5 Qxg2# \xskakcomment{ mate * }} (\variation[level=3]{ \xskakcomment{ or }} \variation[level=3]{ 2... Qxd3 $19 })

\variation[level=2]{\xskakcomment{\noindent\textbf{b)} } \xskakcomment{ and }} \variation[level=2]{ 2. Qxg6 Rxe1# \xskakcomment{ mate * }} (\variation[level=3]{ \xskakcomment{ or }} \variation[level=3]{ 2... Nxg6 $19 })

\mainline[level=1]{ 2... Rxe1+ }

\variation[level=2]{ \xskakcomment{ Or }} \variation[level=2]{ 2... Qxd3 $19 }

\mainline[level=1]{ 3. Qxe1 Qxd3 }
 * 




\clearpage 
\newpage 

\section{203.??.??}
\fenboard{2b3k1/r2r1pp1/p3p2p/1pq1P3/4B2Q/2P5/P4PPP/3RR1K1 w - - 0 1}
\begin{center}
\showboard 
\end{center}
\clearpage 
\newpage 
\mainline{1. Qd8+ $1 { An X-ray/reloader theme supported by the strongly-placed bishop on e4. } 1... Rxd8 2. Rxd8+ Qf8 3. Bh7+ $1 Kxh7 4. Rxf8 $16 { * (1-0, 30 moves) }}\fenboard{2b3k1/r2r1pp1/p3p2p/1pq1P3/4B2Q/2P5/P4PPP/3RR1K1 w - - 0 1}
\begin{center}
\showboard 
\end{center}
\fenboard{2b3k1/r2r1pp1/p3p2p/1pq1P3/4B2Q/2P5/P4PPP/3RR1K1 w - - 0 1}
\mainline[level=1]{1. Qd8+ $1 }
 An X-ray/reloader theme supported by the strongly-placed bishop on e4. 

\mainline[level=1]{ 1... Rxd8 }

\variation[level=2]{ 1... Qf8 2. Bh7+ $18 \xskakcomment{ * }} 

\mainline[level=1]{ 2. Rxd8+ Qf8 3. Bh7+ $1 }

\variation[level=2]{ \xskakcomment{ Or }} \variation[level=2]{ 3. Red1 Rc7 4. Bh7+ $16 }

\mainline[level=1]{ 3... Kxh7 4. Rxf8 $16 }
 * (\whiteWins, 30 moves) 




\clearpage 
\newpage 

\section{204.??.??}
\fenboard{2r5/R2b1pk1/3q1bpp/1p1Pp3/4P3/3P3P/1Q1NBPP1/6K1 w - - 0 1}
\begin{center}
\showboard 
\end{center}
\clearpage 
\newpage 
\mainline{1. Rxd7 $1 { 1-0 Winning another pawn and exchanging some pieces. } 1... Qxd7 2. Bg4 Qd8 3. Bxc8 Qxc8 4. Qxb5 $18 { * }}\fenboard{2r5/R2b1pk1/3q1bpp/1p1Pp3/4P3/3P3P/1Q1NBPP1/6K1 w - - 0 1}
\begin{center}
\showboard 
\end{center}
\fenboard{2r5/R2b1pk1/3q1bpp/1p1Pp3/4P3/3P3P/1Q1NBPP1/6K1 w - - 0 1}
\mainline[level=1]{1. Rxd7 $1 }
 \whiteWins Winning another pawn and exchanging some pieces. 

\variation[level=2]{ \xskakcomment{ Too kind is }} \variation[level=2]{ 1. Bg4 Bxg4 2. hxg4 $16 }

\mainline[level=1]{ 1... Qxd7 2. Bg4 Qd8 3. Bxc8 Qxc8 4. Qxb5 $18 }
 * 




\clearpage 
\newpage 

\section{205.??.??}
\fenboard{r6k/7p/1Qp2rpq/p3p3/2B1P3/2B2P1n/P5PK/1R6 w - - 0 1}
\begin{center}
\showboard 
\end{center}
\clearpage 
\newpage 
\mainline{1. Qb8+ $1 Rxb8 2. Rxb8+ Kg7 3. Rg8# { mate * }}\fenboard{r6k/7p/1Qp2rpq/p3p3/2B1P3/2B2P1n/P5PK/1R6 w - - 0 1}
\begin{center}
\showboard 
\end{center}
\fenboard{r6k/7p/1Qp2rpq/p3p3/2B1P3/2B2P1n/P5PK/1R6 w - - 0 1}
\mainline[level=1]{1. Qb8+ $1 }

\variation[level=2]{ \xskakcomment{ The breathing hole on g7 is not enough, neither is either of the
two possible blocks on f8. }} \variation[level=2]{ 1. Bxe5 $2 Ng5+ 2. Kg1 } (\variation[level=3]{ 2. Kg3 Nxe4+ 3. fxe4 Qg5+ $18 })
\variation[level=2]{ 2... Nxf3+ $18 }

\mainline[level=1]{ 1... Rxb8 2. Rxb8+ Kg7 }

\variation[level=2]{\xskakcomment{\noindent\textbf{a)} } 2... Rf8 3. Bxe5+ } (\variation[level=3]{ 3. Rxf8+ Qxf8 4. Bxe5+ $18 })
\variation[level=2]{ 3... Qg7 4. Rxf8# \xskakcomment{ mate * }} 

\variation[level=2]{\xskakcomment{\noindent\textbf{b)} } 2... Qf8 3. Rxf8+ Rxf8 4. Bxe5+ Rf6 5. Bxf6# \xskakcomment{ mate * }} 

\mainline[level=1]{ 3. Rg8# }
 mate * 




\clearpage 
\newpage 

\section{206.??.??}
\fenboard{6rk/p6n/3p4/2pPb2q/4N3/P2BQPp1/6K1/7R b - - 0 1}
\begin{center}
\showinverseboard 
\end{center}
\clearpage 
\newpage 
\mainline{1... Qxh1+ $1 { Forcing a winning endgame. } 2. Kxh1 g2+ 3. Kg1 Bd4 { * With his
al advantage and dangerous passed pawn, Black is winning. The game ended
swiftly: } 4. Qxd4+ cxd4 5. Nxd6 Ng5}\fenboard{6rk/p6n/3p4/2pPb2q/4N3/P2BQPp1/6K1/7R b - - 0 1}
\begin{center}
\showinverseboard 
\end{center}
\fenboard{6rk/p6n/3p4/2pPb2q/4N3/P2BQPp1/6K1/7R b - - 0 1}
\mainline[level=1]{1... Qxh1+ $1 }
 Forcing a winning endgame. 

\variation[level=2]{ \xskakcomment{ Not }} \variation[level=2]{ 1... Qh2+ 2. Rxh2 gxh2+ 3. Kh1 Rg1+ 4. Qxg1 hxg1=Q+ 5. Kxg1 $15 }

\mainline[level=1]{ 2. Kxh1 g2+ 3. Kg1 Bd4 }
 * With his
al advantage and dangerous passed pawn, Black is winning. The game ended
swiftly: 




\clearpage 
\newpage 

\section{207.??.??}
\fenboard{5rk1/p4qp1/2p4p/1p2Pn2/2pP2Q1/2P4P/P2B3K/5R2 b - - 0 1}
\begin{center}
\showinverseboard 
\end{center}
\clearpage 
\newpage 
\mainline{1... Ne3 $1 2. Bxe3 Qxf1 $19 { * (0-1, 34 moves) }}\fenboard{5rk1/p4qp1/2p4p/1p2Pn2/2pP2Q1/2P4P/P2B3K/5R2 b - - 0 1}
\begin{center}
\showinverseboard 
\end{center}
\fenboard{5rk1/p4qp1/2p4p/1p2Pn2/2pP2Q1/2P4P/P2B3K/5R2 b - - 0 1}
\mainline[level=1]{1... Ne3 $1 }

\variation[level=2]{ 1... b4 $6 $17 }

\mainline[level=1]{ 2. Bxe3 }

\variation[level=2]{ 2. Rxf7 Nxg4+ $18 \xskakcomment{ * }} 

\mainline[level=1]{ 2... Qxf1 $19 }
 * (\blackWins, 34 moves) 




\clearpage 
\newpage 

\section{208.??.??}
\fenboard{1r3qbk/1p2rpp1/p1p2n1p/PnN1NP2/1P2PQ2/3P1B1P/7K/1R4R1 w - - 0 1}
\begin{center}
\showboard 
\end{center}
\clearpage 
\newpage 
\mainline{1. Ne6 $1 $18 { White is clearly better after other moves, but this finishes the game. } 1... fxe6 2. Ng6+ Kh7 3. Nxf8+ $18 { * }}\fenboard{1r3qbk/1p2rpp1/p1p2n1p/PnN1NP2/1P2PQ2/3P1B1P/7K/1R4R1 w - - 0 1}
\begin{center}
\showboard 
\end{center}
\fenboard{1r3qbk/1p2rpp1/p1p2n1p/PnN1NP2/1P2PQ2/3P1B1P/7K/1R4R1 w - - 0 1}
\mainline[level=1]{1. Ne6 $1 $18 }
 White is clearly better after other moves, but this finishes the game. 

\mainline[level=1]{ 1... fxe6 }

\variation[level=2]{\xskakcomment{\noindent\textbf{a)} } \xskakcomment{ The game try }} \variation[level=2]{ 1... g5 \xskakcomment{ is plain hopeless; the game continued }} \variation[level=2]{ 2. Nxf8 } (\variation[level=3]{ 2. fxg6 \xskakcomment{ and other moves are winning as well }})
\variation[level=2]{ 2... gxf4 3. Nfg6+ $1 fxg6 4. Nxg6+ \xskakcomment{ and Black resigned. }} 

\variation[level=2]{\xskakcomment{\noindent\textbf{b)} } \xskakcomment{ No better is }} \variation[level=2]{ 1... Qc8 2. Rxg7 \xskakcomment{ * with mate. }} 

\mainline[level=1]{ 2. Ng6+ Kh7 3. Nxf8+ $18 }
 * 




\clearpage 
\newpage 

\section{209.??.??}
\fenboard{4r1k1/2BR3p/8/5Rp1/4n1K1/4P1P1/1P2P2P/5r2 b - - 0 1}
\begin{center}
\showinverseboard 
\end{center}
\clearpage 
\newpage 
\mainline{1... h5+ $1 { Deflecting the king from the defence of the rook. } 2. Kxh5 Rxf5 { * }}\fenboard{4r1k1/2BR3p/8/5Rp1/4n1K1/4P1P1/1P2P2P/5r2 b - - 0 1}
\begin{center}
\showinverseboard 
\end{center}
\fenboard{4r1k1/2BR3p/8/5Rp1/4n1K1/4P1P1/1P2P2P/5r2 b - - 0 1}
\mainline[level=1]{1... h5+ $1 }
 Deflecting the king from the defence of the rook. 

\variation[level=2]{ 1... Nf6+ $2 2. Rxf6 Rxf6 \xskakcomment{ and all the pawns make up for the exchange. }} 

\mainline[level=1]{ 2. Kxh5 Rxf5 }
 * 




\clearpage 
\newpage 

\section{210.??.??}
\fenboard{4rrk1/p2pB1pp/2p5/2P1R3/3QN3/Pq2n3/1P4PP/2KR4 w - - 0 1}
\begin{center}
\showboard 
\end{center}
\clearpage 
\newpage 
\mainline{1. Nf6+ $1 { * 1-0 Getting rid of the pesky knight on e3 with tempo, leaving
White totally winning. }}\fenboard{4rrk1/p2pB1pp/2p5/2P1R3/3QN3/Pq2n3/1P4PP/2KR4 w - - 0 1}
\begin{center}
\showboard 
\end{center}
\fenboard{4rrk1/p2pB1pp/2p5/2P1R3/3QN3/Pq2n3/1P4PP/2KR4 w - - 0 1}
\mainline[level=1]{1. Nf6+ $1 }
 * \whiteWins Getting rid of the pesky knight on e3 with tempo, leaving
White totally winning. 




\clearpage 
\newpage 

\section{211.??.??}
\fenboard{1n3q1k/r2r3p/p3Q3/1p6/8/2P3R1/P4PPP/4R1K1 w - - 0 1}
\begin{center}
\showboard 
\end{center}
\clearpage 
\newpage 
\mainline{1. Qf6+ $1 { Black resigned. } 1... Qxf6 2. Re8+ Qf8 3. Rxf8# { mate * }}\fenboard{1n3q1k/r2r3p/p3Q3/1p6/8/2P3R1/P4PPP/4R1K1 w - - 0 1}
\begin{center}
\showboard 
\end{center}
\fenboard{1n3q1k/r2r3p/p3Q3/1p6/8/2P3R1/P4PPP/4R1K1 w - - 0 1}
\mainline[level=1]{1. Qf6+ $1 }
 Black resigned. 

\mainline[level=1]{ 1... Qxf6 2. Re8+ Qf8 3. Rxf8# }
 mate * 




\clearpage 
\newpage 

\section{212.??.??}
\fenboard{1qr1r1k1/2b2ppp/Bpb1p3/4n3/P7/2N1BP2/1PP3PP/3RR1QK b - - 0 1}
\begin{center}
\showinverseboard 
\end{center}
\clearpage 
\newpage 
\mainline{1... Nxf3 $1 2. gxf3 Bxf3+ { * 0-1 Winning the queen. }}\fenboard{1qr1r1k1/2b2ppp/Bpb1p3/4n3/P7/2N1BP2/1PP3PP/3RR1QK b - - 0 1}
\begin{center}
\showinverseboard 
\end{center}
\fenboard{1qr1r1k1/2b2ppp/Bpb1p3/4n3/P7/2N1BP2/1PP3PP/3RR1QK b - - 0 1}
\mainline[level=1]{1... Nxf3 $1 2. gxf3 Bxf3+ }
 * \blackWins Winning the queen. 




\clearpage 
\newpage 

\section{213.??.??}
\fenboard{2rb2k1/5pp1/2p1pn1p/2P5/1N1Pp3/4P1BP/5PP1/1R4K1 w - - 0 1}
\begin{center}
\showboard 
\end{center}
\clearpage 
\newpage 
\mainline{1. Nxc6 $1 Rxc6 2. Rb8 Kh7 3. Rxd8 $18 { * }}\fenboard{2rb2k1/5pp1/2p1pn1p/2P5/1N1Pp3/4P1BP/5PP1/1R4K1 w - - 0 1}
\begin{center}
\showboard 
\end{center}
\fenboard{2rb2k1/5pp1/2p1pn1p/2P5/1N1Pp3/4P1BP/5PP1/1R4K1 w - - 0 1}

 White wins a pawn by exploiting the potential pin on the eighth rank, either
by: 

\mainline[level=1]{ 1. Nxc6 $1 }

\variation[level=2]{ \xskakcomment{ Or the almost equally good }} \variation[level=2]{ 1. Ra1 \xskakcomment{ threatening Ra1-a6, so }} \variation[level=2]{ 1... Nd5 2. Nxc6 \xskakcomment{ with the same theme but having given away ...Nf6-d5. }} 

\mainline[level=1]{ 1... Rxc6 2. Rb8 Kh7 3. Rxd8 $18 }
 * 




\clearpage 
\newpage 

\section{214.??.??}
\fenboard{5r1k/q3bQpp/2B5/1P2p3/4P3/p1P5/5PPP/5RK1 w - - 0 1}
\begin{center}
\showboard 
\end{center}
\clearpage 
\newpage 
\mainline{1. b6 $1 { 1-0 }}\fenboard{5r1k/q3bQpp/2B5/1P2p3/4P3/p1P5/5PPP/5RK1 w - - 0 1}
\begin{center}
\showboard 
\end{center}
\fenboard{5r1k/q3bQpp/2B5/1P2p3/4P3/p1P5/5PPP/5RK1 w - - 0 1}
\mainline[level=1]{1. b6 $1 }
 \whiteWins 

\variation[level=2]{\xskakcomment{\noindent\textbf{a)} } 1. b6 $1 \xskakcomment{ The pawn queens after }} \variation[level=2]{ 1... Rxf7 2. bxa7 $18 \xskakcomment{ * }} 

\variation[level=2]{\xskakcomment{\noindent\textbf{b)} } 1. b6 $1 \xskakcomment{ and the threat on the rook leaves White a piece up after }} \variation[level=2]{ 1... Qxb6 2. Qxe7 \xskakcomment{ * }} 




\clearpage 
\newpage 

\section{215.??.??}
\fenboard{2r3k1/4bpp1/4pn1p/1p6/q1r5/2BR1BP1/PQ2PPKP/3R4 b - - 0 1}
\begin{center}
\showinverseboard 
\end{center}
\clearpage 
\newpage 
\mainline{1... Rxc3 { 0-1 }}\fenboard{2r3k1/4bpp1/4pn1p/1p6/q1r5/2BR1BP1/PQ2PPKP/3R4 b - - 0 1}
\begin{center}
\showinverseboard 
\end{center}
\fenboard{2r3k1/4bpp1/4pn1p/1p6/q1r5/2BR1BP1/PQ2PPKP/3R4 b - - 0 1}
\mainline[level=1]{1... Rxc3 }
 \blackWins 

\variation[level=2]{ 1... Rxc3 \xskakcomment{ The rook on d3 is overloaded: }} \variation[level=2]{ 2. Rxc3 Rxc3 3. Qxc3 Qxd1 $19 \xskakcomment{ * }} 




\clearpage 
\newpage 

\section{216.??.??}
\fenboard{4r1k1/5p2/p5pp/P1p2q2/3Nr3/3Q4/5PPP/1RR3K1 b - - 0 1}
\begin{center}
\showinverseboard 
\end{center}
\clearpage 
\newpage 
\mainline{1... Re1+ { A discovered attack. } 2. Rxe1 Rxe1+ 3. Rxe1 Qxd3 $19 { * }}\fenboard{4r1k1/5p2/p5pp/P1p2q2/3Nr3/3Q4/5PPP/1RR3K1 b - - 0 1}
\begin{center}
\showinverseboard 
\end{center}
\fenboard{4r1k1/5p2/p5pp/P1p2q2/3Nr3/3Q4/5PPP/1RR3K1 b - - 0 1}
\mainline[level=1]{1... Re1+ }
 A discovered attack. 

\mainline[level=1]{ 2. Rxe1 Rxe1+ 3. Rxe1 Qxd3 $19 }
 * 




\clearpage 
\newpage 

\section{217.??.??}
\fenboard{2r3k1/5p2/p3bBp1/1p5p/4P3/1P4qP/P5P1/1B1RR2K b - - 0 1}
\begin{center}
\showinverseboard 
\end{center}
\clearpage 
\newpage 
\mainline{1... Bxh3 $1 { 0-1 }}\fenboard{2r3k1/5p2/p3bBp1/1p5p/4P3/1P4qP/P5P1/1B1RR2K b - - 0 1}
\begin{center}
\showinverseboard 
\end{center}
\fenboard{2r3k1/5p2/p3bBp1/1p5p/4P3/1P4qP/P5P1/1B1RR2K b - - 0 1}
\mainline[level=1]{1... Bxh3 $1 }
 \blackWins 

\variation[level=2]{ 1... Bxh3 $1 \xskakcomment{ Black is up too much material after }} \variation[level=2]{ 2. gxh3 Qxh3+ 3. Kg1 Qg3+ 4. Kh1 Qf3+ 5. Kg1 Qxf6 \xskakcomment{ * }} 




\clearpage 
\newpage 

\section{218.??.??}
\fenboard{1r5k/6pp/2p1r3/1R1p4/pn1P1q2/3Q1P2/PP3BPP/3R2K1 w - - 0 1}
\begin{center}
\showboard 
\end{center}
\clearpage 
\newpage 
\mainline{1. Qf5 $1 { * 1-0 Black is back-rank mated or loses the rook on e6. }}\fenboard{1r5k/6pp/2p1r3/1R1p4/pn1P1q2/3Q1P2/PP3BPP/3R2K1 w - - 0 1}
\begin{center}
\showboard 
\end{center}
\fenboard{1r5k/6pp/2p1r3/1R1p4/pn1P1q2/3Q1P2/PP3BPP/3R2K1 w - - 0 1}
\mainline[level=1]{1. Qf5 $1 }
 * \whiteWins Black is back-rank mated or loses the rook on e6. 




\clearpage 
\newpage 

\section{219.??.??}
\fenboard{r2r2k1/1bqp1ppp/ppnbpn2/8/2PNP3/P1N1BP2/1P2B1PP/2RQ1R1K w - - 0 1}
\begin{center}
\showboard 
\end{center}
\clearpage 
\newpage 
\mainline{1. Ndb5 { * 1-0 Winning a pawn, with the bishop pair and d6-square, gives a
decisive advantage. }}\fenboard{r2r2k1/1bqp1ppp/ppnbpn2/8/2PNP3/P1N1BP2/1P2B1PP/2RQ1R1K w - - 0 1}
\begin{center}
\showboard 
\end{center}
\fenboard{r2r2k1/1bqp1ppp/ppnbpn2/8/2PNP3/P1N1BP2/1P2B1PP/2RQ1R1K w - - 0 1}
\mainline[level=1]{1. Ndb5 }
 * \whiteWins Winning a pawn, with the bishop pair and d6-square, gives a
decisive advantage. 

\variation[level=2]{ \xskakcomment{ Somewhat weaker is winning the b6-pawn with: }} \variation[level=2]{ 1. Ncb5 Qb8 2. Nxd6 Qxd6 3. Nxc6 Qxc6 4. Qd4 $18 }




\clearpage 
\newpage 

\section{220.??.??}
\fenboard{2k1r3/p1pbbpp1/1p5p/2pNP3/2P3P1/1P3PBP/P5K1/3R4 w - - 0 1}
\begin{center}
\showboard 
\end{center}
\clearpage 
\newpage 
\mainline{1. Nxc7 $1 Kxc7 $2 2. e6+ $18 { * White's rook will penetrate to the seventh rank with devastating effect. }}\fenboard{2k1r3/p1pbbpp1/1p5p/2pNP3/2P3P1/1P3PBP/P5K1/3R4 w - - 0 1}
\begin{center}
\showboard 
\end{center}
\fenboard{2k1r3/p1pbbpp1/1p5p/2pNP3/2P3P1/1P3PBP/P5K1/3R4 w - - 0 1}
\mainline[level=1]{1. Nxc7 $1 Kxc7 $2 }

\variation[level=2]{ \xskakcomment{ Instead the game saw }} \variation[level=2]{ 1... Rd8 2. Nd5 $16 \xskakcomment{ when White had simply won a pawn, also stabilizing the knight on d5. }} 

\mainline[level=1]{ 2. e6+ $18 }
 * White's rook will penetrate to the seventh rank with devastating effect. 




\clearpage 
\newpage 

\section{221.??.??}
\fenboard{4r1qk/2p4n/1b3pQB/p1p5/2P1P3/1P1R3P/P5P1/6K1 w - - 0 1}
\begin{center}
\showboard 
\end{center}
\clearpage 
\newpage 
\mainline{1. Bg7+ $1 Qxg7 2. Qxe8+ { * With a winning advantage due to the two pawns,
Black's exposed king and the pawn-like bishop on b6. You don't have to see any
further. }}\fenboard{4r1qk/2p4n/1b3pQB/p1p5/2P1P3/1P1R3P/P5P1/6K1 w - - 0 1}
\begin{center}
\showboard 
\end{center}
\fenboard{4r1qk/2p4n/1b3pQB/p1p5/2P1P3/1P1R3P/P5P1/6K1 w - - 0 1}
\mainline[level=1]{1. Bg7+ $1 Qxg7 2. Qxe8+ }
 * With a winning advantage due to the two pawns,
Black's exposed king and the pawn-like bishop on b6. You don't have to see any
further. 




\clearpage 
\newpage 

\section{222.??.??}
\fenboard{2R5/4bppk/1p1p4/5R1P/4PQ2/5P2/r4q1P/7K w - - 0 1}
\begin{center}
\showboard 
\end{center}
\clearpage 
\newpage 
\mainline{1. Qh6+ { * 1-0 What a way to finish a World Championship (Magnus Carlsen -
Sergey Karjakin, New York (rapid 4) 2016)! It's mate in one however Black
takes back. }}\fenboard{2R5/4bppk/1p1p4/5R1P/4PQ2/5P2/r4q1P/7K w - - 0 1}
\begin{center}
\showboard 
\end{center}
\fenboard{2R5/4bppk/1p1p4/5R1P/4PQ2/5P2/r4q1P/7K w - - 0 1}
\mainline[level=1]{1. Qh6+ }
 * \whiteWins What a way to finish a World Championship (Magnus Carlsen -
Sergey Karjakin, New York (rapid 4) 2016)! It's mate in one however Black
takes back. 




\clearpage 
\newpage 

\end{document}
